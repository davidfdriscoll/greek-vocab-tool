\documentclass[13pt]{article}
\usepackage{fontspec}
\usepackage{polyglossia}
\usepackage{geometry}
\usepackage{setspace}
\usepackage{parskip}
\usepackage{titlesec}
\usepackage{multicol}
\usepackage{tabularx}
\usepackage{parcolumns}
\usepackage{microtype}

\setmainlanguage{english}
\setotherlanguage[variant=ancient]{greek}
\newfontfamily\greekfont[Script=Greek]{Linux Libertine O} % Pre-installed on Overleaf

\geometry{
  paperwidth=6in,
  paperheight=9in,
  top=0.5in,
  bottom=0.5in,
  left=0.3in,
  right=0.7in
}
\titleformat{\section}{\normalfont\large\bfseries}{}{0pt}{}

\newcommand{\vocabentry}[2]{\textbf{#1}: #2\vspace{0.0em}\\}
\newcommand{\glossitem}[2]{\textbf{\textgreek{#1}}: \textit{#2}}
% Greek poetry line macro with optional line number
\newcommand{\poemline}[2]{%
  \ifx&#1&%
    {\hspace{2em}} & {\Large\begin{greek}#2\end{greek}} \\
  \else
    {\small #1} & {\Large\begin{greek}#2\end{greek}} \\
  \fi
}

% For pure literal glosses
\newcommand{\glosstxt}[2]{\textbf{\textgreek{#1}}: \textit{#2}}

% For pure grammatical/explanatory glosses
\newcommand{\glossexpl}[2]{\textbf{\textgreek{#1}}: \textup{#2}}

% For mixed glosses (manual control of italics)
\newcommand{\glossmix}[2]{\textbf{\textgreek{#1}}: \textup{#2}}

% One full line of commentary: line number + multiple glosses
\newcommand{\glossline}[2]{%
  \colchunk[1]{\textbf{#1}}%
  \colchunk[2]{#2}%
  \colplacechunks
}


\date{}
\begin{document}
\Large
\begin{greek}
\begin{spacing}{1.5}

\noindent\textit{\MakeUppercase{\textls[200]{Ξανθίας}}}

\begin{tabularx}{\textwidth}{@{}lXr@{}}
  \phantom{Ξα.} & Εἴπω τι τῶν εἰωθότων, ὦ δέσποτα, & \\
  \phantom{Ξα.} & ἐφ’ οἷς ἀεὶ γελῶσιν οἱ θεώμενοι; & \\
\end{tabularx}

\vspace{1em}

\noindent\textit{\MakeUppercase{\textls[200]{Διόνυσος}}}

\begin{tabularx}{\textwidth}{@{}lXr@{}}
  & νὴ τὸν Δί’ ὅ τι βούλει γε, πλὴν “πιέζομαι”. & \\
  & τοῦτο δὲ φύλαξαι· πάνυ γάρ ἐστ’ ἤδη χολή. & \\
  \textit{Ξα.} & μηδ’ ἕτερον ἀστεῖόν τι; & \\
  \textit{Δι.} & \hspace*{10em}πλήν γ’ “ὡς θλίβομαι”. & 5 \\
  \textit{Ξα.} & τί δαί; τὸ πάνυ γέλοιον εἴπω; & \\
  \textit{Δι.} & \hspace*{12.5em}νὴ Δία & \\
  & θαρρῶν γε· μόνον ἐκεῖν’ ὅπως μὴ ’ρεῖς, & \\
  \textit{Ξα.} & \hspace*{16em}τὸ τί; & \\
  \textit{Δι.} & μεταβαλλόμενος τἀνάφορον ὅτι χεζητιᾷς. & \\
  \textit{Ξα.} & μηδ᾽ ὅτι τοσοῦτον ἄχθος ἐπ᾽ ἐμαυτῷ φέρων, & \\
  & εἰ μὴ καθαιρήσει τις, ἀποπαρδήσομαι; & 10 \\
  \textit{Δι.} & μὴ δῆθ᾽, ἱκετεύω, πλήν γ᾽ ὅταν μέλλω 'ξεμεῖν. & \\
  \textit{Ξα.} & τί δῆτ᾽ ἔδει με ταῦτα τὰ σκεύη φέρειν, & \\
  & εἴπερ ποιήσω μηδὲν ὧνπερ Φρύνιχος & \\
  & εἴωθε ποιεῖν καὶ Λύκις κἀμειψίας; & \\
  
\end{tabularx}

\end{spacing}

\newpage

\begin{multicols}{2}
\small % roughly 9pt
\vocabentry{ἄχθος, εος, τό}{a weight, burden, load}
\vocabentry{ἀεί}{(adv.) always, for ever}
\vocabentry{Ἀμειψίας, ὁ}{Ameipsias, comic poet}
\vocabentry{ἀνά-φορον, τό}{a pole}
\vocabentry{ἀπο-πέρδομαι}{fart}
\vocabentry{ἀστεῖος, ός, ά, όν}{of the town; urbane; witty}
\vocabentry{γέλοιος, ός, ά, όν}{causing laughter, laughable, funny}
\vocabentry{γελάω}{to laugh}
\vocabentry{δαί}{colloquial form of δή, used after interrogatives}
\vocabentry{δεῖ}{it is necessary}
\vocabentry{δεσπότης, ὁ}{a master, lord, the master of the house}
\vocabentry{δῆτα}{particle, more emphatic δή; τί δῆτα what then? μῆ δῆτα, just don't...}
\vocabentry{ἔθω}{be accustomed (see also εἴωθα)}
\vocabentry{ἐμαυτοῦ}{of me, of myself}
\vocabentry{ἐξ-εμέω}{to vomit forth, disgorge}
\vocabentry{ἕτερος, ός, ά, όν}{the one; the other (of two)}
\vocabentry{Ζεύς, ος, ὁ (acc. Δία)}{Zeus}
\vocabentry{θαρσέω}{to be of good courage, take courage}
\vocabentry{θεάομαι}{to look on, gaze at, view, behold}
\vocabentry{θλίβω}{to press, squeeze, pinch}
\vocabentry{καθ-αιρέω}{to take down (i.e. off one's shoulders)}
\vocabentry{Λύκις, ιδος, ἡ}{Lycis, a comic poet}
\vocabentry{μετα-βάλλω}{move over (prob. from one shoulder to the other)}
\vocabentry{νή}{(yes) by.., with acc.;  with γε 'yes indeed'}
\vocabentry{πιέζω}{to press, squeeze; oppress, distress}
\vocabentry{πλήν}{except}
\vocabentry{σκεῦος, εος, τό}{a vessel; bag, baggage}
\vocabentry{τοσοῦτος, ὁ}{so large, so tall}
\vocabentry{φυλάσσω}{to keep guard; (med.) avoid}
\vocabentry{χεζητιάω}{(<χέζω to shit) need to shit}
\vocabentry{χολή, ἡ}{gall, bile; (cause of) gall, bile}
\end{multicols}

\vspace{-1.5em}
\noindent\rule{\linewidth}{0.4pt}
\vspace{-2em}

\begin{multicols}{2}
\begin{parcolumns}[colwidths={1=1.5em, 2=0.9\linewidth}]{2}
\small

\glossline{}{%
  \glossmix{}{Two actors enter. One (Dionsyus) is dressed in a full-length yellow dress with a lion-skin and holds a club. The other (Xanthias), his slave, is sitting on a donkey and has a lot of luggage hanging from a pole over his shoulders. }
}

\glossline{1}{%
  \glossmix{}{"Master, should I say one of the usual things that the spectators always laugh at?"}
  \glossmix{Εἴπω}{Deliberative subj., \textit{should I say}}\\
  \glossmix{εἰωθότων}{< εἴωθα}%
}

\glossline{3}{%
  \glossmix{ὅ τι}{ὅς τις, whatever (you like)}\\
  
}

\glossline{4}{%
  \glossmix{φύλαξαι}{aor. impv. mid. φυλάσσω}\\
  \glossmix{πάνυ... ἐστ’ ἤδη χολή}{\textit{it's entirely by now a source of bile, makes me sick}}\\
}

\glossline{5}{%
    \glossmix{μηδ’}{μή at the start of a question expects a negative response.}\\
    \glossmix{ὡς}{exclamatory, \textit{how}}\\
}

\glossline{6}{%
    \glossmix{τὸ πάνυ γέλοιον}{\textit{the really funny [joke]}}\\
}

\glossline{7}{%
  \glossmix{θαρρῶν}{assimilated form of θαρσῶν.}\\
  \glossmix{ὅπως μὴ 'ρεῖς}{ὅπως + fut. indicative = impv. 'ρεῖς = ἐρεῖς < λέγω. \textit{Don't say}.}\\
  \glossmix{τὸ τί;}{\textit{the what [joke]?} Interrogatives can take an article when asking about an already mentioned object, Smyth 1186}\\  
}

\glossline{8}{%
    \glossmix{τἀνάφορον}{τὸ ἀνάφορον}\\
    \glossmix{ὅτι}{assume a verb of speaking. \textit{[saying] as you shift your pole that you need to shit}}
}

\glossline{10}{%
    \glossmix{καθαιρήσει}{fut. καθαιρέω}\\
    \glossmix{ἀποπαρδήσομαι}{fut. ἀποπέρδομαι. future most vivid condition with fut. ind. in protasis.}
}

\glossline{11}{%
    \glossmix{πλήν γ᾽ ὅταν μέλλω 'ξεμεῖν}{\textit{except whenever I'm going to puke}}
}

\glossline{13}{%
    \glossmix{μηδὲν ὧνπερ}{μηδὲν τουτῶν ἅπερ... Partitive genitive and assimilation.}\\
    \glossmix{Φρύνιχος... Λύκις... κἀμειψίας;}{Phrynichus, Lukis, and Ameipsias were three comic poets and competitors of Aristophanes. Phrynichus' \textit{Muses} took second place after \textit{Frogs} in 405.}\\
}

\glossline{14}{%
    \glossmix{κἀμειψίας}{καὶ Ἀμειψίας.}\\
}

\end{parcolumns}
\end{multicols}

\newpage

\begin{spacing}{1.5}

\begin{tabularx}{\textwidth}{@{}lXr@{}}
  \textit{Δι.} & μή νυν ποιήσῃς· ὡς ἐγὼ θεώμενος, & \\
  & ὅταν τι τούτων τῶν σοφισμάτων ἴδω, & \\
  & πλεῖν ἢ 'νιαυτῷ πρεσβύτερος ἀπέρχομαι. & \\
  \textit{Ξα.} & ὦ τρισκακοδαίμων ἄρ᾽ ὁ τράχηλος οὑτοσί, & \\
  & ὅτι θλίβεται μέν, τὸ δὲ γέλοιον οὐκ ἐρεῖ. & 20 \\
  \textit{Δι.} & εἶτ᾽ οὐχ ὕβρις ταῦτ᾽ ἐστὶ καὶ πολλὴ τρυφή, & \\
  & ὅτ᾽ ἐγὼ μὲν ὢν Διόνυσος υἱὸς Σταμνίου & \\
  & αὐτὸς βαδίζω καὶ πονῶ, τοῦτον δ᾽ ὀχῶ, & \\ 
  & ἵνα μὴ ταλαιπωροῖτο μηδ᾽ ἄχθος φέροι; & \\
  \textit{Ξα.} & οὐ γὰρ φέρω 'γώ; & \\
  \textit{Δι.} & \hspace*{7.5em}πῶς φέρεις γὰρ ὅς γ᾽ ὀχεῖ; & 25 \\
  \textit{Ξα.} & φέρων γε ταυτί. & \\
  \textit{Δι.} & \hspace*{6.5em}τίνα τρόπον; & \\
  \textit{Ξα.} & \hspace*{12em}βαρέως πάνυ. & \\
  \textit{Δι.} & οὔκουν τὸ βάρος τοῦθ᾽ ὃ σὺ φέρεις ὄνος φέρει; & \\
  \textit{Ξα.} & οὐ δῆθ᾽ ὅ γ᾽ ἔχω 'γὼ καὶ φέρω μὰ τὸν Δί᾽ οὔ. & \\
  \textit{Δι.} & πῶς γὰρ φέρεις, ὅς γ᾽ αὐτὸς ὑφ᾽ ἑτέρου φέρει; & \\
  \textit{Ξα.} & οὐκ οἶδ᾽: ὁ δ᾽ ὦμος οὑτοσὶ πιέζεται. & 30 \\  
\end{tabularx}

\end{spacing}

\newpage

\begin{multicols}{2}
\small % roughly 9pt
\vocabentry{ἄχθος, εος, τό}{a weight, burden, load}
\vocabentry{ἀπέρχομαι}{to go away, depart from}
\vocabentry{βάρος, εος, τό}{weight}
\vocabentry{βαρέως}{(adv.) heavily,  < βαρύς}
\vocabentry{ἐνιαυτός, ὁ}{year}
\vocabentry{θλίβω}{to press, squeeze, pinch}
\vocabentry{ὄνος, ὁ, ἡ}{an ass, donkey}
\vocabentry{οὔκουν}{certainly not; (in questions) ... not ..., expecting yes}
\vocabentry{ὀχέω}{to hold fast; let (another) ride, mount; (mid). ride}
\vocabentry{πιέζω}{to press, squeeze; oppress, distress}
\vocabentry{πονέω}{to work hard, do work, suffer toil}
\vocabentry{πρεσβύτερος, α, ον}{older (comp. πρέσβυς)}
\vocabentry{σόφισμα, ματος, τό}{any skilful act; sophism; stage-trick}
\vocabentry{Σταμνίας, ὁ}{(Comic proper noun) Wine-jar, < στάμνος wine-jar }
\vocabentry{ταλαιπωρέω}{to go through hard labour, to suffer hardship}
\vocabentry{τράχηλος, ὁ}{the neck, throat}
\vocabentry{τρισκακοδαίμων, ων, ον}{thrice unlucky}
\vocabentry{τρυφή, ἡ}{softness, delicacy, daintiness}
\vocabentry{ὕβρις, εως, ἡ}{wantonness, wanton violence}
\vocabentry{υἱός, ὁ}{a son}
\vocabentry{ὦμος, ὁ}{shoulder (with the upper arm)}
\end{multicols}

\vspace{-1.5em}
\noindent\rule{\linewidth}{0.4pt}
\vspace{-2em}

\begin{multicols}{2}
\begin{parcolumns}[colwidths={1=1.5em, 2=0.9\linewidth}]{2}
\small

\glossline{16}{%
  \glossmix{νυν}{enclitic with commands, \textit{come now}}\\
  \glossmix{ὡς}{as, since}\\
  \glossmix{θεώμενος}{\textit{spectator}; Athenian comedies were performed at festivals to Dionysus and a statue of the god was placed in the theater.}\\
}
\glossline{18}{%
  \glossmix{πλεῖν}{= πλεῖον, πλέον, i.e. neut. sg. of πλείων. Adverbial accusative.}\\
  \glossmix{'νιαυτῷ}{= ἐνιαυτῷ. dat. of degree of difference, redundant with ἢ. \textit{more than [by] a year}}\\
}
\glossline{19}{%
  \glossmix{οὑτοσί}{deictic ('pointing') iota as suffix to οὗτος, \textit{this here}}. Cf. ταυτί, v. 26.\\
  \glossmix{τρισκακοδαίμων ἄρ᾽ ὁ τράχηλος οὑτοσί}{nominatives, supply ἐστιν. \textit{thrice-unlucky is this...}}\\
}
\glossline{21}{%
    \glossmix{εἶτ᾽}{εἶτα}\\
    \glossmix{οὐχ}{Questions beginning with οὐ expect a positive response. Cf. v. 25. \textit{Then aren't these things... }}\\
}

\glossline{22}{%
    \glossmix{ὅτ'}{= ὅτε. "the iota of ὅτι is never elided in Attic" (Stanford).}\\
    \glossmix{Σταμνίου}{Dionysus is the son of Zeus, but for humorous effect here Aristophanes invents Stamnios(/as) derived from σταμνός ("wine jar").}\\
}

\glossline{23}{%
  \glossmix{τοῦτον}{i.e. Xanthias}\\
  \glossmix{μὴ... μηδ᾽}{\textit{not... nor}}\\
  }

\glossline{24}{%
    \glossmix{ταλαιπωροῖτο... φέροι}{Normally the subjunctive is used in a purpose clause after a primary sequence main verb. In this case, the optative is used because ὀχῶ implies a reference to the past ("I let you mount in the past and now you ride"). Cf. Smyth 2200}\\
}
  
\glossline{25}{%
  \glossmix{γὰρ}{in abrupt questions, \textit{what, why}; \textit{What, I'm not the one carrying?}. Stanford sees these joke as parodies of sophistic argumentation about the active/passive voice, e.g. Euthyphro. Cf. 17}\\
  \glossmix{ὀχεῖ}{2s mid. ὀχέω. \textit{you who are riding [lit. being held]}}\\
}
\glossline{26}{%
  \glossmix{ταυτί}{ταῦτα and deictic iota, \textit{these things [in front of us]}}\\
  \glossmix{Τίνα τρόπον}{\textit{How?}}\\
  \glossmix{βαρέως πάνυ}{\textit{[carrying them] very heavily}}}\\
}
\glossline{27}{%
  \glossmix{φέρει}{Active, unlike 29 φέρει}}\\
}

\glossline{28}{%
    \glossmix{οὐ δῆθ’... μὰ τὸν Δι’ οὔ}{An extremely emphatic negative.}\\
    \glossmix{ὅ γ'... φέρω}{relative clause introduced by ὅ, \textit{not what I'm holding and carrying at least}}\\
}

\glossline{29}{%
    \glossmix{φέρει}{2s pass., \textit{you who are being carried}}
}

\end{parcolumns}
\end{multicols}

\newpage

\begin{spacing}{1.5}

\begin{tabularx}{\textwidth}{@{}lXr@{}}
  \textit{Δι.} & σὺ δ᾽ οὖν ἐπειδὴ τὸν ὄνον οὐ φῄς σ᾽ ὠφελεῖν, & \\
  & ἐν τῷ μέρει σὺ τὸν ὄνον ἀράμενος φέρε. & \\
  \textit{Ξα.} & οἴμοι κακοδαίμων· τί γὰρ ἐγὼ οὐκ ἐναυμάχουν; & \\
  & ἦ τἄν σε κωκύειν ἂν ἐκέλευον μακρά. & \\
  \textit{Δι.} & κατάβα πανοῦργε. καὶ γὰρ ἐγγὺς τῆς θύρας & 35 \\
  & ἤδη βαδίζων εἰμὶ τῆσδ᾽, οἷ πρῶτά με & \\
  & ἔδει τραπέσθαι. παιδίον, παῖ, ἠμί, παῖ. & \\
\end{tabularx}

\noindent\textit{\MakeUppercase{\textls[200]{Ἡρακλῆς}}}

\begin{tabularx}{\textwidth}{@{}lXr@{}}
  & τίς τὴν θύραν ἐπάταξεν; ὡς κενταυρικῶς & \\
  & ἐνήλαθ᾽ ὅστις· εἰπέ μοι τουτὶ τί ἦν; & \\
  \textit{Δι.} & ὁ παῖς. & \\
  \textit{Ξα.} & \hspace*{3em}τί ἔστιν; & \\
  \textit{Δι.} & \hspace*{6.5em}οὐκ ἐνεθυμήθης; & \\
  \textit{Ξα.} & \hspace*{13.5em}τὸ τί; & 40 \\
  \textit{Δι.} & ὡς σφόδρα μ᾽ ἔδεισε. & \\
  \textit{Ξα.} & \hspace*{8.5em}νὴ Δία μὴ μαίνοιό γε. & \\
  \textit{Ἡρ.} & οὔ τοι μὰ τὴν Δήμητρα δύναμαι μὴ γελᾶν· & \\
  & καίτοι δάκνω γ᾽ ἐμαυτόν· ἀλλ᾽ ὅμως γελῶ. & \\
  \textit{Δι.} & ὦ δαιμόνιε πρόσελθε· δέομαι γάρ τί σου. & \\
\end{tabularx}

\end{spacing}

\newpage

\begin{multicols}{2}
\small % roughly 9pt
\vocabentry{αἴρω}{to lift, raise up}
\vocabentry{δαιμόνιος, α, ον}{of/belonging to a δαίμων; marvelous; (voc.) good sir/lady}
\vocabentry{δάκνω}{to bite}
\vocabentry{δείδω}{to fear}
\vocabentry{δέομαι}{to need, want (w. gen. of person and acc. of thing)}
\textbf{\vocabentry{ἐγγύς}{(adv.) near, nigh, at hand}}
\vocabentry{ἐν-άλλομαι}{to leap in}
\vocabentry{ἐν-θυμέομαι}{to lay to heart, ponder; notice, consider}
\vocabentry{ἠμί}{to say}
\vocabentry{καίτοι}{and indeed, and further; and yet}
\vocabentry{κακο-δαίμων, ον}{ill-fated; (freq in Com.) poor devil!}
\vocabentry{κατα-βαίνω}{to step down, go}
\vocabentry{κελεύω}{to urge}
\textbf{\vocabentry{Κενταυρικός}{(adv.) like a Centaur}}
\vocabentry{κωκύω}{to shriek, cry, wail}
\vocabentry{μαίνομαι}{to rage, be crazy}
\vocabentry{μακρός, ός, ά, όν}{long}
\vocabentry{ναυ-μαχέω}{to fight by sea}
\vocabentry{οἷ}{whither}
\vocabentry{ὅμως}{nevertheless, still}
\vocabentry{ὄνος, ὁ, ἡ}{an ass}
\vocabentry{παιδίον, τό}{a child}
\vocabentry{παν-οῦργος, ον}{willing to do anything, tricky; (in comedy) general term of abuse}
\vocabentry{πατάσσω}{to beat, knock}
\vocabentry{προσ-έρχομαι}{to come}
\vocabentry{τρέπω}{turn; (mid). turn or betake oneself, go}
\vocabentry{σφόδρα}{(adv.) very, very much, exceedingly, violently}
\vocabentry{ὠφελέω}{to help, aid, assist, to be of use}
\end{multicols}

\vspace{-1.5em}
\noindent\rule{\linewidth}{0.4pt}
\vspace{-2em}

\begin{multicols}{2}
\begin{parcolumns}[colwidths={1=1.5em, 2=0.9\linewidth}]{2}
\small

\glossline{32}{%
 \glosstxt{ἐν τῷ μέρει}{in turn}\\
}

\glossline{33}{%
 \glossmix{τί γὰρ ἐγὼ οὐκ ἐναυμάχουν;}{Slaves who fought in the naval battle at Arginusai in 406 BCE were granted their freedom.}\\
}

\glossline{34}{%
  \glossmix{ἦ τἄν}{ἦ τοι ἄν, \textit{Then, you know...}. The repetition of ἄν is not unusual.}\\
  \glossmix{μακρά}{Adverbial/internal accusative as usual adv. for μακρός, \textit{at length}.}\\
}

\glossline{35}{%
 \glossmix{κατάβα}{aor. impv. καταβαίνω, \textit{dismount}. The actors arrive at a door in the stage building. After this point there is no further mention of the donkey, which is presumably led offstage by a mute actor.}\\
}

\glossline{37}{%
 \glossmix{παιδίον, παῖ}{Referring to a slave expected to open the door.}\\
}

\glossline{38}{%
 \glossmix{}{Heracles himself unexpectedly opens the door.}\\
}

\glossline{39}{%
 \glossmix{ὅστις}{\textit{whoever [it was]}}\\
 \glossmix{τουτὶ τί ἦν}{\textit{what's this thing here?}. Impf. slightly difficult (Stanford claims 'imperfect of intention'}\\
}

\glossline{40}{%
 \glossmix{ὁ παῖς}{to Xanthias. Masters often use the nominative rather than the vocative in addressing slaves, cf. 521}\\
}

\glossline{41}{%
 \glossmix{μὴ μαίνοιό γε}{Fear clause picking up on ἔδεισε, \textit{yes by Zeus, [afraid] that you're crazy at least}}\\
}

\end{parcolumns}
\end{multicols}

\newpage

\begin{spacing}{1.5}

\begin{tabularx}{\textwidth}{@{}lXr@{}}
  \textit{Ἡρ.} & Ἀλλ' οὐχ οἷός τ' εἴμ' ἀποσοβῆσαι τὸν γέλων & 45 \\
  & ὁρῶν λεοντῆν ἐπὶ κροκωτῷ κειμένην. & \\
  & Τίς ὁ νοῦς; Τί κόθορνος καὶ ῥόπαλον ξυνηλθέτην; & \\
  & Ποῖ γῆς ἀπεδήμεις; & \\
  \textit{Δι.} & \hspace*{8em}Ἐπεβάτευον Κλεισθένει. & \\
  \textit{Ἡρ.} & Κἀναυμάχησας; & \\
  \textit{Δι.} & \hspace*{7em}Καὶ κατεδύσαμέν γε ναῦς & \\
  & τῶν πολεμίων ἢ δώδεκ' ἢ τρεισκαίδεκα. & 50 \\
  \textit{Ἡρ.} & Σφώ; & \\
  \textit{Δι.} & \hspace*{2.5em}Νὴ τὸν Ἀπόλλω. & \\
  \textit{Ξα.} & \hspace*{9.5em}Κᾆτ' ἔγωγ' ἐξηγρόμην. & \\
  \textit{Δι.} & Καὶ δῆτ' ἐπὶ τῆς νεὼς ἀναγιγνώσκοντί μοι & \\
  & τὴν Ἀνδρομέδαν πρὸς ἐμαυτὸν ἐξαίφνης πόθος & \\
  & τὴν καρδίαν ἐπάταξε πῶς οἴει σφόδρα. & \\
  \textit{Ἡρ.} & Πόθος; πόσος τις; & \\
  \textit{Δι.} & \hspace*{7.5em}Σμικρός, ἡλίκος Μόλων. & 55 \\
  \textit{Ἡρ.} & Γυναικός; & \\
  \textit{Δι.} & \hspace*{4em}Οὐ δῆτ'. & \\
  \textit{Ἡρ.} & \hspace*{7.5em}Ἀλλὰ παιδός; & \\
  \textit{Δι.} & \hspace*{13em}Οὐδαμῶς. & \\
\end{tabularx}

\end{spacing}

\newpage

\begin{multicols}{2}
\small % roughly 9pt
\vocabentry{ἀνα-γιγνώσκω}{to know well, know certainly; to read}
\vocabentry{Ἀνδρομέδα, ἡ}{Andromeda; heroine, lost tragedy by Euripides, produced 413/2}
\vocabentry{ἀπο-δημέω}{to be away from home, be abroad}
\vocabentry{ἀπο-σοβέω}{to scare away; (metaph.) to keep off}
\vocabentry{γέλως, ωτος, ὁ (poet. acc. γέλων)}{laughter}
\vocabentry{δώ-δεκα}{twelve}
\vocabentry{ἐξ-αίφνης}{(adv.) suddenly}
\vocabentry{ἐξ-εγείρω}{to awaken}
\vocabentry{ἐπι-βατεύω}{to serve as a marine (ἐπιβάτης)}
\vocabentry{ἡλίκος, ός, ά, όν}{as big as, of the same age as; how great, what size..!}
\vocabentry{καρδία, ἡ}{the heart}
\vocabentry{κατά-δύω}{to go down; (causal) to make to sink, sink}
\vocabentry{Κλεισθένης, ους, ὁ}{Cleisthenes, frequently attacked in comedy for his alleged effeminacy}
\vocabentry{κόθορνος, ὁ}{high boot associated with women and Dionysus, in post-classical theater worn by tragic actors}
\vocabentry{κροκωτός, ὁ}{a saffron-colored robe worn by women on special occasions < κροκωτός saffron-dyed}
\vocabentry{λεοντέη, ἡ}{a lion's skin}
\vocabentry{Μόλων, οντος, ὁ}{Molon}
\vocabentry{ναυ-μαχέω}{to fight by sea}
\vocabentry{νή}{Particle of strong affirmation, with acc. of the divinity invoked}
\vocabentry{οἷός τ' εἴμ' + inf.}{to be able (to do)}
\vocabentry{οὐδαμῶς}{(adv.) in no way}
\vocabentry{πατάσσω}{to beat, knock}
\vocabentry{πόθος, ὁ}{a longing, yearning, fond desire}
\vocabentry{πολέμιος, ός, ά, όν}{hostile; enemy}
\vocabentry{ῥόπαλον, τό}{a club, cudgel}
\vocabentry{σμικρός, ός, ά, όν}{= μικρός}
\vocabentry{συν-έρχομαι (Attic ξυν-)}{come together, meet}
\vocabentry{Σφώ}{nom. dual. of σύ, you two}
\vocabentry{σφόδρα}{(adv.) very, very much, exceedingly, violently}
\vocabentry{τρει-καί-δεκα, ὁ, ἡ}{thirteen}
\end{multicols}

\vspace{-1.5em}
\noindent\rule{\linewidth}{0.4pt}
\vspace{-2em}

\begin{multicols}{2}
\begin{parcolumns}[colwidths={1=1.5em, 2=0.9\linewidth}]{2}
\small

\glossline{47}{%
 \glossmix{Τίς ὁ νοῦς}{\textit{What's the idea}}\\
 \glossmix{ξυνηλθέτην}{3rd person dual aorist of συνέρχομαι, \textit{come together}}\\
}

\glossline{48}{%
 \glossmix{Ποῖ γῆς}{\textit{Where in the world did you go off to?}}\\
 \glossmix{Ἐπεβάτευον}{\textit{I served as a marine for Cleisthenes.}, but possibly a pun on ἐπιβαίνω \textit{I was mounting Cleisthenes}}\\
}

\glossline{49}{%
 \glossmix{Κἀναυμάχησας}{= καὶ ἐναυμάχησας}\\
}

\glossline{51}{%
 \glossmix{Κᾆτ}{= καὶ εἶτα}\\
 \glossmix{ἐξηγρόμην}{aor. mid. ἐξεγείρω}\\
}

\glossline{53}{%
 \glossmix{πῶς οἴει}{parenthetical question, "used virtually as an adverb of intensification" (Dover), \textit{you can't imagine how, unbelievably}}\\
 \glossmix{ἐξηγρόμην}{aor. mid. ἐξεγείρω}\\
}

\glossline{55}{%
 \glossmix{Μόλων}{A famous actor, apparently a large man.}\
} 

\glossline{56}{%
 \glossmix{Οὐ δῆτ'. Ἀλλὰ}{\textit{Certainly now. Well, then...} (Stanford)}\
}

\end{parcolumns}
\end{multicols}

\newpage

\begin{spacing}{1.5}

\begin{tabularx}{\textwidth}{@{}lXr@{}}
  \textit{Ἡρ.} & Ἀλλ' ἀνδρός; & \\
  \textit{Δι.} & \hspace*{5.5em}Ἀπαπαῖ. & \\
  \textit{Ἡρ.} & \hspace*{9em}Ξυνεγένου τῷ Κλεισθένει; & \\
  \textit{Δι.} & Μὴ σκῶπτέ μ', ὦδέλφ'· οὐ γὰρ ἀλλ' ἔχω κακῶς· & \\
  & τοιοῦτος ἵμερός με διαλυμαίνεται. & \\
  \textit{Ἡρ.} & Ποῖός τις, ὦδελφίδιον; & \\
  \textit{Δι.} & \hspace*{9.5em}Οὐκ ἔχω φράσαι. & 60 \\
  & Ὅμως γε μέντοι σοι δι' αἰνιγμῶν ἐρῶ. & \\
  & Ἤδη ποτ' ἐπεθύμησας ἐξαίφνης ἔτνους; & \\
  \textit{Ἡρ.} & Ἔτνους; Βαβαιάξ, μυριάκις γ' ἐν τῷ βίῳ. & \\
  \textit{Δι.} & Ἆρ' ἐκδιδάσκω τὸ σαφὲς, ἢ 'τέρᾳ φράσω; & \\
  \textit{Ἡρ.} & Μὴ δῆτα περὶ ἔτνους γε· πάνυ γὰρ μανθάνω. & 65 \\
  \textit{Δι.} & Τοιουτοσὶ τοίνυν με δαρδάπτει πόθος & \\
  & Εὐριπίδου. & \\
  \textit{Ἡρ.} & \hspace*{5em}Καὶ ταῦτα τοῦ τεθνηκότος; & \\
  \textit{Δι.} & Κοὐδείς γέ μ' ἂν πείσειεν ἀνθρώπων τὸ μὴ οὐκ & \\
  & ἐλθεῖν ἐπ' ἐκεῖνον. & \\
  \textit{Ἡρ.} & \hspace*{7.5em}Πότερον εἰς Ἅιδου κάτω; & \\
  \textit{Δι.} & Καὶ νὴ Δί' εἴ τί γ' ἔστιν ἔτι κατωτέρω. & 70 \\
\end{tabularx}

\end{spacing}

\newpage

\begin{multicols}{2}
\small % roughly 9pt
\vocabentry{ἀδελφίδιον, τό}{diminutive of ἀδελφός}
\vocabentry{ἀδελφός, ὁ}{brother}
\vocabentry{ἀπαπαῖ}{"an inarticulate expression of grief or pain" (Stanford)}
\vocabentry{αἰνιγμός, ὁ}{a riddle; allusion, allegory}
\vocabentry{βαβαιάξ}{strengthened form of βαβαί}
\vocabentry{βαβαί}{exclamation of a reaction to misfortune}
\vocabentry{βίος, ὁ}{life}
\vocabentry{δαρδάπτω}{to devour}
\vocabentry{δια-λυμαίνομαι}{to maltreat shamefully, ruin}
\vocabentry{ἐξαίφνης}{(adv.) suddenly}
\vocabentry{ἐκ-διδάσκω}{to teach thoroughly}
\vocabentry{ἐπι-θυμέω}{to set one's heart upon (a thing), long for (+ gen.)}
\vocabentry{ἔτνος, εος, τό}{thick soup}
\vocabentry{ἔχω}{to have; (+ inf.) to be able, can}
\vocabentry{ἵμερος, ὁ}{a longing}
\vocabentry{κατωτέρω}{(adv.) lower (comp. κατώ)}
\vocabentry{μανθάνω}{to learn; to understand}
\vocabentry{μυριάκις}{(adv.) ten thousand times}
\vocabentry{πόθος, ὁ}{a longing, yearning, fond desire}
\vocabentry{πότερον}{(adv.) introduces question with two alternatives}
\vocabentry{σκώπτω}{to mock, make fun of}
\vocabentry{συγ-γίγνομαι (Att. ξυγ-)}{to be with}
\vocabentry{φράζω}{to point out, show; tell, declare, explain}
\end{multicols}

\vspace{-1.5em}
\noindent\rule{\linewidth}{0.4pt}
\vspace{-2em}

\begin{multicols}{2}
\begin{parcolumns}[colwidths={1=1.5em, 2=0.9\linewidth}]{2}
\small

\glossline{57}{%
 \glossmix{Ξυνεγένου}{= συνεγένου, aor. συγγίγνομαι.}\\
}

\glossline{58}{%
 \glossmix{ὦδέλφ}{= ὦ ἀδέλφε. Both Heracles and Dionysus were sons of Zeus.}\\
 \glossmix{οὐ γὰρ ἀλλ'}{\textit{because... really...}}\\
 \glossmix{ἔχω κακῶς}{ἔχω + adverb describes a person's state. \textit{I'm doing poorly}}\\
}

\glossline{61}{%
 \glossmix{Ὅμως γε μέντοι}{\textit{Nevertheless}}\\
}

\glossline{64}{%
 \glossmix{Ἆρ' ἐκδιδάσκω τὸ σαφὲς}{\textit{Am I spelling out what's clear}}\\
 \glossmix{'τέρᾳ}{= ἐτέρᾳ. \textit{in another way}}\\
 \glossmix{ἑτέρᾳ φράσω}{The scholia claim that this half-line comes from Euripides' Hypsipyle (ἔστι δὲ τὸ ἡμιστίχιον ἐξ Ὑψιπύλης Εὐριπίδου.)}\\
}

\glossline{67}{%
 \glossmix{καὶ ταῦτα}{\textit{and that (lit. these things), and what's more}}
}

\glossline{68}{%
 \glossmix{τὸ μὴ οὐκ}{τὸ μὴ οὐκ + inf. normal construction after a verb of persuasion. \textit{could persuade me not to...}}
}

\glossline{69}{%
 \glossmix{ἐπ' ἐκεῖνον}{\textit{after him, in search of him, to get him}}
}

\end{parcolumns}
\end{multicols}

\newpage

\begin{spacing}{1.5}

\begin{tabularx}{\textwidth}{@{}lXr@{}}
  \textit{Ἡρ.} & Τί βουλόμενος; & \\
  \textit{Δι.} & \hspace*{6.5em}Δέομαι ποητοῦ δεξιοῦ. & \\
  & Οἱ μὲν γὰρ οὐκέτ' εἰσίν, οἱ δ' ὄντες κακοί. & \\
  \textit{Ἡρ.} & Τί δ'; Οὐκ Ἰοφῶν ζῇ; & \\
  \textit{Δι.} & \hspace*{8.5em}Τοῦτο γάρ τοι καὶ μόνον & \\
  & ἔτ' ἐστὶ λοιπὸν ἀγαθόν, εἰ καὶ τοῦτ' ἄρα· & \\
  & οὐ γὰρ σάφ' οἶδ' οὐδ' αὐτὸ τοῦθ' ὅπως ἔχει. & 75 \\
  \textit{Ἡρ.} & Εἶτ' οὐ Σοφοκλέα πρότερον ὄντ' Εὐριπίδου & \\
  & μέλλεις ἀναγαγεῖν, εἴπερ ἐκεῖθεν δεῖ σ' ἄγειν; & \\
  \textit{Δι.} & Οὔ, πρίν γ' ἂν Ἰοφῶντ', ἀπολαβὼν αὐτὸν μόνον, & \\
  & ἄνευ Σοφοκλέους ὅ τι ποεῖ κωδωνίσω. & \\
  & Κἄλλως ὁ μέν γ' Εὐριπίδης πανοῦργος ὢν & 80 \\
  & κἂν ξυναποδρᾶναι δεῦρ' ἐπιχειρήσειέ μοι· & \\
  & ὁ δ' εὔκολος μὲν ἐνθάδ', εὔκολος δ' ἐκεῖ. & \\
  \textit{Ἡρ.} & Ἀγάθων δὲ ποῦ 'στιν; & \\
  \textit{Δι.} & \hspace*{9em}Ἀπολιπών μ' ἀποίχεται, & \\
  & ἀγαθὸς ποητὴς καὶ ποθεινὸς τοῖς φίλοις. & \\
\end{tabularx}

\end{spacing}

\newpage

\begin{multicols}{2}
\small % roughly 9pt
\vocabentry{ἄνευ}{without + gen.}
\vocabentry{Ἀγάθων, οντος, ὁ}{Agathon, famous tragedian}
\vocabentry{ἀν-άγω}{to lead up}
\vocabentry{ἀπο-λείπω}{to leave behind; desert, abandon}
\vocabentry{ἀπ-οίχομαι}{to be gone away, to be far from}
\vocabentry{δεξιός, ά, όν}{right; skillful, clever}
\vocabentry{δέω}{to lack; (mid.) to need + gen.}
\vocabentry{ἐκεῖ}{(adv.) there, in that place}
\vocabentry{ἐκεῖθεν}{(adv.) from that place, thence}
\vocabentry{ἐπι-χειρέω}{to attempt, try}
\vocabentry{εὔκολος, ον}{easily satisfied, contented}
\vocabentry{ζῶ}{to live}
\vocabentry{Ἰοφῶν, ὁ}{Iophon, Sophocles' son, successful tragedian}
\vocabentry{κωδωνίζω}{to prove by ringing < κώδων bell}
\vocabentry{λοιπός, ά, όν}{remaining, left}
\vocabentry{πανοῦργος, ον}{ready to do anything, roguish}
\vocabentry{ποθεινός, ά, όν}{longed for, desired, much desired}
\vocabentry{ποιητής, οῦ, ὁ (Att. ποη-)}{maker; poet < ποιέω}
\vocabentry{σάφα}{(adv.) clearly, plainly, assuredly}
\vocabentry{Σοφοκλῆς, ὁ}{Sophocles, big 3 tragedian}
\vocabentry{συν-απο-διδράσκω (Att. ξυν-)}{to run away along with}
\end{multicols}

\vspace{-1.5em}
\noindent\rule{\linewidth}{0.4pt}
\vspace{-2em}

\begin{multicols}{2}
\begin{parcolumns}[colwidths={1=1.5em, 2=0.9\linewidth}]{2}
\small

\glossline{72}{%
 \glossmix{}{a quotation from Euripides' Oeneus. \textit{σὺ δ' ὧδ' ἔρημος ξυμμάχων ἀπόλλυσαι·
  οἱ μὲν γὰρ οὐκ ἔτ' εἰσὶν, οἱ δ' ὄντες κακοὶ.}}\\
}

\glossline{73}{%
 \glossmix{ζῇ}{3s pres. act. ind. ζῶ}\\
 \glossmix{Τοῦτο γάρ τοι καὶ μόνον...}{\textit{Yes, for this is still just the only good thing left}}\\
 \glossmix{εἰ καὶ τοῦτ' ἄρα}{\textit{if in fact it is}}\\
}

\glossline{76}{%
 \glossmix{πρότερον ὄντ'}{\textit{since he's superior to, better than}}\\
}

\glossline{77}{%
 \glossmix{ἀναγαγεῖν}{aor. inf. ἀνάγω}\\
}

\glossline{78}{%
 \glossmix{πρίν ἄν}{LSJ II.2.: πρίν in sense of \textit{until} after a negative regularly takes ἄν + subj. \textit{No, not until I test...}}\\
}

\glossline{79}{%
 \glossmix{ποεῖ}{= ποιεῖ}\\
}

\glossline{80}{%
 \glossmix{Κἄλλως}{= καὶ ἄλλως, \textit{and anyway}}\\
 \glossmix{}{The sense is: That old rogue Euripides might try to escape up here anyway.}
}

\glossline{81}{%
 \glossmix{ξυναποδρᾶναι}{aor. inf. συναποδιδράσκω, elsewhere used of deserters and fugitive slaves.}\\
}

\glossline{83}{%
 \glossmix{ἀπολιπών}{aor. pple. ἀπολείπω}\\
}

\end{parcolumns}
\end{multicols}

\newpage

\begin{spacing}{1.5}

\begin{tabularx}{\textwidth}{@{}lXr@{}}
  \textit{Ἡρ.} & Ποῖ γῆς ὁ τλήμων; & \\
  \textit{Δι.} & \hspace*{8em}Ἐς μακάρων εὐωχίαν. & 85 \\
  \textit{Ἡρ.} & Ὁ δὲ Ξενοκλέης; & \\
  \textit{Δι.} & \hspace*{7em}Ἐξόλοιτο νὴ Δία. & \\
  \textit{Ἡρ.} & Πυθάγγελος δέ; & \\
  \textit{Ξα.} & \hspace*{7em}Περὶ ἐμοῦ δ' οὐδεὶς λόγος & \\
  & ἐπιτριβομένου τὸν ὦμον οὑτωσὶ σφόδρα. & \\
  \textit{Ἡρ.} & Οὔκουν ἕτερ' ἔστ' ἐνταῦθα μειρακύλλια & \\
  & τραγῳδίας ποιοῦντα πλεῖν ἢ μυρία, & 90 \\
  & Εὐριπίδου πλεῖν ἢ σταδίῳ λαλίστερα; & \\
  \textit{Δι.} & Ἐπιφυλλίδες ταῦτ' ἐστὶ καὶ στωμύλματα, & \\
  & χελιδόνων μουσεῖα, λωβηταὶ τέχνης, & \\
  & ἃ φροῦδα θᾶττον, ἢν μόνον χορὸν λάβῃ, & \\
  & ἅπαξ προσουρήσαντα τῇ τραγῳδίᾳ. & 95 \\
  & Γόνιμον δὲ ποιητὴν ἂν οὐχ εὕροις ἔτι & \\
  & ζητῶν ἄν, ὅστις ῥῆμα γενναῖον λάκοι. & \\
  \textit{Ἡρ.} & Πῶς γόνιμον; & \\
  \textit{Δι.} & \hspace*{6em}Ὡδὶ γόνιμον, ὅστις φθέγξεται & \\
  & τοιουτονί τι παρακεκινδυνευμένον, & \\
  & “αἰθέρα Διὸς δωμάτιον,” ἢ “χρόνου πόδα,” & 100 \\

\end{tabularx}

\end{spacing}

\newpage

\begin{multicols}{2}
\small % roughly 9pt
\vocabentry{αἰθήρ, έρος, ἡ/ὁ}{ether, the heaven}
\vocabentry{ἅπαξ}{(adv.) once}
\vocabentry{γενναῖος, α, ον}{high-born, noble; high-minded}
\vocabentry{γόνιμος, ον}{fruitful, fertile}
\vocabentry{δωμάτιον, τό}{a chamber, bed-chamber}
\vocabentry{ἐξ-όλλυμι}{to destroy utterly; (mid.) perish utterly}
\vocabentry{ἐπιτρίβω}{to rub on the surface, to crush}
\vocabentry{ἐπιφυλλίς, ος, ἡ}{the small grapes left for gleaners (though meaning disputed)}
\vocabentry{εὐωχία, ἡ}{good cheer, feasting}
\vocabentry{θάσσων, ον (Att. θάττων)}{(neut. as adv.) very quickly, comp. ταχύς}
\vocabentry{λάλος, ός, ά, όν}{talkative, babbling, loquacious; irreg. comp. λαλίστερος}
\vocabentry{λάσκω}{to ring, rattle; to scream, shout; to utter}
\vocabentry{λαβή, ἡ}{a handle, haft}
\vocabentry{λωβητής, ὁ}{destroyer, < λωβᾶσθαι damage, spoil}
\vocabentry{μάκαρ, αρος, ὁ}{blessed}
\vocabentry{μειρακύλλιον, τό}{dim. μειράκιον, little lad, kid}
\vocabentry{Μουσεῖον, τό}{shrine of the Muses; music-hall}
\vocabentry{μυρίος, ός, ά, όν}{numberless, countless, infinite}
\vocabentry{Ξενοκλῆς, ὁ}{Xenokles, minor tragedian}
\vocabentry{οὔκουν}{certainly not; (in questions) ... not ..., expecting yes}
\vocabentry{παρα-κινδυνεύω}{to venture, risk}
\vocabentry{προσ-ουρέω}{urinate on; piss on (+ dat.)}
\vocabentry{Πυθάγγελος, ὁ}{Pythangelus, tragedian, otherwise unknown}
\vocabentry{ῥῆμα, ατος, τό}{that which is said or spoken, word, saying}
\vocabentry{στάδιον, τό}{a stade, = ca. 600 feet}
\vocabentry{στώμυλμα, ματος, τό}{chatterbox < στωμύλος talkative < στόμα mouth}
\vocabentry{σφόδρα}{(adv.) very, very much}
\vocabentry{τλήμων, ων, ον}{suffering, enduring; wretched, miserable}
\vocabentry{φροῦδος, ός, ά, όν}{gone away, clean gone; (of persons) gone, fled, departed}
\vocabentry{χελιδών, όνος, ἡ}{swallow (bird)}
\vocabentry{ὦμος, ὁ}{shoulder (with the upper arm)}
\vocabentry{ὡδί}{(adv.) }
\end{multicols}

\vspace{-1.5em}
\noindent\rule{\linewidth}{0.4pt}
\vspace{-2em}

\begin{multicols}{2}
\begin{parcolumns}[colwidths={1=1.5em, 2=0.9\linewidth}]{2}
\small

\glossline{86}{%
 \glossmix{Ἐξόλοιτο}{aor. opt. ἐξόλλυμι, \textit{may he perish}}\\
}

\glossline{87}{%
 \glossmix{Περὶ ἐμοῦ δ' οὐδεὶς λόγος}{\textit{but about me there's no consideration}}\\
}

\glossline{91}{%
 \glossmix{Εὐριπίδου πλεῖν ἢ σταδίῳ λαλίστερα;}{\textit{'And miles verboser than Euripides'} (Murray)}\\
}

\glossline{93}{%
 \glossmix{χελιδόνων μουσεῖα}{\textit{performance halls of swallows}, a parody of Euripides' Alcmene: πολὺς δ' ἀνεῖρπε κισσὸς, εὐφυὴς κλάδος, / χελιδόνων μουσεῖον. Greeks frequently compared non-Greek language to the sound of swallows (Σ: ἀντὶ τοῦ βάρβαρα καὶ ἀσύνετα)}\\
}

\glossline{94}{%
 \glossmix{}{Supply ἐστιν. \textit{who [are] gone in a flash (lit. very quickly), if they get a single chorus, after pissing once and for all on tragedy.} To 'get a chorus' is to be granted a performance slot in the festival competition.}\\
}

\glossline{99}{%
 \glossmix{τοιουτονί τι παρακεκινδυνευμένον,}{\textit{a risky kind of (expression) like this}}\\
}

\end{parcolumns}
\end{multicols}

\newpage

\begin{spacing}{1.5}

\begin{tabularx}{\textwidth}{@{}lXr@{}}
  & ἢ «φρένα μὲν οὐκ ἐθέλουσαν ὀμόσαι καθ' ἱερῶν, & \\
  & γλῶτταν δ' ἐπιορκήσασαν ἰδίᾳ τῆς φρενός.» & \\
  \textit{Ἡρ.} & Σὲ δὲ ταῦτ' ἀρέσκει; & \\
  \textit{Δι.} & \hspace*{8.5em}Μἀλλὰ πλεῖν ἢ μαίνομαι. & \\
  \textit{Ἡρ.} & Ἦ μὴν κόβαλά γ' ἐστίν, ὡς καὶ σοὶ δοκεῖ. & \\
  \textit{Δι.} & Μὴ τὸν ἐμὸν οἴκει νοῦν· ἔχεις γὰρ οἰκίαν. & 105 \\
  \textit{Ἡρ.} & Καὶ μὴν ἀτεχνῶς γε παμπόνηρα φαίνεται. & \\
  \textit{Δι.} & Δειπνεῖν με δίδασκε. & \\
  \textit{Ξα.} & \hspace*{8.5em}Περὶ ἐμοῦ δ' οὐδεὶς λόγος. & \\
  \textit{Δι.} & Ἀλλ' ὧνπερ ἕνεκα τήνδε τὴν σκευὴν ἔχων & \\
  & ἦλθον κατὰ σὴν μίμησιν, ἵνα μοι τοὺς ξένους & \\
  & τοὺς σοὺς φράσειας, εἰ δεοίμην, οἷσι σὺ & 110 \\
  & ἐχρῶ τόθ', ἡνίκ' ἦλθες ἐπὶ τὸν Κέρβερον, & \\
  & τούτους φράσον μοι, λιμένας, ἀρτοπώλια, & \\
  & πορνεῖ', ἀναπαύλας, ἐκτροπάς, κρήνας, ὁδούς, & \\
  & πόλεις, διαίτας, πανδοκευτρίας, ὅπου & \\
  & κόρεις ὀλίγιστοι. & \\
  \textit{Ξα.} & \hspace*{7.5em}Περὶ ἐμοῦ δ' οὐδεὶς λόγος. & 115 \\

\end{tabularx}

\end{spacing}

\newpage

\begin{multicols}{2}
\small % roughly 9pt
\vocabentry{ἀνά-παυλα, ἡ}{repose, rest; resting-place, inn}
\vocabentry{ἀρέσκω}{to make good, make up; to please}
\vocabentry{ἀρτο-πώλιον, τό}{a baker's shop, bakery}
\vocabentry{ἀ-τεχνῶς}{(adv.) simply}
\vocabentry{δέω}{to lack, miss, stand in need of}
\vocabentry{δειπνέω}{to eat dinner < δεῖπνον dinner}
\vocabentry{δίαιτα, ἡ}{a way of living, mode of life; dwelling, abode; room}
\vocabentry{δίδασκω}{to teach}
\vocabentry{ἐκ-τροπή, ἡ}{a turning off; fork, branch (in road)}
\vocabentry{ἐπι-ορκέω}{to swear falsely, forswear oneself}
\vocabentry{ἕνεκα}{(+ gen.) on account of, for the sake of, because of, for}
\vocabentry{ἰδίᾳ}{(adv.) by oneself, privately, separately (+ gen.) < ἴδιος one's own}
\vocabentry{ἱερά, τά}{sacrifices, offerings, victims (< ἱερός holy)}
\vocabentry{κατά}{(+ gen.) down from; (LSJ A.II.4) (of vows or oaths) by}
\vocabentry{Κέρβερος, ὁ}{Cerberus, the guard-dog of Hades}
\vocabentry{κόβαλα, τά}{dirty tricks, rogueries}
\vocabentry{κόρις, ιος, ὀ}{bedbug}
\vocabentry{κρήνη, ἡ}{a well, spring, fountain}
\vocabentry{λιμήν, ένος, ὁ}{harbor; (metaphor.) haven, refuge}
\vocabentry{μίμησις, ἡ}{imitation}
\vocabentry{ὀλίγιστος, ός, ά, όν}{}
\vocabentry{οἰκέω}{to inhabit; to manage, direct}
\vocabentry{ὄμνυμι}{to swear}
\vocabentry{ὅπου}{where (relative pronoun)}
\vocabentry{παμ-πόνηρος, ος, ον}{thoroughly depraved, very bad}
\vocabentry{παν-δοκεύτρια, ἡ}{a hostess}
\vocabentry{πορνεῖον, τό}{brothel}
\vocabentry{σκευή, ἡ}{equipment, dress, costume}
\vocabentry{χράομαι}{to use (+ dat.)}
\vocabentry{φράζω}{to point out, show; to declare, explain}
\end{multicols}

\vspace{-1.5em}
\noindent\rule{\linewidth}{0.4pt}
\vspace{-2em}

\begin{multicols}{2}
\begin{parcolumns}[colwidths={1=1.5em, 2=0.9\linewidth}]{2}
\small

\glossline{101}{%
 \glossmix{}{cf. Eur. Hipp. 612 ἡ γλῶσσ' ὀμώμοχ', ἡ δὲ φρὴν ἀνώμοτος. ("The tongue swore, but the mind is unsworn.)}\\
 \glossmix{ἐθέλουσαν}{with φρένα}\\
 \glossmix{ὀμόσαι}{aor. inf. ὄμνυμι}\\
}

\glossline{103}{%
 \glossmix{Μἀλλὰ...}{μή, ἀλλά. \textit{Don't (say that), but I'm more than crazy (about them)}}\\
}

\glossline{104}{%
 \glossmix{ὡς καὶ σοὶ δοκεῖ.}{\textit{as it seems to you too}. I.e. Heracles thinks that Dionsyus actually agrees with him that these tragic lines are dirty tricks.}\\
}

\glossline{105}{%
 \glossmix{Μὴ τὸν ἐμὸν οἴκει νοῦν}{According to the scholia this phrase is from the \textit{Andromache}, but the extant \textit{Andromache} does not include this phrase. Possibly the scholia's reference is an error for the \textit{Andromeda}.}\\
}

\glossline{106}{%
 \glossmix{Καὶ μὴν}{\textit{Yes, and}}\\
}

\glossline{107}{%
 \glossmix{Δειπνεῖν με δίδασκε.}{i.e. stick to your strengths and don't try to talk to me about poetry}\\
}

\glossline{108}{%
 \glossmix{Ἀλλ' ὧνπερ ἕνεκα}{\textit{Now, as to why I came... [it was] in order that you show me...}.}\\
}

\glossline{109}{%
 \glossmix{κατὰ σὴν μίμησιν,}{\textit{according to your imitation}, i.e. \textit{in imitation of you}}\\
 \glossmix{ξένους}{i.e. the hosts that helped Heracles on his trip to the underworld.}
}

\glossline{110}{%
 \glossmix{εἰ δεοίμην}{\textit{if I may ask} ???. Stanford has "in case of need"}\\
}

\glossline{111}{%
 \glossmix{ἐχρῶ}{impf. 2s χράομαι, \textit{whom you used}}\\
 \glossmix{ἐπὶ}{cf. 69}\\
}

\end{parcolumns}
\end{multicols}

\end{greek}
\end{document}
