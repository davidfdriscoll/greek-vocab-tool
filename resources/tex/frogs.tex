\documentclass[13pt]{article}
\usepackage{fontspec}
\usepackage{polyglossia}
\usepackage{geometry}
\usepackage{setspace}
\usepackage{parskip}
\usepackage{titlesec}
\usepackage{multicol}
\usepackage{tabularx}
\usepackage{parcolumns}
\usepackage{microtype}
\usepackage{graphicx}
\usepackage{enumitem}

\setmainlanguage{english}
\setotherlanguage[variant=ancient]{greek}
\newfontfamily\greekfont[Script=Greek]{Linux Libertine O} % Pre-installed on Overleaf

\geometry{
  paperwidth=6in,
  paperheight=9in,
  top=0.5in,
  bottom=0.5in,
  left=0.3in,
  right=0.7in
}
\titleformat{\section}{\normalfont\large\bfseries}{}{0pt}{}

\newcommand{\vocabentry}[2]{\textbf{#1}: #2\vspace{0.0em}\\}
\newcommand{\glossitem}[2]{\textbf{\textgreek{#1}}: \textit{#2}}
% Greek poetry line macro with optional line number
\newcommand{\poemline}[2]{%
  \ifx&#1&%
    {\hspace{2em}} & {\Large\begin{greek}#2\end{greek}} \\
  \else
    {\small #1} & {\Large\begin{greek}#2\end{greek}} \\
  \fi
}

% For pure literal glosses
\newcommand{\glosstxt}[2]{\textbf{\textgreek{#1}}: \textit{#2}}

% For pure grammatical/explanatory glosses
\newcommand{\glossexpl}[2]{\textbf{\textgreek{#1}}: \textup{#2}}

% For mixed glosses (manual control of italics)
\newcommand{\glossmix}[2]{\textbf{\textgreek{#1}}: \textup{#2}}

% One full line of commentary: line number + multiple glosses
\newcommand{\glossline}[2]{%
  \colchunk[1]{\textbf{#1}}%
  \colchunk[2]{#2}%
  \colplacechunks
}


\date{}
\begin{document}
\Large
\begin{greek}
\begin{spacing}{1.5}

\noindent\textit{\MakeUppercase{\textls[200]{Ξανθίας}}}

\begin{tabularx}{\textwidth}{@{}lXr@{}}
  \phantom{Ξα.} & Εἴπω τι τῶν εἰωθότων, ὦ δέσποτα, & \\
  \phantom{Ξα.} & ἐφ’ οἷς ἀεὶ γελῶσιν οἱ θεώμενοι; & \\
\end{tabularx}

\vspace{1em}

\noindent\textit{\MakeUppercase{\textls[200]{Διόνυσος}}}

\begin{tabularx}{\textwidth}{@{}lXr@{}}
  & νὴ τὸν Δί’ ὅ τι βούλει γε, πλὴν “πιέζομαι”. & \\
  & τοῦτο δὲ φύλαξαι· πάνυ γάρ ἐστ’ ἤδη χολή. & \\
  \textit{Ξα.} & μηδ’ ἕτερον ἀστεῖόν τι; & \\
  \textit{Δι.} & \hspace*{10em}πλήν γ’ “ὡς θλίβομαι”. & 5 \\
  \textit{Ξα.} & τί δαί; τὸ πάνυ γέλοιον εἴπω; & \\
  \textit{Δι.} & \hspace*{12.5em}νὴ Δία & \\
  & θαρρῶν γε· μόνον ἐκεῖν’ ὅπως μὴ ’ρεῖς, & \\
  \textit{Ξα.} & \hspace*{16em}τὸ τί; & \\
  \textit{Δι.} & μεταβαλλόμενος τἀνάφορον ὅτι χεζητιᾷς. & \\
  \textit{Ξα.} & μηδ᾽ ὅτι τοσοῦτον ἄχθος ἐπ᾽ ἐμαυτῷ φέρων, & \\
  & εἰ μὴ καθαιρήσει τις, ἀποπαρδήσομαι; & 10 \\
  \textit{Δι.} & μὴ δῆθ᾽, ἱκετεύω, πλήν γ᾽ ὅταν μέλλω 'ξεμεῖν. & \\
  \textit{Ξα.} & τί δῆτ᾽ ἔδει με ταῦτα τὰ σκεύη φέρειν, & \\
  & εἴπερ ποιήσω μηδὲν ὧνπερ Φρύνιχος & \\
  & εἴωθε ποιεῖν καὶ Λύκις κἀμειψίας; & \\
  
\end{tabularx}

\end{spacing}

\newpage

\begin{multicols}{2}
\small % roughly 9pt
\vocabentry{ἄχθος, εος, τό}{a weight, burden, load}
\vocabentry{ἀεί}{(adv.) always, for ever}
\vocabentry{Ἀμειψίας, ὁ}{Ameipsias, comic poet}
\vocabentry{ἀνά-φορον, τό}{a pole}
\vocabentry{ἀπο-πέρδομαι}{fart}
\vocabentry{ἀστεῖος, ός, ά, όν}{of the town; urbane; witty}
\vocabentry{γέλοιος, ός, ά, όν}{causing laughter, laughable, funny}
\vocabentry{γελάω}{to laugh}
\vocabentry{δαί}{colloquial form of δή, used after interrogatives}
\vocabentry{δεῖ}{it is necessary}
\vocabentry{δεσπότης, ὁ}{a master, lord, the master of the house}
\vocabentry{δῆτα}{particle, more emphatic δή; τί δῆτα what then? μῆ δῆτα, just don't...}
\vocabentry{ἔθω}{be accustomed (see also εἴωθα)}
\vocabentry{ἐμαυτοῦ}{of me, of myself}
\vocabentry{ἐξ-εμέω}{to vomit forth, disgorge}
\vocabentry{ἕτερος, ός, ά, όν}{the one; the other (of two)}
\vocabentry{Ζεύς, ος, ὁ (acc. Δία)}{Zeus}
\vocabentry{θαρσέω}{to be of good courage, take courage}
\vocabentry{θεάομαι}{to look on, gaze at, view, behold}
\vocabentry{θλίβω}{to press, squeeze, pinch}
\vocabentry{καθ-αιρέω}{to take down (i.e. off one's shoulders)}
\vocabentry{Λύκις, ιδος, ἡ}{Lycis, a comic poet}
\vocabentry{μετα-βάλλω}{move over (prob. from one shoulder to the other)}
\vocabentry{νή}{(yes) by.., with acc.;  with γε 'yes indeed'}
\vocabentry{πιέζω}{to press, squeeze; oppress, distress}
\vocabentry{πλήν}{except}
\vocabentry{σκεῦος, εος, τό}{a vessel; bag, baggage}
\vocabentry{τοσοῦτος, ὁ}{so large, so tall}
\vocabentry{φυλάσσω}{to keep guard; (med.) avoid}
\vocabentry{χεζητιάω}{(<χέζω to shit) need to shit}
\vocabentry{χολή, ἡ}{gall, bile; (cause of) gall, bile}
\end{multicols}

\vspace{-1.5em}
\noindent\rule{\linewidth}{0.4pt}
\vspace{-2em}

\begin{multicols}{2}
\begin{parcolumns}[colwidths={1=1.5em, 2=0.9\linewidth}]{2}
\small

\glossline{}{%
  \glossmix{}{The comedy was first performed at the Lenaea in early 405 BCE. Two actors enter. One (Dionsyus) is dressed in a full-length yellow dress with a lion-skin and holds a club. The other (Xanthias), his slave, is sitting on a donkey and has a lot of luggage hanging from a pole over his shoulders. }
}

\glossline{1}{%
  \glossmix{}{"Master, should I say one of the usual things that the spectators always laugh at?"}
  \glossmix{Εἴπω}{Deliberative subj., \textit{should I say}}\\
  \glossmix{εἰωθότων}{< εἴωθα}%
}

\glossline{3}{%
  \glossmix{ὅ τι}{ὅς τις, whatever (you like)}\\
  \glossmix{πιέζομαι}{i.e .a tired old joke, perhaps the beginning of the constipation jokes.}
}

\glossline{4}{%
  \glossmix{φύλαξαι}{aor. impv. mid. φυλάσσω}\\
  \glossmix{πάνυ... ἐστ’ ἤδη χολή}{\textit{it's entirely by now a source of bile, makes me sick}}
}

\glossline{5}{%
    \glossmix{μηδ’}{μή at the start of a question expects a negative response.}\\
    \glossmix{πλήν γ’...}{i.e. \textit{[say anything], except...}}\\
    \glossmix{ὡς}{exclamatory, \textit{how}}
}

\glossline{6}{%
    \glossmix{τί δαί;}{\textit{What then?}}\\
    \glossmix{τὸ πάνυ γέλοιον}{\textit{the really funny [joke]}}
}

\glossline{7}{%
  \glossmix{θαρρῶν}{assimilated form of θαρσῶν, i.e. pres. pple. θαρσέω. \textit{Yes by Zeus, [do say it], courageously}}\\
  \glossmix{ὅπως μὴ 'ρεῖς}{ὅπως + fut. indicative = impv. 'ρεῖς = ἐρεῖς < λέγω. \textit{Don't say}.}\\
  \glossmix{τὸ τί;}{\textit{the what [joke]?} Interrogatives can take an article when asking about an already mentioned object, Smyth 1186}
}

\glossline{8}{%
    \glossmix{τἀνάφορον}{τὸ ἀνάφορον}\\
    \glossmix{ὅτι}{assume a verb of speaking. \textit{[saying] as you shift your pole that you need to shit}}
}

\glossline{10}{
    \glossmix{μηδ’}{continuing the prohibition of ὅπως v.7. \textit{and don't [say]...}, i.e. \textit{and [I should] not [say]...}}
}

\glossline{10}{%
    \glossmix{καθαιρήσει}{fut. καθαιρέω}\\
    \glossmix{ἀποπαρδήσομαι}{fut. ἀποπέρδομαι. future most vivid condition with fut. ind. in protasis.}
}

\glossline{11}{%
    \glossmix{μὴ δῆθ᾽...}{\textit{just don't [say that]...}}
    \glossmix{πλήν γ᾽ ὅταν μέλλω 'ξεμεῖν}{\textit{except whenever I'm going to puke}}
}

\glossline{12}{%
    \glossmix{...ταῦτα τὰ σκεύη φέρειν...}{Metatheatrical humor. Xanthias complains: why is he presented as a typical comic slave with typical baggage if he can't make the typical slave jokes?}
}

\glossline{13}{%
    \glossmix{μηδὲν ὧνπερ}{μηδὲν τουτῶν ἅπερ... Partitive genitive and assimilation.}\\
    \glossmix{Φρύνιχος... Λύκις... κἀμειψίας}{Phrynichus, Lukis, and Ameipsias were three comic poets and competitors of Aristophanes. Phrynichus' \textit{Muses} took second place after \textit{Frogs} in 405.}
}

\glossline{14}{%
    \glossmix{κἀμειψίας}{καὶ Ἀμειψίας.}
}

\glossline{15}{%
    \glossmix{}{}
}

\end{parcolumns}
\end{multicols}

\newpage

\begin{spacing}{1.5}

\begin{tabularx}{\textwidth}{@{}lXr@{}}
  \textit{Δι.} & μή νυν ποιήσῃς· ὡς ἐγὼ θεώμενος, & \\
  & ὅταν τι τούτων τῶν σοφισμάτων ἴδω, & \\
  & πλεῖν ἢ 'νιαυτῷ πρεσβύτερος ἀπέρχομαι. & \\
  \textit{Ξα.} & ὦ τρισκακοδαίμων ἄρ᾽ ὁ τράχηλος οὑτοσί, & \\
  & ὅτι θλίβεται μέν, τὸ δὲ γέλοιον οὐκ ἐρεῖ. & 20 \\
  \textit{Δι.} & εἶτ᾽ οὐχ ὕβρις ταῦτ᾽ ἐστὶ καὶ πολλὴ τρυφή, & \\
  & ὅτ᾽ ἐγὼ μὲν ὢν Διόνυσος υἱὸς Σταμνίου & \\
  & αὐτὸς βαδίζω καὶ πονῶ, τοῦτον δ᾽ ὀχῶ, & \\ 
  & ἵνα μὴ ταλαιπωροῖτο μηδ᾽ ἄχθος φέροι; & \\
  \textit{Ξα.} & οὐ γὰρ φέρω 'γώ; & \\
  \textit{Δι.} & \hspace*{7.5em}πῶς φέρεις γὰρ ὅς γ᾽ ὀχεῖ; & 25 \\
  \textit{Ξα.} & φέρων γε ταυτί. & \\
  \textit{Δι.} & \hspace*{6.5em}τίνα τρόπον; & \\
  \textit{Ξα.} & \hspace*{12em}βαρέως πάνυ. & \\
  \textit{Δι.} & οὔκουν τὸ βάρος τοῦθ᾽ ὃ σὺ φέρεις ὄνος φέρει; & \\
  \textit{Ξα.} & οὐ δῆθ᾽ ὅ γ᾽ ἔχω 'γὼ καὶ φέρω μὰ τὸν Δί᾽ οὔ. & \\
  \textit{Δι.} & πῶς γὰρ φέρεις, ὅς γ᾽ αὐτὸς ὑφ᾽ ἑτέρου φέρει; & \\
  \textit{Ξα.} & οὐκ οἶδ᾽: ὁ δ᾽ ὦμος οὑτοσὶ πιέζεται. & 30 \\  
\end{tabularx}

\end{spacing}

\newpage

\begin{multicols}{2}
\small % roughly 9pt
\vocabentry{ἄχθος, εος, τό}{a weight, burden, load}
\vocabentry{ἀπέρχομαι}{to go away, depart from}
\vocabentry{βάρος, εος, τό}{weight}
\vocabentry{βαρέως}{(adv.) heavily,  < βαρύς}
\vocabentry{ἐνιαυτός, ὁ}{year}
\vocabentry{Ζεύς, ος, ὁ (acc. Δία)}{Zeus}
\vocabentry{θλίβω}{to press, squeeze, pinch}
\vocabentry{ὄνος, ὁ, ἡ}{an ass, donkey}
\vocabentry{οὔκουν}{certainly not; (in questions) ... not ..., expecting yes}
\vocabentry{ὀχέω}{to hold fast; let (another) ride, mount; (mid). ride}
\vocabentry{πιέζω}{to press, squeeze; oppress, distress}
\vocabentry{πονέω}{to work hard, do work, suffer toil}
\vocabentry{πρεσβύτερος, α, ον}{older (comp. πρέσβυς)}
\vocabentry{σόφισμα, ματος, τό}{any skilful act; sophism; stage-trick}
\vocabentry{Σταμνίας, ὁ}{(Comic proper noun) Wine-jar, < στάμνος wine-jar }
\vocabentry{ταλαιπωρέω}{to go through hard labour, to suffer hardship; (pass.) to be distressed, suffer hardship}
\vocabentry{τράχηλος, ὁ}{the neck, throat}
\vocabentry{τρισκακοδαίμων, ων, ον}{thrice unlucky}
\vocabentry{τρυφή, ἡ}{softness, delicacy, daintiness}
\vocabentry{ὕβρις, εως, ἡ}{wanton violence; violation, outrage}
\vocabentry{υἱός, ὁ}{a son}
\vocabentry{ὦμος, ὁ}{shoulder (with the upper arm)}
\end{multicols}

\vspace{-1.5em}
\noindent\rule{\linewidth}{0.4pt}
\vspace{-2em}

\begin{multicols}{2}
\begin{parcolumns}[colwidths={1=1.5em, 2=0.9\linewidth}]{2}
\small

\glossline{16}{%
  \glossmix{νυν}{enclitic with commands, \textit{come now}}\\
  \glossmix{ὡς}{as, since}\\
  \glossmix{θεώμενος}{\textit{ as a spectator}; Athenian comedies were performed at festivals to Dionysus and a statue of the god was placed in the theater.}
}
\glossline{18}{%
  \glossmix{πλεῖν}{= πλεῖον, πλέον, i.e. neut. sg. of πλείων. Adverbial accusative. The noun after πλεῖν ἢ retains its case and number from its use in the rest of the sentence, see Smyth 1074. Cf. 90, 91.}\\
  \glossmix{'νιαυτῷ}{= ἐνιαυτῷ. dat. of degree of difference. \textit{by more than a year}. i.e. time moves very slowly for Dionysus when he listens to bad jokes.}
}
\glossline{19}{%
  \glossmix{οὑτοσί}{deictic ('pointing') iota as suffix to οὗτος, \textit{this here}}. Cf. ταυτί, v. 26.\\
  \glossmix{τρισκακοδαίμων ἄρ᾽ ὁ τράχηλος οὑτοσί}{nominatives, supply ἐστιν. \textit{thrice-unlucky is this...}}
}
\glossline{21}{%
    \glossmix{εἶτ᾽}{εἶτα}\\
    \glossmix{οὐχ}{Questions beginning with οὐ expect a positive response. Cf. v. 25. \textit{Then aren't these things... }}
}

\glossline{22}{%
    \glossmix{ὅτ'}{= ὅτε. "the iota of ὅτι is never elided in Attic" (Stanford).}\\
    \glossmix{Σταμνίου}{Dionysus is the son of Zeus, but for humorous effect here Aristophanes invents Stamnios(/as) derived from σταμνός ("wine jar").}
}

\glossline{23}{%
  \glossmix{τοῦτον}{i.e. Xanthias}\\
  \glossmix{μὴ... μηδ᾽}{\textit{not... nor}}
  }

\glossline{24}{%
    \glossmix{ταλαιπωροῖτο... φέροι}{Normally the subjunctive is used in a purpose clause after a primary sequence main verb. In this case, the optative is used because ὀχῶ implies a reference to the past ("I let you mount in the past and now you ride"). Cf. Smyth 2200}
}
  
\glossline{25}{%
  \glossmix{γὰρ}{in abrupt questions, \textit{what, why}; \textit{What, I'm not the one carrying?}. Stanford sees these joke as parodies of sophistic argumentation about the active/passive voice, e.g. Euthyphro. Cf. 17}\\
  \glossmix{ὀχεῖ}{2s mid. ὀχέω. \textit{you who are riding [lit. being held]}}
}
\glossline{26}{%
  \glossmix{ταυτί}{ταῦτα and deictic iota, \textit{these things [in front of us]}}\\
  \glossmix{Τίνα τρόπον}{\textit{How?}}\\
  \glossmix{βαρέως πάνυ}{\textit{[carrying them] very heavily}}
}
\glossline{27}{%
  \glossmix{φέρει}{Active, unlike 29 φέρει}
}

\glossline{28}{%
    \glossmix{Δι’}{Δίa < Ζεύς}
    \glossmix{οὐ δῆθ’... μὰ τὸν Δι’ οὔ}{An extremely emphatic negative.}\\
    \glossmix{ὅ γ'... φέρω}{relative clause introduced by ὅ, \textit{not what I'm holding and carrying at least}}
}

\glossline{29}{%
    \glossmix{ὅς γ'... φέρει}{2s pass., \textit{you who... are being carried}}
}

\end{parcolumns}
\end{multicols}

\newpage

\begin{spacing}{1.5}

\begin{tabularx}{\textwidth}{@{}lXr@{}}
  \textit{Δι.} & σὺ δ᾽ οὖν ἐπειδὴ τὸν ὄνον οὐ φῄς σ᾽ ὠφελεῖν, & \\
  & ἐν τῷ μέρει σὺ τὸν ὄνον ἀράμενος φέρε. & \\
  \textit{Ξα.} & οἴμοι κακοδαίμων· τί γὰρ ἐγὼ οὐκ ἐναυμάχουν; & \\
  & ἦ τἄν σε κωκύειν ἂν ἐκέλευον μακρά. & \\
  \textit{Δι.} & κατάβα πανοῦργε. καὶ γὰρ ἐγγὺς τῆς θύρας & 35 \\
  & ἤδη βαδίζων εἰμὶ τῆσδ᾽, οἷ πρῶτά με & \\
  & ἔδει τραπέσθαι. παιδίον, παῖ, ἠμί, παῖ. & \\
\end{tabularx}

\noindent\textit{\MakeUppercase{\textls[200]{Ἡρακλῆς}}}

\begin{tabularx}{\textwidth}{@{}lXr@{}}
  & τίς τὴν θύραν ἐπάταξεν; ὡς κενταυρικῶς & \\
  & ἐνήλαθ᾽ ὅστις· εἰπέ μοι τουτὶ τί ἦν; & \\
  \textit{Δι.} & ὁ παῖς. & \\
  \textit{Ξα.} & \hspace*{3em}τί ἔστιν; & \\
  \textit{Δι.} & \hspace*{6.5em}οὐκ ἐνεθυμήθης; & \\
  \textit{Ξα.} & \hspace*{13.5em}τὸ τί; & 40 \\
  \textit{Δι.} & ὡς σφόδρα μ᾽ ἔδεισε. & \\
  \textit{Ξα.} & \hspace*{8.5em}νὴ Δία μὴ μαίνοιό γε. & \\
  \textit{Ἡρ.} & οὔ τοι μὰ τὴν Δήμητρα δύναμαι μὴ γελᾶν· & \\
  & καίτοι δάκνω γ᾽ ἐμαυτόν· ἀλλ᾽ ὅμως γελῶ. & \\
  \textit{Δι.} & ὦ δαιμόνιε πρόσελθε· δέομαι γάρ τί σου. & \\
\end{tabularx}

\end{spacing}

\newpage

\begin{multicols}{2}
\small % roughly 9pt
\vocabentry{αἴρω}{to lift; (mid.) to raise, lift, pick up}
\vocabentry{δαιμόνιος, α, ον}{of/belonging to a δαίμων; marvelous; (voc.) good sir/lady}
\vocabentry{δάκνω}{to bite}
\vocabentry{δείδω}{to fear}
\vocabentry{δέομαι}{to need, want (w. gen. of person and acc. of thing)}
\vocabentry{ἐγγύς}{(adv.) near, nigh, at hand}
\vocabentry{ἐν-άλλομαι}{to leap in}
\vocabentry{ἐν-θυμέομαι}{to lay to heart, ponder; notice, consider}
\vocabentry{ἠμί}{to say}
\vocabentry{καίτοι}{and indeed, and further; and yet}
\vocabentry{κακο-δαίμων, ον}{ill-fated; (freq in Com.) poor devil!}
\vocabentry{κατα-βαίνω}{to step down, go}
\vocabentry{κελεύω}{to urge; to order}
\vocabentry{Κενταυρικός}{(adv.) like a Centaur}
\vocabentry{κωκύω}{to shriek, cry, wail}
\vocabentry{μά}{(in oaths) by (+ acc. of deity); ναὶ μὰ yes by, οὐ μὰ no by}
\vocabentry{μαίνομαι}{to rage, be crazy}
\vocabentry{μακρός, ός, ά, όν}{long}
\vocabentry{ναυ-μαχέω}{to fight by sea}
\vocabentry{οἷ}{whither}
\vocabentry{ὅμως}{nevertheless, still}
\vocabentry{ὄνος, ὁ, ἡ}{an ass}
\vocabentry{παιδίον, τό}{a child; young slave}
\vocabentry{παν-οῦργος, ον}{willing to do anything, tricky; (in comedy) general term of abuse}
\vocabentry{πατάσσω}{to beat, knock}
\vocabentry{προσ-έρχομαι}{to come}
\vocabentry{τρέπω}{turn; (mid). turn or betake oneself, go}
\vocabentry{σφόδρα}{(adv.) very, very much, exceedingly, violently}
\vocabentry{ὠφελέω}{to help, aid, assist, to be of use}
\end{multicols}

\vspace{-1.5em}
\noindent\rule{\linewidth}{0.4pt}
\vspace{-2em}

\begin{multicols}{2}
\begin{parcolumns}[colwidths={1=1.5em, 2=0.9\linewidth}]{2}
\small

\glossline{31}{%
 \glossmix{οὐ φῄς}{\textit{deny}, not  \textit{don't say}.}
}

\glossline{32}{%
 \glosstxt{ἐν τῷ μέρει}{in turn}\\
 \glossmix{ἀράμενος}{pple. aor. mid. αἴρω}
}

\glossline{33}{%
 \glossmix{οἴμοι κακοδαίμων}{bemoaning himself, \textit{poor unlucky [me]!}}\\
 \glossmix{τί γὰρ ἐγὼ οὐκ ἐναυμάχουν;}{Slaves who fought in the naval battle at Arginusai in 406 BCE had been granted their freedom approximately six months before the performance of the Frogs in early 405.}
}

\glossline{34}{%
  \glossmix{ἦ τἄν}{ἦ τοι ἄν, \textit{Then, you know...}. The repetition of ἄν is not unusual.}\\
  \glossmix{μακρά}{Adverbial/internal accusative as usual adv. for μακρός, \textit{intensely}.}\\
  \glossmix{}{\textit{I would tell you to wail intensely...}, i.e. if I were free, I would tell you to go yell and moan, and I wouldn't care in the least about your distress.}
}

\glossline{35}{%
 \glossmix{κατάβα}{aor. impv. καταβαίνω, \textit{dismount}. The actors arrive at a door in the stage building. After this point there is no further mention of the donkey, which is presumably led offstage by a mute actor.}
}

\glossline{36}{%
 \glossmix{βαδίζων εἰμί}{Periphrastic construction, cf. Smyth 1961}\\
 \glossmix{τῆσδ'}{with τῆς θύρας, \textit{this one here}}\\
 \glossmix{πρῶτά}{adverbial accusative.}\\
 \glossmix{οἷ πρῶτά με / ἔδει τραπέσθαι}{\textit{to where (i.e. the door) I had to go in the first place}}
}

\glossline{37}{%
 \glossmix{τραπέσθαι}{aor. inf. mid. τρέπω}\\
 \glossmix{παιδίον, παῖ}{Referring to a slave expected to open the door.}
}

\glossline{38}{%
 \glossmix{}{Heracles himself unexpectedly opens the door.}\\
 \glossmix{ὡς κενταυρικῶς}{\textit{How centaur-ically, how much like a centaur}. Centaurs were famously violent, e.g. in the battle between the Centaurs and the Lapiths.}
}

\glossline{39}{%
 \glossmix{ἐνήλαθ'}{i.e. ἐνήλατο, aor. ἐνάλλομαι.}\\
 \glossmix{ὅστις}{\textit{whoever [it was]}}\\
 \glossmix{τουτὶ τί ἦν}{\textit{what's this thing here?}. Impf. slightly difficult (Stanford claims 'imperfect of intention')}
}

\glossline{40}{%
 \glossmix{ὁ παῖς}{\textit{slave}, to Xanthias. Masters often use the nominative rather than the vocative in addressing slaves, cf. 521}\\
  \glossmix{ἐνεθυμήθης}{aor. ἐνθυμέομαι, deponent}
}

\glossline{41}{%
 \glossmix{ὡς...ἔδεισε}{Sarcastically addressing the audience or Heracles.}\\
 \glossmix{ἔδεισε}{aor. δείδω}\\
 \glossmix{μὴ μαίνοιό γε}{2s. pres. opt. mid. μαίνομαι. Fear clause picking up on ἔδεισε, \textit{yes by Zeus, [afraid] that you're crazy at least}}
}

\glossline{42}{%
 \glossmix{οὔ τοι μὰ... δύναμαι μὴ γελᾶν}{double negative. \textit{No by... I am not able not to laugh.}}
}

\glossline{43}{%
 \glossmix{καίτοι δάκνω γ᾽ ἐμαυτόν}{i.e. biting his tongue or lips to try to keep in his laughter.}
}

\glossline{44}{%
 \glossmix{τί}{indefinite, \textit{something}; acute accent from enclitic σου.}
}

\glossline{45}{%
 \glossmix{δαιμόνιε}{"The exact meaning of the vocative δαιμόνιε is disputed... in Aristophanes δαιμόνιε is normally used to superiors and always has an element of deference" (Dickey 1996: 141)}
}

\end{parcolumns}
\end{multicols}

\newpage

\begin{spacing}{1.5}

\begin{tabularx}{\textwidth}{@{}lXr@{}}
  \textit{Ἡρ.} & Ἀλλ' οὐχ οἷός τ' εἴμ' ἀποσοβῆσαι τὸν γέλων & 45 \\
  & ὁρῶν λεοντῆν ἐπὶ κροκωτῷ κειμένην. & \\
  & Τίς ὁ νοῦς; Τί κόθορνος καὶ ῥόπαλον ξυνηλθέτην; & \\
  & Ποῖ γῆς ἀπεδήμεις; & \\
  \textit{Δι.} & \hspace*{8em}Ἐπεβάτευον Κλεισθένει. & \\
  \textit{Ἡρ.} & Κἀναυμάχησας; & \\
  \textit{Δι.} & \hspace*{7em}Καὶ κατεδύσαμέν γε ναῦς & \\
  & τῶν πολεμίων ἢ δώδεκ' ἢ τρεισκαίδεκα. & 50 \\
  \textit{Ἡρ.} & Σφώ; & \\
  \textit{Δι.} & \hspace*{2.5em}Νὴ τὸν Ἀπόλλω. & \\
  \textit{Ξα.} & \hspace*{9.5em}Κᾆτ' ἔγωγ' ἐξηγρόμην. & \\
  \textit{Δι.} & Καὶ δῆτ' ἐπὶ τῆς νεὼς ἀναγιγνώσκοντί μοι & \\
  & τὴν Ἀνδρομέδαν πρὸς ἐμαυτὸν ἐξαίφνης πόθος & \\
  & τὴν καρδίαν ἐπάταξε πῶς οἴει σφόδρα. & \\
  \textit{Ἡρ.} & Πόθος; πόσος τις; & \\
  \textit{Δι.} & \hspace*{7.5em}Σμικρός, ἡλίκος Μόλων. & 55 \\
  \textit{Ἡρ.} & Γυναικός; & \\
  \textit{Δι.} & \hspace*{4em}Οὐ δῆτ'. & \\
  \textit{Ἡρ.} & \hspace*{7.5em}Ἀλλὰ παιδός; & \\
  \textit{Δι.} & \hspace*{13em}Οὐδαμῶς. & \\
\end{tabularx}

\end{spacing}

\newpage

\begin{multicols}{2}
\small % roughly 9pt
\vocabentry{ἀνα-γιγνώσκω}{to know well, know certainly; to read}
\vocabentry{Ἀνδρομέδα, ἡ}{Andromeda; heroine, lost tragedy by Euripides, produced 413/2}
\vocabentry{ἀπο-δημέω}{to be away from home, be abroad}
\vocabentry{ἀπο-σοβέω}{to scare away; (metaph.) to keep off}
\vocabentry{γέλως, ωτος, ὁ (poet. acc. γέλων)}{laughter}
\vocabentry{δώ-δεκα}{twelve}
\vocabentry{ἐξ-αίφνης}{(adv.) suddenly}
\vocabentry{ἐξ-εγείρω}{to awaken}
\vocabentry{ἐπι-βατεύω}{to serve as a marine (ἐπιβάτης)}
\vocabentry{ἡλίκος, ός, ά, όν}{as big as, of the same age as; how great, what size..!}
\vocabentry{καρδία, ἡ}{the heart}
\vocabentry{κατά-δύω}{to go down; (causal) to make to sink, sink}
\vocabentry{Κλεισθένης, ους, ὁ}{Kleisthenes, frequently attacked in comedy for his alleged effeminacy}
\vocabentry{κόθορνος, ὁ}{high boot associated with women and Dionysus, in post-classical theater worn by tragic actors}
\vocabentry{κροκωτός, ὁ}{a saffron-colored robe worn by women on special occasions < κροκωτός saffron-dyed}
\vocabentry{λεοντέη, ἡ}{a lion's skin}
\vocabentry{Μόλων, οντος, ὁ}{Molon}
\vocabentry{ναυ-μαχέω}{to fight by sea}
\vocabentry{νή}{Particle of strong affirmation, with acc. of the divinity invoked}
\vocabentry{οἷός τ' εἴμ' + inf.}{to be able (to do)}
\vocabentry{οὐδαμῶς}{(adv.) in no way}
\vocabentry{πατάσσω}{to beat, knock}
\vocabentry{πόθος, ὁ}{a longing, yearning, fond desire}
\vocabentry{πολέμιος, ός, ά, όν}{hostile; enemy}
\vocabentry{ῥόπαλον, τό}{a club, cudgel}
\vocabentry{σμικρός, ός, ά, όν}{= μικρός}
\vocabentry{συν-έρχομαι (Attic ξυν-)}{come together, meet}
\vocabentry{Σφώ}{nom. dual. of σύ, you two}
\vocabentry{σφόδρα}{(adv.) very, very much, exceedingly, violently}
\vocabentry{τρει-καί-δεκα, ὁ, ἡ}{thirteen}
\end{multicols}

\vspace{-1.5em}
\noindent\rule{\linewidth}{0.4pt}
\vspace{-2em}

\begin{multicols}{2}
\begin{parcolumns}[colwidths={1=1.5em, 2=0.9\linewidth}]{2}
\small

\glossline{45}{%
 \glossmix{οἷός τ' εἴμ'}{See vocab}
}

\glossline{47}{%
 \glossmix{Τίς ὁ νοῦς}{\textit{What's the idea}}\\
 \glossmix{ξυνηλθέτην}{3rd person dual aorist of συνέρχομαι, \textit{come together}}
}

\glossline{48}{%
 \glossmix{Ποῖ γῆς ἀπεδήμεις}{\textit{Where in the world did you go off to?}}\\
  \glossmix{ἀπεδήμεις}{impf. ἀποδημέω}\\
 \glossmix{Ἐπεβάτευον}{\textit{I served as a marine for Cleisthenes.}, but possibly a pun on ἐπιβαίνω \textit{I was mounting Cleisthenes}}
}

\glossline{49}{%
 \glossmix{Κἀναυμάχησας}{= καὶ ἐναυμάχησας}
}

\glossline{51}{%
 \glossmix{Σφώ}{Dionysus and Xanthias, or possibly Dionysus and Kleisthenes}\\
 \glossmix{Κᾆτ}{= καὶ εἶτα}\\
 \glossmix{ἐξηγρόμην}{aor. mid. ἐξεγείρω. "Xanthias sarcastically completes D.'s brag" (Stanford).}
}

\glossline{52}{%
 \glossmix{ἀναγιγνώσκοντι}{Apparently private reading of a tragedy, as opposed to seeing a production. The implications for Athenian reading culture (i.e. did many Athenians read dramas?) are controversial. See Schmitz 2023.}\\
 \glossmix{μοι}{Dative of interest, possibly going with τὴν καρδίαν}
}

\glossline{53}{%
 \glossmix{τὴν Ἀνδρομέδαν}{The title of a (non-extant) play by Euripides.}\\
 \glossmix{πῶς οἴει}{Like πῶς δοκεῖς, originally something like \textit{how are you thinking} but used equivalently as λίαν \textit{extremely} (LSJ s.v. πῶς III.5)}
}

\glossline{54}{%
 \glossmix{ἐπάταξε}{Aor. πατάσσω, like l. 38.}
}

\glossline{55}{%
 \glossmix{πόσος τις}{\textit{about how big}. Indefinite τις with numbers and size words softens their precision (LSJ s.v. IX).}\\
 \glossmix{Μόλων}{A famous actor, apparently a large man.}\\
 \glossmix{ἡλίκος Μόλων}{sc. ἐστίν.}
} 

\glossline{56}{%
 \glossmix{Γυναικός}{Sc. πόθος, i.e. \textit{[Longing] for a woman}}\\
 \glossmix{Οὐ δῆτ'. Ἀλλὰ}{\textit{Certainly not. Well, then...} (Stanford)}
}

\end{parcolumns}
\end{multicols}

\newpage

\begin{spacing}{1.5}

\begin{tabularx}{\textwidth}{@{}lXr@{}}
  \textit{Ἡρ.} & Ἀλλ' ἀνδρός; & \\
  \textit{Δι.} & \hspace*{5.5em}Ἀπαπαῖ. & \\
  \textit{Ἡρ.} & \hspace*{9em}Ξυνεγένου τῷ Κλεισθένει; & \\
  \textit{Δι.} & Μὴ σκῶπτέ μ', ὦδέλφ'· οὐ γὰρ ἀλλ' ἔχω κακῶς· & \\
  & τοιοῦτος ἵμερός με διαλυμαίνεται. & \\
  \textit{Ἡρ.} & Ποῖός τις, ὦδελφίδιον; & \\
  \textit{Δι.} & \hspace*{9.5em}Οὐκ ἔχω φράσαι. & 60 \\
  & Ὅμως γε μέντοι σοι δι' αἰνιγμῶν ἐρῶ. & \\
  & Ἤδη ποτ' ἐπεθύμησας ἐξαίφνης ἔτνους; & \\
  \textit{Ἡρ.} & Ἔτνους; Βαβαιάξ, μυριάκις γ' ἐν τῷ βίῳ. & \\
  \textit{Δι.} & Ἆρ' ἐκδιδάσκω τὸ σαφὲς, ἢ 'τέρᾳ φράσω; & \\
  \textit{Ἡρ.} & Μὴ δῆτα περὶ ἔτνους γε· πάνυ γὰρ μανθάνω. & 65 \\
  \textit{Δι.} & Τοιουτοσὶ τοίνυν με δαρδάπτει πόθος & \\
  & Εὐριπίδου. & \\
  \textit{Ἡρ.} & \hspace*{5em}Καὶ ταῦτα τοῦ τεθνηκότος; & \\
  \textit{Δι.} & Κοὐδείς γέ μ' ἂν πείσειεν ἀνθρώπων τὸ μὴ οὐκ & \\
  & ἐλθεῖν ἐπ' ἐκεῖνον. & \\
  \textit{Ἡρ.} & \hspace*{7.5em}Πότερον εἰς Ἅιδου κάτω; & \\
  \textit{Δι.} & Καὶ νὴ Δί' εἴ τί γ' ἔστιν ἔτι κατωτέρω. & 70 \\
\end{tabularx}

\end{spacing}

\newpage

\begin{multicols}{2}
\small % roughly 9pt
\vocabentry{ἀδελφίδιον, τό}{diminutive of ἀδελφός, bro}
\vocabentry{ἀδελφός, ὁ}{brother}
\vocabentry{ἀπαπαῖ}{"an inarticulate expression of grief or pain" (Stanford)}
\vocabentry{αἰνιγμός, ὁ}{a riddle; allusion, allegory}
\vocabentry{βαβαιάξ}{strengthened form of βαβαί}
\vocabentry{βαβαί}{exclamation of a reaction to misfortune}
\vocabentry{βίος, ὁ}{life}
\vocabentry{δαρδάπτω}{to devour}
\vocabentry{δια-λυμαίνομαι}{to maltreat shamefully, ruin}
\vocabentry{ἐξαίφνης}{(adv.) suddenly}
\vocabentry{ἐκ-διδάσκω}{to teach thoroughly}
\vocabentry{ἐπι-θυμέω}{to set one's heart upon (a thing), long for (+ gen.)}
\vocabentry{ἔτνος, εος, τό}{thick soup}
\vocabentry{ἔχω}{to have; (+ inf.) to be able, can}
\vocabentry{ἵμερος, ὁ}{a longing}
\vocabentry{κατωτέρω}{(adv.) lower (comp. κατώ)}
\vocabentry{μανθάνω}{to learn; to understand}
\vocabentry{μυριάκις}{(adv.) ten thousand times}
\vocabentry{πόθος, ὁ}{a longing, yearning, fond desire}
\vocabentry{πότερον}{(adv.) introduces question with two alternatives}
\vocabentry{σκώπτω}{to mock, make fun of}
\vocabentry{συγ-γίγνομαι (Att. ξυγ-)}{to be with, to have sexual intercourse with}
\vocabentry{φράζω}{to point out, show; tell, declare, explain}
\end{multicols}

\vspace{-1.5em}
\noindent\rule{\linewidth}{0.4pt}
\vspace{-2em}

\begin{multicols}{2}
\begin{parcolumns}[colwidths={1=1.5em, 2=0.9\linewidth}]{2}
\small

\glossline{57}{%
 \glossmix{τῷ}{The article is used with people when the person is famous or has already been mentioned, Smyth 1136.}
 \glossmix{Ξυνεγένου}{= συνεγένου, aor. συγγίγνομαι. "'Did you do it with Kleisthenes?', implying 'Is it Kleisthenes you're longing for?'" (Dover)}
}

\glossline{58}{%
 \glossmix{ὦδέλφ}{= ὦ ἀδέλφε. Both Heracles and Dionysus were sons of Zeus.}\\
 \glossmix{οὐ γὰρ ἀλλ'}{\textit{because... really...}}\\
 \glossmix{ἔχω κακῶς}{ἔχω + adverb describes a person's state. \textit{I'm doing poorly, I have it bad}}
}

\glossline{60}{%
 \glossmix{ὦδελφίδιον}{ὦ ἀδελφίδιον}\\
 \glossmix{ἔχω φράσαι}{ἔχω + infinitive indicates ability (LSJ s.v. III).}
}

\glossline{61}{%
 \glossmix{Ὅμως γε μέντοι}{\textit{Nevertheless}}
 \glossmix{δι' αἰνιγμῶν}{\textit{through analogies}}
}

\glossline{62}{%
 \glossmix{ἔτνους}{gen. sg. ἔτνος (contracted).}
}

\glossline{63}{%
 \glossmix{Βαβαιάξ}{"Not so much a lip-smacking 'Oh, boy!' as 'Oh, how I wish I had some now!'" (Dover)}
}

\glossline{64}{%
 \glossmix{Ἆρ' ἐκδιδάσκω τὸ σαφὲς}{\textit{Am I spelling out what's clear}, i.e. \textit{am I being clear}}\\
 \glossmix{'τέρᾳ}{= ἐτέρᾳ. \textit{in another way}}\\
 \glossmix{φράσω}{Probably deliberative subjunctive, \textit{should I explain}, but perhaps future.}\\
 \glossmix{ἑτέρᾳ φράσω}{The scholia claim that this half-line comes from Euripides' Hypsipyle (ἔστι δὲ τὸ ἡμιστίχιον ἐξ Ὑψιπύλης Εὐριπίδου.)}
}

\glossline{65}{%
 \glossmix{Μὴ δῆτα}{responding to ἢ 'τέρᾳ φράσω; (not Ἆρ' ἐκδιδάσκω τὸ σαφὲς).}
}

\glossline{66}{%
 \glossmix{Τοιουτοσὶ}{emphatic form of Τοιοῦτος with deictic iota.}
}

\glossline{67}{%
 \glossmix{Εὐριπίδου.}{Prominent enjambment.}\\
 \glossmix{καὶ ταῦτα}{\textit{and that (lit. these things), and what's more}}\\
 \glossmix{τοῦ τεθνηκότος}{pf. pple. θνῄσκω, describing Euripides, \textit{for the dead man}. Heracles is disgusted that Dionysus longs for a corpse.}
}

\glossline{68}{%
 \glossmix{Κοὐδείς}{Καὶ οὐδείς}\\
 \glossmix{ἂν πείσειεν}{3s aor opt. πείθω, potential with ἄν.}
 \glossmix{τὸ μὴ οὐκ}{τὸ μὴ οὐκ + inf. after a verb of persuasion. \textit{could persuade me not to...}, Smyth 2749d}
}

\glossline{69}{%
 \glossmix{ἐπ' ἐκεῖνον}{\textit{after him, in search of him, to get him}}\\
 \glossmix{εἰς Ἅιδου}{common phrase in ep., trag., and Att., sc. οἴκῳ}\\
 \glossmix{Πότερον}{Introduces a question with two alternatives, here with second alternative omitted. \textit{Down into Hades, or...}. Dionysus picks up on the omitted alternative and in a sense fills it in.}
}

\glossline{70}{%
 \glossmix{ἔτι κατωτέρω}{\textit{even lower}, i.e. if there is any place even lower than Hades. Tartarus is sometimes thought to be below Hades (e.g \textit{Il}. 8.16) but Dionysus speaks generally here.}\\
}

\end{parcolumns}
\end{multicols}

\newpage

\begin{spacing}{1.5}

\begin{tabularx}{\textwidth}{@{}lXr@{}}
  \textit{Ἡρ.} & Τί βουλόμενος; & \\
  \textit{Δι.} & \hspace*{6.5em}Δέομαι ποητοῦ δεξιοῦ. & \\
  & Οἱ μὲν γὰρ οὐκέτ' εἰσίν, οἱ δ' ὄντες κακοί. & \\
  \textit{Ἡρ.} & Τί δ'; Οὐκ Ἰοφῶν ζῇ; & \\
  \textit{Δι.} & \hspace*{8.5em}Τοῦτο γάρ τοι καὶ μόνον & \\
  & ἔτ' ἐστὶ λοιπὸν ἀγαθόν, εἰ καὶ τοῦτ' ἄρα· & \\
  & οὐ γὰρ σάφ' οἶδ' οὐδ' αὐτὸ τοῦθ' ὅπως ἔχει. & 75 \\
  \textit{Ἡρ.} & Εἶτ' οὐ Σοφοκλέα πρότερον ὄντ' Εὐριπίδου & \\
  & μέλλεις ἀναγαγεῖν, εἴπερ ἐκεῖθεν δεῖ σ' ἄγειν; & \\
  \textit{Δι.} & Οὔ, πρίν γ' ἂν Ἰοφῶντ', ἀπολαβὼν αὐτὸν μόνον, & \\
  & ἄνευ Σοφοκλέους ὅ τι ποεῖ κωδωνίσω. & \\
  & Κἄλλως ὁ μέν γ' Εὐριπίδης πανοῦργος ὢν & 80 \\
  & κἂν ξυναποδρᾶναι δεῦρ' ἐπιχειρήσειέ μοι· & \\
  & ὁ δ' εὔκολος μὲν ἐνθάδ', εὔκολος δ' ἐκεῖ. & \\
  \textit{Ἡρ.} & Ἀγάθων δὲ ποῦ 'στιν; & \\
  \textit{Δι.} & \hspace*{9em}Ἀπολιπών μ' ἀποίχεται, & \\
  & ἀγαθὸς ποητὴς καὶ ποθεινὸς τοῖς φίλοις. & \\
\end{tabularx}

\end{spacing}

\newpage

\begin{multicols}{2}
\small % roughly 9pt
\vocabentry{ἄνευ}{without + gen.}
\vocabentry{Ἀγάθων, οντος, ὁ}{Agathon, famous tragedian}
\vocabentry{ἀν-άγω}{to lead up; to bring back}
\vocabentry{ἀπο-λαμβάνω}{to take from another; to take apart or aside}
\vocabentry{ἀπο-λείπω}{to leave behind; desert, abandon}
\vocabentry{ἀπ-οίχομαι}{to be gone away, to be far from}
\vocabentry{δεξιός, ά, όν}{right; skillful, clever}
\vocabentry{δέω}{to lack; (mid.) to need + gen.}
\vocabentry{εἴπερ}{if really, if indeed}
\vocabentry{ἐκεῖ}{(adv.) there, in that place}
\vocabentry{ἐκεῖθεν}{(adv.) from that place, thence}
\vocabentry{ἐπι-χειρέω}{to attempt, try}
\vocabentry{εὔ-κολος, ον}{easily satisfied, contented}
\vocabentry{ζῶ}{to live}
\vocabentry{Ἰοφῶν, ὁ}{Iophon, Sophocles' son, successful tragedian}
\vocabentry{κωδωνίζω}{to try, test (lit. prove by ringing) < κώδων bell}
\vocabentry{λοιπός, ά, όν}{remaining, left}
\vocabentry{πανοῦργος, ον}{ready to do anything, roguish}
\vocabentry{ποθεινός, ά, όν}{longed for, desired, much desired}
\vocabentry{ποιητής, οῦ, ὁ (Att. ποη-)}{maker; poet < ποιέω}
\vocabentry{σάφα}{(adv.) clearly, plainly, assuredly}
\vocabentry{Σοφοκλῆς, ὁ}{Sophocles, big 3 tragedian}
\vocabentry{συν-απο-διδράσκω (Att. ξυν-)}{to run away along with}
\end{multicols}

\vspace{-1.5em}
\noindent\rule{\linewidth}{0.4pt}
\vspace{-2em}

\begin{multicols}{2}
\begin{parcolumns}[colwidths={1=1.5em, 2=0.9\linewidth}]{2}
\small

\glossline{71}{%
 \glossmix{δεξιοῦ}{See vocab.}
}

\glossline{72}{%
 \glossmix{}{a quotation from Euripides' \textit{Oeneus}. \textit{σὺ δ' ὧδ' ἔρημος ξυμμάχων ἀπόλλυσαι·
  οἱ μὲν γὰρ οὐκ ἔτ' εἰσὶν, οἱ δ' ὄντες κακοὶ.}}\\
 \glossmix{Οἱ μὲν... οἱ δ'}{\textit{the ones [who were good]... and the ones...}.}\\
 \glossmix{ὄντες}{\textit{being [still alive]}, i.e. living poets}
}

\glossline{73}{%
 \glossmix{ζῇ}{3s pres. act. ind. ζῶ}\\
 \glossmix{Τοῦτο γάρ τοι καὶ μόνον...}{\textit{Yes, for this is still just the only good thing left, i.e. the fact that Iophon is alive.}}\\
 \glossmix{εἰ καὶ τοῦτ' ἄρα}{\textit{if it actually is [good]}. ἄρα "expressing the surprise attendant upon disillusionment... ἄρα in a conditional protasis denotes that the hypothesis is one of which the possibility has only just been realized: '\textit{If, after all}'" (GP 35, 37).}
}

\glossline{75}{%
 \glossmix{οὐ... οὐδ'}{redundant double negative, see Smyth 2761}\\
  \glossmix{αὐτὸ τοῦθ' ὅπως ἔχει}{\textit{how this thing itself is}, lit. \textit{this thing itself, how it is}. Prolepsis of the subject of the indirect question into the main sentence (i.e. the 'lilies-of-the-field' construction), Smyth 2182. ἔχω + adv. denoting the state of the subject, 'how it is.' Slight ambiguity as to what the antecedent of τοῦτο is: whether Iophon is good, or alive at all?}
}

\glossline{76}{%
 \glossmix{Εἶτ'}{Εἶτα}\\
 \glossmix{πρότερον ὄντ'}{\textit{since he's superior to, better than}. A metaphor from ranking, LSJ s.v. A.III}
}

\glossline{77}{%
 \glossmix{ἀναγαγεῖν}{aor. inf. ἀνάγω}
}

\glossline{78}{%
 \glossmix{πρίν ἄν}{LSJ II.2.: πρίν in sense of \textit{until} after a negative regularly takes ἄν + subj. \textit{No, not until I test...}. Dionysus accuses Iophon of leaning on his father for any merit in his dramas.}
}

\glossline{79}{%
 \glossmix{ποεῖ}{= ποιεῖ}
}

\glossline{80}{%
 \glossmix{Κἄλλως}{= καὶ ἄλλως, \textit{and anyway}}\\
 \glossmix{}{The sense is: I want to see what Iophon can do without Sophocles. Besides, [it will be easier to steal Euripides than Sophocles], since that old rogue Euripides might try to escape up here anyway}
}

\glossline{81}{%
 \glossmix{ξυναποδρᾶναι}{aor. inf. συναποδιδράσκω, elsewhere used of deserters and fugitive slaves. The ξυν- with the line-end μοι.}
}

\glossline{82}{%
 \glossmix{ὁ δ'}{change of subject, \textit{but he}, i.e. Sophocles}\\
 \glossmix{ἐνθάδ'... ἐκεῖ}{\textit{here}, i.e. in the land of the living and \textit{there}, i.e. in Hades}
}

\glossline{83}{%
 \glossmix{ἀπολιπών}{aor. pple. ἀπολείπω}\\
 \glossmix{Ἀπολιπών μ' ἀποίχεται}{Sometime before 405 Agathon had gone to the court of king Archelaos in Macedonia, like Euripides did in the final years of his life. Dionysus and Heracles speak about him here as though he were dead.}
}

\glossline{84}{%
 \glossmix{ποθεινὸς τοῖς φίλοις}{\textit{longed for by his friends},    prob. dat. of interest}
}

\end{parcolumns}
\end{multicols}

\newpage

\begin{spacing}{1.5}

\begin{tabularx}{\textwidth}{@{}lXr@{}}
  \textit{Ἡρ.} & Ποῖ γῆς ὁ τλήμων; & \\
  \textit{Δι.} & \hspace*{8em}Ἐς μακάρων εὐωχίαν. & 85 \\
  \textit{Ἡρ.} & Ὁ δὲ Ξενοκλέης; & \\
  \textit{Δι.} & \hspace*{7em}Ἐξόλοιτο νὴ Δία. & \\
  \textit{Ἡρ.} & Πυθάγγελος δέ; & \\
  \textit{Ξα.} & \hspace*{7em}Περὶ ἐμοῦ δ' οὐδεὶς λόγος & \\
  & ἐπιτριβομένου τὸν ὦμον οὑτωσὶ σφόδρα. & \\
  \textit{Ἡρ.} & Οὔκουν ἕτερ' ἔστ' ἐνταῦθα μειρακύλλια & \\
  & τραγῳδίας ποιοῦντα πλεῖν ἢ μύρια, & 90 \\
  & Εὐριπίδου πλεῖν ἢ σταδίῳ λαλίστερα; & \\
  \textit{Δι.} & Ἐπιφυλλίδες ταῦτ' ἐστὶ καὶ στωμύλματα, & \\
  & χελιδόνων μουσεῖα, λωβηταὶ τέχνης, & \\
  & ἃ φροῦδα θᾶττον, ἢν μόνον χορὸν λάβῃ, & \\
  & ἅπαξ προσουρήσαντα τῇ τραγῳδίᾳ. & 95 \\
  & Γόνιμον δὲ ποιητὴν ἂν οὐχ εὕροις ἔτι & \\
  & ζητῶν ἄν, ὅστις ῥῆμα γενναῖον λάκοι. & \\
  \textit{Ἡρ.} & Πῶς γόνιμον; & \\
  \textit{Δι.} & \hspace*{6em}Ὡδὶ γόνιμον, ὅστις φθέγξεται & \\
  & τοιουτονί τι παρακεκινδυνευμένον, & \\
  & “αἰθέρα Διὸς δωμάτιον,” ἢ “χρόνου πόδα,” & 100 \\

\end{tabularx}

\end{spacing}

\newpage

\begin{multicols}{2}
\small % roughly 9pt
\vocabentry{αἰθήρ, έρος, ἡ/ὁ}{ether, the heaven}
\vocabentry{ἅπαξ}{(adv.) once, once and for all}
\vocabentry{γενναῖος, α, ον}{high-born, noble; high-minded}
\vocabentry{γόνιμος, ον}{fruitful, fertile}
\vocabentry{δωμάτιον, τό}{a room, bedroom}
\vocabentry{ἐξ-όλλυμι}{to destroy utterly; (mid.) perish utterly}
\vocabentry{ἐπι-τρίβω}{to rub on the surface, to crush}
\vocabentry{ἐπι-φυλλίς, ίδος, ἡ}{the small grapes left for gleaners (though meaning disputed)}
\vocabentry{εὐ-ωχία, ἡ}{good cheer, feasting}
\vocabentry{θάσσων, ον (Att. θάττων)}{(neut. as adv.) very quickly, comp. ταχύς}
\vocabentry{λάλος, α, ον}{talkative, babbling, loquacious; irreg. comp. λαλίστερος}
\vocabentry{λάσκω}{to ring, rattle; to scream, shout; to utter}
\vocabentry{λωβητής, ὁ}{destroyer < λωβᾶσθαι damage, spoil}
\vocabentry{μάκαρ, αρος, ὁ}{blessed}
\vocabentry{μειρακύλλιον, τό}{dim. μειράκιον, little lad, kid}
\vocabentry{Μουσεῖον, τό}{shrine of the Muses; music-hall}
\vocabentry{μύριος, α, ον}{ten thousand ≠ μυρίος countless}
\vocabentry{Ξενοκλῆς, ὁ}{Xenokles, minor tragedian}
\vocabentry{οὔκουν}{certainly not; (in questions) ... not ..., expecting yes}
\vocabentry{παρα-κινδυνεύω}{to venture, risk}
\vocabentry{προσ-ουρέω}{urinate on; piss on (+ dat.)}
\vocabentry{Πυθάγγελος, ὁ}{Pythangelus, tragedian, otherwise unknown}
\vocabentry{ῥῆμα, ατος, τό}{that which is said or spoken, word, saying}
\vocabentry{στάδιον, τό}{a stade, = ca. 600 feet}
\vocabentry{στώμυλμα, ματος, τό}{chatterbox < στωμύλος talkative < στόμα mouth}
\vocabentry{σφόδρα}{(adv.) very, very much}
\vocabentry{τλήμων, ων, ον}{suffering, enduring; wretched, miserable}
\vocabentry{φθέγγομαι}{to utter a sound or voice, esp. speak loud and clear}
\vocabentry{φροῦδος, ός, ά, όν}{gone away, clean gone; (of persons) gone, fled, departed}
\vocabentry{χελιδών, όνος, ἡ}{swallow (bird)}
\vocabentry{ὦμος, ὁ}{shoulder (with the upper arm)}
\vocabentry{ὡδί}{(adv.) in the following way (ὧδε with deictic iota}
\end{multicols}

\vspace{-1.5em}
\noindent\rule{\linewidth}{0.4pt}
\vspace{-2em}

\begin{multicols}{2}
\begin{parcolumns}[colwidths={1=1.5em, 2=0.9\linewidth}]{2}
\small

\glossline{85}{%
 \glossmix{Ποῖ γῆς}{\textit{Where on earth}. Sc. verb of motion.}
}

\glossline{86}{%
 \glossmix{Ἐξόλοιτο}{aor. opt. ἐξόλλυμι, \textit{may he perish}}
}

\glossline{87}{%
 \glossmix{Περὶ ἐμοῦ δ' οὐδεὶς λόγος}{Note change of subject: Xanthias can no longer restrain himself. \textit{but about me there's no consideration}}
}

\glossline{88}{%
 \glossmix{ἐπιτριβομένου}{with ἐμοῦ}\\
 \glossmix{τὸν ὦμον}{acc. respect, Smyth 1601a}\\
 \glossmix{οὑτωσὶ}{Deictic iota on οὕτως}
}

\glossline{90}{%
 \glossmix{τὸν ὦμον}{acc. respect, Smyth 1601a}
}

\glossline{91}{%
 \glossmix{Εὐριπίδου πλεῖν ἢ σταδίῳ λαλίστερα;}{\textit{'And miles verboser than Euripides'} (Murray). On πλεῖν ἢ in this line and the previous line see note on l. 18. Εὐριπίδου genitive of comparison with λαλίστερα, σταδίῳ dat. degree of difference. Essentially two comparative ideas: \textit{1) chattier than Euripides 2) more than by a stade.}}
}

\glossline{93}{%
 \glossmix{χελιδόνων μουσεῖα}{\textit{performance halls of swallows}, a parody of Euripides' \textit{Alcmene}: πολὺς δ' ἀνεῖρπε κισσὸς, εὐφυὴς κλάδος, / χελιδόνων μουσεῖον. Greeks frequently compared non-Greek language to the sound of swallows (Σ: ἀντὶ τοῦ βάρβαρα καὶ ἀσύνετα)}
}

\glossline{94}{%
 \glossmix{}{Supply ἐστιν. \textit{who [are] gone in a flash (lit. very quickly), if they get a single chorus, after pissing once and for all on tragedy.} To 'get a chorus' is to be granted a performance slot in the festival competition.}\\
  \glossmix{ἢν}{= ἐὰν}\\
  \glossmix{λάβῃ}{3s aor. subj. λαμβάνω}
}

\glossline{96}{%
 \glossmix{ἂν... ἄν}{Redundant.}
}

\glossline{97}{%
 \glossmix{ζητῶν}{< ζητέω to seek, not ζῶ to live}\\
 \glossmix{λάκοι}{Aor. opt. λάσκω.}\\
 \glossmix{ὅστις ῥῆμα γενναῖον λάκοι}{\textit{who could utter a noble expression}. "The optative without ἄν (probably potential) occurs in Attic poetry after οὐκ ἔστιν ὅστις..." (Smyth 2552)}
}

\glossline{98}{%
 \glossmix{φθέγξεται}{Fut. ind. φθέγγομαι}
}

\glossline{99}{%
 \glossmix{τοιουτονί τι παρακεκινδυνευμένον,}{\textit{a risky kind of (expression) like this}}
}

\glossline{100}{%
 \glossmix{Διὸς δωμάτιον}{in apposition to αἰθέρα.}
 \glossmix{}{The scholia claim these phrases are parodies of E.'s \textit{Melanippe} and \textit{Alexander} ("αἰθέρα Διὸς": Εὐριπίδου ἐκ Μελανίππης "ὄμνυμι δ' ἱρὸν αἰθέρ', οἴκησιν Διός." τὸ δὲ "χρόνου πόδα" ἐστὶν ἐξ Ἀλεξάνδρας "καὶ χρόνου προὔβαινε πούς."). Strangely, however, χρόνου πόδα occurs in that exact form in addition at \textit{Bacchae} 889 -- strangely, since that tragedy was performed either simultaneously with or after the \textit{Frogs}.}
}

\end{parcolumns}
\end{multicols}

\newpage

\begin{spacing}{1.5}

\begin{tabularx}{\textwidth}{@{}lXr@{}}
  & ἢ “φρένα μὲν οὐκ ἐθέλουσαν ὀμόσαι καθ' ἱερῶν, & \\
  & γλῶτταν δ' ἐπιορκήσασαν ἰδίᾳ τῆς φρενός.” & \\
  \textit{Ἡρ.} & Σὲ δὲ ταῦτ' ἀρέσκει; & \\
  \textit{Δι.} & \hspace*{8.5em}Μἀλλὰ πλεῖν ἢ μαίνομαι. & \\
  \textit{Ἡρ.} & Ἦ μὴν κόβαλά γ' ἐστίν, ὡς καὶ σοὶ δοκεῖ. & \\
  \textit{Δι.} & Μὴ τὸν ἐμὸν οἴκει νοῦν· ἔχεις γὰρ οἰκίαν. & 105 \\
  \textit{Ἡρ.} & Καὶ μὴν ἀτεχνῶς γε παμπόνηρα φαίνεται. & \\
  \textit{Δι.} & Δειπνεῖν με δίδασκε. & \\
  \textit{Ξα.} & \hspace*{8.5em}Περὶ ἐμοῦ δ' οὐδεὶς λόγος. & \\
  \textit{Δι.} & Ἀλλ' ὧνπερ ἕνεκα τήνδε τὴν σκευὴν ἔχων & \\
  & ἦλθον κατὰ σὴν μίμησιν, ἵνα μοι τοὺς ξένους & \\
  & τοὺς σοὺς φράσειας, εἰ δεοίμην, οἷσι σὺ & 110 \\
  & ἐχρῶ τόθ', ἡνίκ' ἦλθες ἐπὶ τὸν Κέρβερον, & \\
  & τούτους φράσον μοι, λιμένας, ἀρτοπώλια, & \\
  & πορνεῖ', ἀναπαύλας, ἐκτροπάς, κρήνας, ὁδούς, & \\
  & πόλεις, διαίτας, πανδοκευτρίας, ὅπου & \\
  & κόρεις ὀλίγιστοι. & \\
  \textit{Ξα.} & \hspace*{7.5em}Περὶ ἐμοῦ δ' οὐδεὶς λόγος. & 115 \\

\end{tabularx}

\end{spacing}

\newpage

\begin{multicols}{2}
\small % roughly 9pt
\vocabentry{ἀνά-παυλα, ἡ}{repose, rest; resting-place, inn}
\vocabentry{ἀρέσκω}{to please}
\vocabentry{ἀρτο-πώλιον, τό}{a baker's shop, bakery}
\vocabentry{ἀ-τεχνῶς}{(adv.) simply}
\vocabentry{δέω}{to lack, miss, stand in need of}
\vocabentry{δειπνέω}{to eat dinner < δεῖπνον dinner}
\vocabentry{δίαιτα, ἡ}{a way of living, mode of life; dwelling, abode; room}
\vocabentry{δίδασκω}{to teach}
\vocabentry{ἐκ-τροπή, ἡ}{a turning off; fork, branch (in road)}
\vocabentry{ἐπι-ορκέω}{to swear falsely, forswear oneself}
\vocabentry{ἕνεκα}{(+ gen.) on account of, for the sake of, because of, for; typically follows its noun}
\vocabentry{ἰδίᾳ}{(adv.) (+ gen.) by oneself, privately (from), separately (from) < ἴδιος one's own}
\vocabentry{ἱερά, τά}{sacrifices, offerings, victims (< ἱερός holy)}
\vocabentry{κατά}{(+ gen.) down from; (LSJ A.II.4) (of vows or oaths) by}
\vocabentry{Κέρβερος, ὁ}{Cerberus, the guard-dog of Hades}
\vocabentry{κόβαλα, τά}{dirty tricks, rogueries}
\vocabentry{κόρις, ιος, ὀ}{bedbug}
\vocabentry{κρήνη, ἡ}{a well, spring, fountain}
\vocabentry{λιμήν, ένος, ὁ}{harbor; (metaphor.) haven, refuge}
\vocabentry{μίμησις, ἡ}{imitation}
\vocabentry{ξένος, ὁ}{guest-friend; guest; host}
\vocabentry{οἰκέω}{to inhabit; to manage, direct}
\vocabentry{ὀλίγιστος, η, ον}{fewest, smallest (LSJ: "always of Number or Quantity"), superlative ὀλίγος}
\vocabentry{ὀλίγος, η, ον}{small; (of number) few; superlative }
\vocabentry{ὄμνυμι}{to swear}
\vocabentry{ὅπου}{where (relative pronoun)}
\vocabentry{παμ-πόνηρος, ος, ον}{thoroughly depraved, very bad}
\vocabentry{παν-δοκεύτρια, ἡ}{a hostess}
\vocabentry{πορνεῖον, τό}{brothel}
\vocabentry{σκευή, ἡ}{equipment, dress, costume}
\vocabentry{χράομαι}{to use (+ dat.)}
\vocabentry{φράζω}{to point out, show; to declare, explain}
\end{multicols}

\vspace{-1.5em}
\noindent\rule{\linewidth}{0.4pt}
\vspace{-2em}

\begin{multicols}{2}
\begin{parcolumns}[colwidths={1=1.5em, 2=0.9\linewidth}]{2}
\small

\glossline{101}{%
 \glossmix{}{cf. Eur. \textit{Hipp}. 612 ἡ γλῶσσ' ὀμώμοχ', ἡ δὲ φρὴν ἀνώμοτος. ("The tongue swore, but the mind is unsworn.)}\\
 \glossmix{φρένα...}{Sentence fragment with nouns in the accusative (like the previous line). \textit{a mind unwilling to swear by sacrifices, and a tongue which swore falsely separately from the mind}}\\
 \glossmix{ἐθέλουσαν}{with φρένα}\\
 \glossmix{ὀμόσαι}{aor. inf. ὄμνυμι}\\
 \glossmix{καθ'}{κατά. See vocab for meaning here.}
}

\glossline{103}{%
 \glossmix{Μἀλλὰ...}{μή, ἀλλά. \textit{Don't (say that), but I'm more than crazy (about them)}. Appropriate for Dionysus. On πλεῖν ἢ see l. 18}
}

\glossline{104}{%
 \glossmix{Ἦ μὴν}{"ἦ μήν introduces a strong and confident asseveration" (GP 350). \textit{I'm telling you, absolutely}}\\
 \glossmix{ὡς καὶ σοὶ δοκεῖ.}{\textit{as it seems to you too}, more loosely \textit{as you know yourself}. i.e. Heracles thinks that Dionysus actually agrees with him that these tragic lines are dirty tricks.}
}

\glossline{105}{%
 \glossmix{οἴκει}{pres. impv. οἰκέω}\\
 \glossmix{Μὴ τὸν ἐμὸν οἴκει νοῦν}{According to the scholia this phrase is from the \textit{Andromache}, but the extant \textit{Andromache} does not include this phrase. Possibly the scholia's reference is an error for the lost \textit{Andromeda}.}\\
 \glossmix{ἔχεις γὰρ οἰκίαν.}{\textit{For you have a house.} Dionysus' joke rests on the literal meaning of οἰκέω.}
}

\glossline{106}{%
 \glossmix{Καὶ μὴν}{\textit{Yes, but}. Adversative, GP 357-8.}
}

\glossline{107}{%
 \glossmix{Δειπνεῖν με δίδασκε.}{i.e. stick to your strengths (soup) and don't try to talk to me about poetry.}
}

\glossline{108}{%
 \glossmix{Ἀλλ' ὧνπερ ἕνεκα... ἵνα... φράσειας...}{\textit{Now, as to why I came... [it was] in order that... you show...} ὧνπερ ἕνεκα, lit. \textit{because of which things}.}
}

\glossline{109}{%
 \glossmix{κατὰ σὴν μίμησιν,}{\textit{according to your imitation}, i.e. \textit{in imitation of you}}\\
 \glossmix{ξένους}{i.e. the hosts that helped Heracles on his trip to the underworld.}
}

\glossline{110}{%
 \glossmix{φράσειας}{2s aor. opt. φράζω}\\
 \glossmix{εἰ δεοίμην}{\textit{if I should need [them]}. FLV protasis}\\
 \glossmix{οἷσι}{relative pronoun after τοὺς ξένους, \textit{whom}}
}

\glossline{111}{%
 \glossmix{ἐχρῶ}{impf. 2s χράομαι, \textit{whom you used}}\\
 \glossmix{ἐπὶ}{cf. 69}
}

\glossline{112}{%
 \glossmix{φράσον}{aor. impv. φράζω}
}

\end{parcolumns}
\end{multicols}


\newpage

\begin{spacing}{1.5}

\begin{tabularx}{\textwidth}{@{}lXr@{}}
  \textit{Ἡρ.} & Ὦ σχέτλιε, τολμήσεις γὰρ ἰέναι; & \\
  \textit{Δι.} & \hspace*{13.5em}Καὶ σύ γε & \\
  & μηδὲν ἔτι πρὸς ταῦτ', ἀλλὰ φράζε τῶν ὁδῶν & \\
  & ὅπῃ τάχιστ' ἀφιξόμεθ' εἰς Ἅιδου κάτω· & \\
  & καὶ μήτε θερμὴν μήτ' ἄγαν ψυχρὰν φράσῃς. & \\
  \textit{Ἡρ.} & Φέρε δή, τίν' αὐτῶν σοι φράσω πρώτην; τίνα; & 120 \\
  & Μία μὲν γὰρ ἔστιν ἀπὸ κάλω καὶ θρανίου, & \\
  & κρεμάσαντι σαυτόν. & \\
  \textit{Δι.} & \hspace*{8.5em}Παῦε, πνιγηρὰν λέγεις. & \\
  \textit{Ἡρ.} & Ἀλλ' ἔστιν ἀτραπὸς ξύντομος τετριμμένη, & \\
  & ἡ διὰ θυείας. & \\
  \textit{Δι.} & \hspace*{5.5em}Ἆρα κώνειον λέγεις; & \\
  \textit{Ἡρ.} & Μάλιστά γε. & \\
  \textit{Δι.} & \hspace*{5.5em}Ψυχράν γε καὶ δυσχείμερον· & 125 \\
  & εὐθὺς γὰρ ἀποπήγνυσι τἀντικνήμια. & \\
  \textit{Ἡρ.} & Βούλει ταχεῖαν καὶ κατάντη σοι φράσω; & \\
  \textit{Δι.} & Νὴ τὸν Δί', ὡς ὄντος γε μὴ βαδιστικοῦ. & \\
  \textit{Ἡρ.} & Καθέρπυσόν νυν εἰς Κεραμεικόν. & \\
  \textit{Δι.} & \hspace*{14em}Κᾆτα τί; & \\
  \textit{Ἡρ.} & Ἀναβὰς ἐπὶ τὸν πύργον τὸν ὑψηλόν –   & \\
  \textit{Δι.} & \hspace*{16em}Τί δρῶ; & 130 \\

\end{tabularx}

\end{spacing}

\newpage

\begin{multicols}{2}
\small % roughly 9pt
\vocabentry{ἄγαν}{(adv.) very, much, very much}
\vocabentry{ἀνα-βαίνω}{to go up, mount, to go up to}
\vocabentry{ἀντι-κνήμιον, τό}{the part of the leg in front of the κνήμη, shin}
\vocabentry{ἀπο-πήγνυμι}{to make to freeze, freeze}
\vocabentry{ἀτραπός, ἡ}{short cut}
\vocabentry{βαδιστικός, ά, όν}{good at walking}
\vocabentry{δράω}{to do}
\vocabentry{δυσ-χείμερος, ον}{wintry, stormy}
\vocabentry{θερμός, ά, όν}{hot, warm}
\vocabentry{θρανίον, τό}{diminutive of θρᾶνος}
\vocabentry{θρᾶνος, ὁ}{bench; wooden beam}
\vocabentry{θυεία, ἡ}{a mortar}
\vocabentry{κάλως, ὁ (gen. κάλω)}{reefing rope (i.e. a rope holding a sail)}
\vocabentry{καθ-έρπω}{to creep down, to steal}
\vocabentry{κατ-άντης, ές}{downhill, downward, steep}
\vocabentry{Κεραμεικός, ὁ}{the Potters’ Quarter in Athens}
\vocabentry{κρεμάννυμι}{to hang, hang up}
\vocabentry{κώνειον, τό}{hemlock}
\vocabentry{μάλιστα}{(adv.) most}
\vocabentry{ὅπη}{(adv.) by which way}
\vocabentry{πνιγηρός, ός, ά, όν}{choking, stifling (< πνίγω)}
\vocabentry{πύργος, ὁ}{a tower}
\vocabentry{σύν-τομος, ον (Att. ξύν-)}{cut short, brief}
\vocabentry{σχέτλιος, α, ον}{unflinching; headstrong; miserable, wretched}
\vocabentry{τολμάω}{to undertake, take heart}
\vocabentry{τρίβω}{to rub; to wear out; to use constantly}
\vocabentry{ὑψηλός, ά, όν}{high, lofty, high-raised}
\vocabentry{ψυχρός, ά, όν}{cold, chill}
\end{multicols}

\vspace{-1.5em}
\noindent\rule{\linewidth}{0.4pt}
\vspace{-2em}

\begin{multicols}{2}
\begin{parcolumns}[colwidths={1=1.5em, 2=0.9\linewidth}]{2}
\small

\glossline{116}{%
 \glossmix{ὦ σχέτλιε}{The vocative often criticizes someone for stubbornness or ruthlessness, but here is probably sympathetic, \textit{you poor idiot}. Cf. Isemene's ὦ σχετλία to Antigone "reproaching her rashness" (Dover)}\\
 \glossmix{τολμήσεις}{Fut. τολμάω}\\
 \glossmix{γὰρ}{"Γάρ gives the motive for saying that which has just been said: 'I say this because...'" (GP 60), including justifying vocatives.}
}

\glossline{117}{%
 \glossmix{μηδὲν ἔτι πρὸς ταῦτ'}{Sc. λέγε or similar.}\\
 \glossmix{τῶν ὁδῶν}{Take as partitive genitive with ὅπῃ. \textit{of the ways, by which way}. Dionysus continues the joke that a trip to Hades is like any trip, and asks for the best route of the many options available. In addition, at least five locations were said to be the place where Heracles had descended to retrieve Cerberus (Friese 2018: 218n13).}
}

\glossline{118}{%
 \glossmix{ἀφιξόμεθ'}{Fut. ἀφικνέομαι}\\
 \glossmix{εἰς Ἅιδου}{cf. note on l. 69}
}

\glossline{119}{%
 \glossmix{ἄγαν ψυχρὰν}{\textit{nor a very cold [route, sc. ὁδόν]}. Possibly an example of an "apo koinou" construction, where the adverb ἄγαν goes with both θερμὴν and ψυχρὰν.}
 \glossmix{φράσῃς}{Prohib. subj.}
}

\glossline{120}{%
 \glossmix{Φέρε δή}{\textit{come on}, LSJ ΙΧ.4}\\
 \glossmix{φράσω}{Deliberative subj.}
}

\glossline{121}{%
 \glossmix{Μία μὲν γὰρ}{Μία [ὁδός] μὲν γὰρ [σοι] ἔστιν..., dat. of poss. Heracles seems at first to suggest a sailing trip, but then the joke lies in the use of the rope and bench for suicide.}\\
 \glossmix{ἀπὸ}{LSJ III.3, "of the instrument \textit{from} or \textit{by} which a thing is done," i.e. \textit{by means of}}
}

\glossline{122}{%
 \glossmix{κρεμάσαντι}{Aor. pple. κρεμάννυμι.}\\
 \glossmix{πνιγηρὰν}{Sc. ὁδόν}
}

\glossline{123}{%
 \glossmix{τετριμμένη}{Pf. pass. pple. τρίβω, \textit{well-worn}}
}

\glossline{125}{%
 \glossmix{δυσχείμερον}{Feminine. Most compound adjectives use masculine endings for the feminine, Smyth 288.}
}

\glossline{126}{%
 \glossmix{ἀποπήγνυσι}{3s pres. act. ind., like δίδωσι.}\\
 \glossmix{τἀντικνήμια}{= τὰ ἀντικνήμια. Dionsyus plays on a journey in the cold with exposed and freezing shins. Furthermore, hemlock was thought to numb the body from the feet up (cf. \textit{Phaedo} 117e).}
}

\glossline{127}{%
 \glossmix{Βούλει}{2s pres. act. βούλομαι. βούλει + subj. is an Attic idiom for "do you want that..." (Smyth 1806, LSJ s.v. βούλομαι II.1)}\\
 \glossmix{κατάντη}{fem. acc. sg. κατάντης}
}

\glossline{128}{%
 \glossmix{ὡς}{ὡς + pple. expressing the state of mind / intent of the agent, \textit{on the grounds that...} (Smyth 2086d)}\\
 \glossmix{ὄντος γε μὴ βαδιστικοῦ}{Sc. μου as the subject of the genitive absolute.}
}

\glossline{129}{%
 \glossmix{Καθέρπυσόν}{2s aor. impv. καθέρπω. Τhe κατα prefix strengthens the simple verb or suggests a trip down to the underworld.}\\
 \glossmix{Κᾆτα}{= Καὶ εἶτα}\\
 \glossmix{Κεραμεικόν}{The potter's district immediately outside the Dipylon gate, a significant passage through Athens' walls.}
}

\glossline{130}{%
 \glossmix{Ἀναβὰς}{Aor. pple. ἀναβαίνω.}\\
 \glossmix{πύργον}{Perhaps the "tower of Timon" in the Academy near the Kerameikos (Paus. 1.30.4).}\\
 \glossmix{δρῶ}{Deliberative subj. δράω.}
}

\end{parcolumns}
\end{multicols}

\newpage

\begin{spacing}{1.5}

\begin{tabularx}{\textwidth}{@{}lXr@{}}
  \textit{Ἡρ.} & ἀφιεμένην τὴν λαμπάδʼ ἐντεῦθεν θεῶ, & \\
  & κἄπειτʼ ἐπειδὰν φῶσιν οἱ θεώμενοι & \\
  & "εἷναι," τόθʼ εἷναι καὶ σὺ σαυτόν. & \\
  \textit{Δι.} & \hspace*{14em}ποῖ & \\
  \textit{Ἡρ.} & \hspace*{16em}κάτω. & \\
  \textit{Δι.} & ἀλλʼ ἀπολέσαιμʼ ἂν ἐγκεφάλου θρίω δύο. & \\
  & οὐκ ἂν βαδίσαιμι τὴν ὁδὸν ταύτην. & 135 \\
  \textit{Ἡρ.} & \hspace*{15em}τί δαί; & \\
  \textit{Δι.} & ἥνπερ σὺ τότε κατῆλθες. & \\
  \textit{Ἡρ.} &\hspace*{11em}ἀλλʼ ὁ πλοῦς πολύς.  & \\
  & εὐθὺς γὰρ ἐπὶ λίμνην μεγάλην ἥξεις πάνυ & \\
  & ἄβυσσον. & \\
  \textit{Δι.} & \hspace*{5.5em}εἶτα πῶς περαιωθήσομαι; & \\
  \textit{Ἡρ.} & ἐν πλοιαρίῳ τυννουτῳί σʼ ἀνὴρ γέρων & \\
  & ναύτης διάξει δύʼ ὀβολὼ μισθὸν λαβών. & 140 \\
  \textit{Δι.} & φεῦ,  & 141a \\
  & ὡς μέγα δύνασθον πανταχοῦ τὼ δύʼ ὀβολώ. & 141b \\
  & πῶς ἠλθέτην κἀκεῖσε; & \\
  \textit{Ἡρ.} & \hspace*{10em}Θησεὺς ἤγαγεν.& \\
  & μετὰ ταῦτʼ ὄφεις καὶ θηρίʼ ὄψει μυρία &  \\
  & δεινότατα. & \\
  

\end{tabularx}

\end{spacing}



\newpage

\begin{multicols}{2}
\small % roughly 9pt
\vocabentry{ἄβυσσος, ος, ον}{with no bottom, bottomless, unfathomed}
\vocabentry{ἀφίημι}{to send forth, discharge}
\vocabentry{γέρων, οντος, ὁ}{old man; as adj., old}
\vocabentry{διάγω}{to carry over}
\vocabentry{ἐγκέφαλος, ὁ}{that which is within the head, the brain}
\vocabentry{ἐκεῖσε}{(adv.) thither, to that place}
\vocabentry{ἐντεῦθεν}{(adv.) hence, thence}
\vocabentry{ἐπειδάν}{when, whenever}
\vocabentry{ἥκω}{to have come, be present}
\vocabentry{θηρίον, τό}{a wild animal, beast}
\vocabentry{Θησεύς, ὁ}{Theseus, the mythological founder of Athens}
\vocabentry{θρῖον, τό}{fig-leaf, stuffed fig-leaf}
\vocabentry{λαμπάς, ἡ}{a torch, torch-race beacon}
\vocabentry{μισθός, ὁ}{wages, pay, hire}
\vocabentry{μυρίος, ός, ά, όν}{numberless, countless, infinite}
\vocabentry{ναύτης, ὁ}{a sailor}
\vocabentry{ὄφις, ὁ}{a serpent, snake}
\vocabentry{πανταχοῦ}{(adv.) everywhere}
\vocabentry{πάνυ}{altogether; (with adj.) very, exceedingly}
\vocabentry{περαιόω}{to carry to the opposite side, carry over}
\vocabentry{πλόος, ὁ}{a sailing, voyage}
\vocabentry{πλοιάριον, τό}{a skiff, boat (dim. πλοῖον)}
\vocabentry{τυννοῦτος, ον/ο}{so small, so little}
\vocabentry{φεῦ}{ah! alas! woe!}
\end{multicols}

\vspace{-1.5em}
\noindent\rule{\linewidth}{0.4pt}
\vspace{-2em}

\begin{multicols}{2}
\begin{parcolumns}[colwidths={1=1.5em, 2=0.9\linewidth}]{2}
\small

\glossline{131}{%
 \glossmix{ἀφιεμένην}{Pres. mid/pass. pple. ἀφίημι}
 \glossmix{ἀφιεμένην τὴν λαμπάδ’}{\textit{the race begun.} λαμπάς is used by metonymy to refer to a torch-race (relay-race) conducted in the course of the procession of the Panathenaia or other festivals. According to Pausanias (1.30.2), one such race began in the Academy and went to the city, and presumably passed through the Kerameikos en route.}\\
 \glossmix{θεῶ}{present middle imperative of θεάομαι.}
 }

 \glossline{132}{%
 \glossmix{φῶσιν}{3p subj. act. φημί}
}

\glossline{133}{%
 \glossmix{εἷναι}{Aor. inf. ἵημι. Either the infinitive used as an imperative (\textit{go!}, or the infinitive in indirect speech with an implied subject of the racers (\textit{they're off!}). The second εἷναι must be the infinitive as imperative.}
}

\glossline{134}{%
 \glossmix{ἀπολέσαιμʼ}{ἀπολέσαιμι, 1s aor. opt. ἀπόλλυμι}\\
 \glossmix{ἐγκεφάλου θρίω δύο}{Stuffed fig-leaves are compared to the hemispheres of the brain, either by a similarity of shape, or because animal brains were roasted in fig-leaves (according to a scholiast). θρίω is the dual accusative.}
}

\glossline{135}{%
 \glossmix{τί δαί}{\textit{What, then?} δαί always follows an interrogative, either emphasizing or connecting to the previous thought. Perhaps here connective, "especially... after the rejection of an idea: 'Well, what ...'" (GP 262-3).}
}

\glossline{136}{%
 \glossmix{ἥνπερ σὺ τότε κατῆλθες.}{As one of his labors, Heracles descended (κατ-) to the Underworld to capture its guard-dog Cerberus.}
}

\glossline{137}{%
 \glossmix{ἥξεις}{Fut. ἥκω.}
}

\glossline{138}{%
 \glossmix{περαιωθήσομαι}{1s fut. pass. περαιόω.}
}

\glossline{139}{%
\glossmix{τυννουτῳί}{Perhaps accompanied with a gesture of size, "this big."}\\
 \glossmix{ἀνὴρ γέρων/ ναύτης}{Charon, the boatman who carries souls across the river Styx deeper into the Underworld.}
}

\glossline{140}{%
 \glossmix{ὀβολὼ}{accusative dual, object of λαβών. δύʼ ὀβολὼ in the dual 5x in Aristophanes, later taken as a feature of Attic style (cf. Lucian \textit{Lexiphanes} 2).}\\
 \glossmix{μισθὸν}{in apposition to δύ’ ὀβολὼ.}
}

\glossline{141a}{%
 \glossmix{φεῦ}{Extrametrical, i.e. 141b scans as a full line on its own.}
}

\glossline{141b}{%
 \glossmix{δύνασθον}{pres. mid. 3rd person dual. Contemporary Attic inscriptions refer to a regular payment of two obols (διωβελία), possibly to support people impoverished by the war (Dover).}
}

\glossline{142}{%
 \glossmix{ἠλθέτην}{aor. act. 3rd person dual.}\\
 \glossmix{Θησεὺς}{The hero Theseus accompanied his friend Pirithous into the Underworld in an attempt to abduct Persephone. He failed and was trapped in the Underworld until Heracles retrieved him on his way out with Cerberus. Plutarch also claims he minted coins (\textit{Thes.} 25.3}
}

\glossline{143}{%
 \glossmix{ὄψει}{2s fut. ὁράω (ὄψομαι).}
}


\end{parcolumns}
\end{multicols}

\newpage

\begin{spacing}{1.5}

\begin{tabularx}{\textwidth}{@{}lXr@{}}
  \textit{Δι.} & \hspace*{6em}Μή μ' ἔκπληττε μηδὲ δειμάτου· & \\
  & οὐ γάρ μ' ἀποτρέψεις. & \\
  \textit{Ἡρ.} & \hspace*{9.5em}Εἶτα βόρβορον πολὺν & 145 \\
  & καὶ σκῶρ ἀείνων· ἐν δὲ τούτῳ κειμένους, & \\
  & εἴ που ξένον τις ἠδίκησε πώποτε, & \\
  & ἢ παῖδα κινῶν τἀργύριον ὑφείλετο, & \\
  & ἢ μητέρ' ἠλόησεν, ἢ πατρὸς γνάθον & \\
  & ἐπάταξεν, ἢ 'πίορκον ὅρκον ὤμοσεν, & 150 \\
  & ἢ Μορσίμου τις ῥῆσιν ἐξεγράψατο. & \\
  \textit{Δι.} & Νὴ τοὺς θεοὺς ἐχρῆν γε πρὸς τούτοισι κεἰ & \\
  & τὴν πυρρίχην τις ἔμαθε τὴν Κινησίου. & \\
  \textit{Ἡρ.} & Ἐντεῦθεν αὐλῶν τίς σε περίεισιν πνοή, & \\
  & ὄψει τε φῶς κάλλιστον ὥσπερ ἐνθάδε, & 155 \\
  & καὶ μυρρινῶνας καὶ θιάσους εὐδαίμονας & \\
  & ἀνδρῶν γυναικῶν καὶ κρότον χειρῶν πολύν. & \\
  \textit{Δι.} & Οὗτοι δὲ δὴ τίνες εἰσίν; & \\
  \textit{Ἡρ.} & \hspace*{10em}Οἱ μεμυημένοι –  & \\
  \textit{Χα.} & Νὴ τὸν Δί' ἐγὼ γοῦν ὄνος ἄγω μυστήρια. & \\
  & Ἀτὰρ οὐ καθέξω ταῦτα τὸν πλείω χρόνον. & 160 \\

\end{tabularx}

\end{spacing}

\newpage

\begin{multicols}{2}
\small % roughly 9pt
\vocabentry{ἀ-δικέω}{to do wrong}
\vocabentry{ἀέναος, ον}{ever-flowing}
\vocabentry{ἀλοάω}{to tread, thresh; to cudgel, thrash}
\vocabentry{ἀπο-τρέπω}{to turn away from, dissuade}
\vocabentry{ἀργύριον, τό}{small coin; money; silver}
\vocabentry{αὐλός, ὁ}{aulos (wind instrument resembling oboe)}
\vocabentry{βόρβορος, ὁ}{mud, mire}
\vocabentry{γνάθος, ἡ}{the jaw}
\vocabentry{δειματόω}{to frighten (< δεῖμα fear)}
\vocabentry{ἐκ-γράφω}{to write out; (mid.) to copy for oneself}
\vocabentry{ἐκ-πλήσσω (Att. ἐκπλήττω)}{to strike out of, drive away from; to astound, shock, amaze}
\vocabentry{ἐντεῦθεν}{(adv.) hence, thence; henceforth, thereupon, i.e. then, next}
\vocabentry{ἐπί-ορκος, ον}{sworn falsely, perjured}
\vocabentry{εὐ-δαίμων, ον}{fortunate, wealthy, happy}
\vocabentry{θίασος, ὁ}{a group marching through the streets with dance and song, esp. in honor of Bacchus; a band of revelers}
\vocabentry{κατ-έχω}{to hold fast; to check, restrain}
\vocabentry{κινέω}{to set in motion, move; (LSJ II.4) to have sex with, screw, fuck}
\vocabentry{Κινησίας, ὁ}{Cinesias, a dithyrambic poet often mocked by Ar.}
\vocabentry{κρότος, ὁ}{a striking, the sound made by striking; clapping}
\vocabentry{μήτηρ, ερος, ἡ}{a mother}
\vocabentry{Μόρσιμος, ὁ}{Morsimus, a tragedian elsewhere mocked by Ar.}
\vocabentry{μυέω}{to initiate (into the mysteries)}
\vocabentry{μυρσινών (Att. μυρρινών), ῶνος, ὁ}{myrtle-grove}
\vocabentry{μυστήριον, τό}{a mystery}
\vocabentry{ὄνος, ὁ/ἡ}{an ass}
\vocabentry{ὅρκος, ὁ}{oath}
\vocabentry{περί-ειμι}{go around}
\vocabentry{πνοή, ἡ}{a blowing, blast, breeze; (LSJ IV) breath (of a wind instrument)}
\vocabentry{πυρρίχη, ἡ}{the pyrrhic dance, a kind of war-dance}
\vocabentry{πώποτε}{(adv.) ever yet}
\vocabentry{ῥῆσις, ἡ}{a saying, speaking, speech}
\vocabentry{σκῶρ, σκάτος, τό}{dung (cogn. dial. Eng. sharn)}
\vocabentry{ὑφ-αιρέω}{(lit. to seize underneath), take away, steal, take gradually, deduct}
\vocabentry{φάος, εος, τό}{light, daylight}
\end{multicols}

\vspace{-1.5em}
\noindent\rule{\linewidth}{0.4pt}
\vspace{-2em}

\begin{multicols}{2}
\begin{parcolumns}[colwidths={1=1.5em, 2=0.9\linewidth}]{2}
\small

\glossline{143}{%
 \glossmix{δειμάτου}{pres. act. impv. δειματόω (< *δειμάτοε)}
}

\glossline{145}{%
 \glossmix{Εἶτα βόρβορον πολὺν}{Continues the ὄψει from l. 143.}
}

\glossline{146}{%
 \glossmix{ἀείνων}{Neut. acc. sg. ἀέναος. Contracted Attic form}\\
 \glossmix{κειμένους}{masc. acc. pl., \textit{[those people] lying}}
}

\glossline{148}{%
 \glossmix{παῖδα}{\textit{slave}, gender ambiguous. In Athens slaves often worked as prostitutes (Cohen 2015: 46-8).}\\
 \glossmix{ἀργύριον ὑφείλετο}{i.e. secretly took back what he paid the prostitute.}
}

\glossline{150}{%
 \glossmix{ἐπάταξεν}{Aor πατάσσω.}\\
 \glossmix{'πίορκον}{= ἐπίορκον. Etymological play.}\\
 \glossmix{ὤμοσεν}{aor. ὄμνυμι}
}

\glossline{151}{%
 \glossmix{Νὴ τοὺς θεοὺς ἐχρῆν γε πρὸς τούτοισι}{i.e. \textit{yes, by gods, it [i.e. punishment in Hades] \textbf{was} necessary, in addition to those [sins], if also someone...}}
}

\glossline{152}{%
 \glossmix{κεἰ}{καὶ εἰ}
}

\glossline{154}{%
 \glossmix{τίς}{indefinite τις, accent from enclitic σε}\\
 \glossmix{περίεισιν}{Fut.}
}

\glossline{155}{%
 \glossmix{ὥσπερ ἐνθάδε}{\textit{just like here}. But where is here? The land of the living? The spectators' here, i.e. in Attica?}
}

\glossline{157}{%
 \glossmix{ἀνδρῶν γυναικῶν}{asyndeton}
}

\glossline{159}{%
 \glossmix{Δι'}{cf. l. 28}\\
 \glossmix{ἐγὼ γοῦν ὄνος ἄγω μυστήρια}{Difficult. Perhaps \textit{I'm the donkey celebrating the mysteries}, calling back to Xanthias' earlier complaints about carrying burdens without reward}
}

\glossline{160}{%
 \glossmix{ταῦτα}{i.e. the baggage}\\
 \glossmix{πλείω}{masc. acc. sg. πλείων \textit{more}}\\
 \glossmix{τὸν πλείω χρόνον}{\textit{for any longer time} (Stanford)} 
}

\end{parcolumns}
\end{multicols}

\newpage

\begin{spacing}{1.5}

\begin{tabularx}{\textwidth}{@{}lXr@{}}
  \textit{Ἡρ.} & οἵ σοι φράσουσ' ἁπαξάπανθ' ὧν ἂν δέῃ. & \\
  & Οὗτοι γὰρ ἐγγύτατα παρ' αὐτὴν τὴν ὁδὸν & \\
  & ἐπὶ ταῖσι τοῦ Πλούτωνος οἰκοῦσιν θύραις. & \\
  & Καὶ χαῖρε πόλλ', ὦδελφέ. & \\
  \textit{Δι.} & \hspace*{10.5em}Νὴ Δία καὶ σύ γε & \\
  & ὑγίαινε. Σὺ δὲ τὰ στρώματ' αὖθις λάμβανε. & 165 \\
  \textit{Ξα.} & Πρὶν καὶ καταθέσθαι; & \\
  \textit{Δι.} & \hspace*{9.5em}Καὶ ταχέως μέντοι πάνυ. & \\
  \textit{Ξα.} & Μὴ δῆθ', ἱκετεύω σ', ἀλλὰ μίσθωσαί τινα & \\
  & τῶν ἐκφερομένων, ὅστις ἐπὶ τοῦτ' ἔρχεται. & \\
  \textit{Δι.} & Ἐὰν δὲ μηὕρω; & \\
  \textit{Ξα.} & \hspace*{6.5em}Τότ' ἔμ' ἄγειν. & \\
  \textit{Δι.} & \hspace*{12.5em}Καλῶς λέγεις. & \\
  & Καὶ γάρ τιν' ἐκφέρουσι τουτονὶ νεκρόν. & 170 \\
  & Οὗτος, σὲ λέγω μέντοι, σὲ τὸν τεθνηκότα. & \\
  & Ἄνθρωπε, βούλει σκευάρι' εἰς Ἅιδου φέρειν; & \\
  \end{tabularx}

  \noindent\textit{\MakeUppercase{\textls[200]{Νεκρός}}}

\begin{tabularx}{\textwidth}{@{}lXr@{}}
  \phantom{Ξα.} & Πόσ' ἄττα; & \\
  \textit{Δι.} & \hspace*{5em}Ταυτί. & \\
  \textit{Νε.} & \hspace*{8em}Δύο δραχμὰς μισθὸν τελεῖς; & \\
\end{tabularx}

\end{spacing}

\newpage

\begin{multicols}{2}
\small % roughly 9pt
\vocabentry{ἀδελφός, ὁ}{brother}
\vocabentry{ἁπαξάπας, ασα, αν}{all together, the whole; (pl.) all at once, all together}
\vocabentry{δέομαι}{to lack, need}
\vocabentry{δραχμή, ἡ}{a handful; a drachma}
\vocabentry{ἐγγύτατα}{(adv.) nearest, very near (< ἐγγύς near)}
\vocabentry{ἐκ-φέρω}{to carry out, esp. to carry out (a corpse for burial)}
\vocabentry{εὑρίσκω}{to find}
\vocabentry{μισθός, ὁ}{wages, pay, hire}
\vocabentry{μισθόω}{to let out for hire, to farm out; (mid.) to hire}
\vocabentry{Πλούτων, ωνος, ὁ}{Pluto, god of the netherworld}
\vocabentry{πόσος, ός, ά, όν}{how much? how many?}
\vocabentry{σκευάριον, τό}{a small vessel or utensil (dim. σκεῦος vessel; (pl.) baggage)}
\vocabentry{στρῶμα, ατος, τό}{anything spread or laid out for lying or sitting upon, mattress, bed; (pl.) bedding}
\vocabentry{τελέω}{to complete, fulfill, accomplish; to pay}
\vocabentry{ὑγιαίνω}{to be sound, healthy; (impv.) farewell, goodbye, like χαῖρε}
\end{multicols}

\vspace{-1.5em}
\noindent\rule{\linewidth}{0.4pt}
\vspace{-2em}

\begin{multicols}{2}
\begin{parcolumns}[colwidths={1=1.5em, 2=0.9\linewidth}]{2}
\small

\glossline{161}{%
 \glossmix{φράσουσ'}{φράσουσι}\\
 \glossmix{δέῃ}{2s subj. δέομαι. For lack of contraction see Smyth 397a.}\\
 \glossmix{ὧν ἂν δέῃ}{\textit{whatever you need}; generalizing relative clause with attracted relative pronoun.}
}

\glossline{162}{%
 \glossmix{παρ'}{LSJ C.1.2. \textit{beside, near, by}}\\
 \glossmix{αὐτὴν τὴν ὁδὸν}{\textit{the road itself}, i.e. close to the road} 
}

\glossline{163}{%
 \glossmix{ἐπὶ}{LSJ Β.1.1. \textit{at, near}}\\
 \glossmix{οἰκοῦσιν}{Finite verb inside the noun clause}
}

\glossline{164}{%
 \glossmix{ὦδελφέ}{ὦ ἀδελφέ. Retention of circumflex an apparent violation of Smyth 131 and KG 51.5.c.}
}

\glossline{165}{%
 \glossmix{Σὺ δὲ}{Heracles returns to the house, and D. turns to Xanthias.}
}

\glossline{166}{%
 \glossmix{πρὶν}{πρὶν + inf. = \textit{before ...ing}. Xanthias is apparently already beginning to put down some of his baggage.}\\
 \glossmix{καταθέσθαι}{aor. mid. imp. κατατίθημι}
 \glossmix{Καὶ... μέντοι}{"μέντοι gives liveliness and force to the addition... almost always progressive in meaning" (GP 413).}
}

\glossline{167}{%
 \glossmix{Μὴ δῆθ'}{Cf. l. 11. Sc. something like "don't [ask that of me]."}\\
 \glossmix{μίσθωσαί}{Aor. mid. impv. μισθόω}
}

\glossline{168}{%
 \glossmix{ἐκφερομένων}{The joke lies partly in the ambiguity of whether this form is middle or passive.}
 \glossmix{ὅστις ἐπὶ τοῦτ' ἔρχεται}{\textit{whoever is coming for this [purpose]}, where the purpose might be "(a) to go to Hades; (b) to be buried; (c) to carry the baggage" (Stanford)}
}

\glossline{169}{%
 \glossmix{μηὕρω}{μὴ εὕρω, where εὕρω aor. subj. εὑρίσκω}\\
 \glossmix{Τότ' ἔμ' ἄγειν.}{\textit{then take me} (Stanford). Inf. for impv. Xanthias is willing but only if he must.}
}

\glossline{170}{%
 \glossmix{}{People appear, carrying a bier with a corpse.}\\
 \glossmix{Καὶ γάρ}{"For in fact." γάρ picks up on Xanthias' proposal. GP 108.}
}

\glossline{171}{%
 \glossmix{Οὗτος}{"οὗτος is regular in address" (Smyth 1288a), \textit{you there}}\\
 \glossmix{τεθνηκότα}{pf. pple. θνήσκω. D. unexpectedly talks to the corpse instead of the people carrying it.}
}

\glossline{173}{%
 \glossmix{}{The corpse responds.}\\
 \glossmix{ἄττα}{= τινά. \textit{about how many (bags)?}}\\
 \glossmix{ταυτί}{ταῦτα + deictic iota. D points to the baggage}\\
 \glossmix{μισθὸν}{\textit{[as my] wage}, in apposition.}\\
 \glossmix{τελεῖς}{Ambiguous but prob. fut. τελέω}
}

\end{parcolumns}
\end{multicols}

\newpage

\begin{spacing}{1.5}

\begin{tabularx}{\textwidth}{@{}lXr@{}}
  \textit{Δι.} & Μὰ Δί', ἀλλ' ἔλαττον. & \\
  \textit{Νε.} & \hspace*{9em} Ὑπάγεθ' ὑμεῖς τῆς ὁδοῦ. & \\
  \textit{Δι.} & Ἀνάμεινον, ὦ δαιμόνι', ἐὰν ξυμβῶ τί σοι. & 175 \\
  \textit{Νε.} & Εἰ μὴ καταθήσεις δύο δραχμάς, μὴ διαλέγου. & \\
  \textit{Δι.} & Λάβ' ἐννέ' ὀβολούς. & \\
  \textit{Νε.} & \hspace*{8em} Ἀναβιοίην νυν πάλιν. & \\
  \textit{Ξα.} & Ὡς σεμνὸς ὁ κατάρατος. Οὐκ οἰμώξεται; & \\
  & Ἐγὼ βαδιοῦμαι. & \\
  \textit{Δι.} & \hspace*{7em} Χρηστὸς εἶ καὶ γεννάδας. & \\
  & Χωρῶμεν ἐπὶ τὸ πλοῖον. & \\
  
\end{tabularx}

\noindent\textit{\MakeUppercase{\textls[200]{Χάρων}}}

\begin{tabularx}{\textwidth}{@{}lXr@{}}
  \phantom{Xά.} & \hspace*{10em} Ὠόπ, παραβαλοῦ. & 180 \\
  \textit{Ξα.} & Τουτὶ τί ἐστι; & \\
  \textit{Δι.} & \hspace*{6em} Τοῦτο; λίμνη νὴ Δία & \\
  & αὕτη 'στὶν ἣν ἔφραζε, καὶ πλοῖόν γ' ὁρῶ. & \\
  \textit{Ξα.} & Νὴ τὸν Ποσειδῶ κἄστι γ' ὁ Χάρων οὑτοσί. & \\
  \textit{Δι.} & Χαῖρ', ὦ Χάρων, χαῖρ', ὦ Χάρων, χαῖρ', ὦ Χάρων. & \\
  \textit{Χά.} & Τίς εἰς ἀναπαύλας ἐκ κακῶν καὶ πραγμάτων; & 185 \\
  & Τίς εἰς τὸ Λήθης πεδίον, ἢ 'ς Ὄκνου πλοκάς, & \\
  & ἢ 'ς Κερβερίους, ἢ 'ς κόρακας, ἢ 'πὶ Ταίναρον; & \\
\end{tabularx}

\end{spacing}

\newpage

\begin{multicols}{2}
\small % roughly 9pt
\vocabentry{ἀνάπαυλα, ἡ}{repose, rest; resting-place, inn}
\vocabentry{ἀνα-βιόω}{to come to life again, return to life}
\vocabentry{ἀνα-μένω}{to wait for, await}
\vocabentry{γεννάδας, ου, ὁ}{noble, generous}
\vocabentry{δαιμόνιος, α, ον}{of/belonging to a δαίμων;
marvelous; (voc.) good sir/lady}
\vocabentry{δια-λέγομαι}{to have discourse with, talk to}
\vocabentry{δραχμή, ἡ}{a handful; a drachma}
\vocabentry{ἐλάσσων, ον}{smaller, less}
\vocabentry{ἐννέα}{nine}
\vocabentry{κατ-άρατος, ον}{accursed, abominable}
\vocabentry{κόραξ, ακος, ὁ}{raven, crow}
\vocabentry{Λάβος, ὁ}{Labus}
\vocabentry{λήθη, ἡ}{forgetting, forgetfulness; (after Hom.) place of oblivion in Hades; "Λήθη as pr. n. of a river is not found" (LSJ)}
\vocabentry{οἰμώζω}{to wail aloud, lament}
\vocabentry{παρα-βάλλω}{to throw beside; (LSJ III.3) to bring alongside, (mid.) to bring [your boat] alongside [a dock]}
\vocabentry{πεδίον, τό}{a plain}
\vocabentry{πλοῖον, τό}{a floating vessel, a ship, vessel}
\vocabentry{πλοκή, ἡ}{twining, twisting}
\vocabentry{πρᾶγμα, ατος, τό}{deed, act; (pl.) circumstances, affairs; difficulties}
\vocabentry{σεμνός, ή, όν}{revered, august, holy; (LSJ III) proud, haughty}
\vocabentry{συμ-βαίνω}{to stand with the feet together; to come to an agreement, to come to terms}
\vocabentry{Ταίναρος, ἡ}{Taenarus (later neut. Ταίναρον)}
\vocabentry{ὑπ-άγω}{to lead; to go away; to go forward}
\vocabentry{Χαίρις, ὁ}{}
\vocabentry{χρηστός, ή, όν}{useful; (of people) good, honest}
\vocabentry{χωρέω}{to give way, withdraw; (after Hom.) go forward, advance}
\vocabentry{ὠόπ}{a cry of the κελευστής to give the time to the rowers}
\end{multicols}

\vspace{-1.5em}
\noindent\rule{\linewidth}{0.4pt}
\vspace{-2em}

\begin{multicols}{2}
\begin{parcolumns}[colwidths={1=1.5em, 2=0.9\linewidth}]{2}
\small

\glossline{174}{%
 \glossmix{Μὰ Δί', ἀλλ'}{\textit{God no, rather...}}\\
 \glossmix{τῆς ὁδοῦ}{\textit{along the road}, addressed to the bier bearers. Genitive difficult, perhaps a kind of genitive within which.}
}

\glossline{175}{%
 \glossmix{Ἀνάμεινον}{Aor. impv. ἀναμένω}\\
 \glossmix{δαιμόνι'}{δαιμόνιε. For meaning see l. 45.}
 \glossmix{ἐὰν}{\textit{in the hope that}, Smyth 2354}\\
 \glossmix{ξυμβῶ}{Aor. subj. συμβαίνω}
}

\glossline{176}{%
 \glossmix{καταθήσεις}{Fut. κατατίθημι}
}

\glossline{177}{%
 \glossmix{ἐννέ' ὀβολούς}{= 1.5 drachmas}\\
 \glossmix{Ἀναβιοίην νυν πάλιν}{Aor. opt. ἀναβιόω. A comic reversal of e.g. Ar. \textit{Eccl}. 977 ἀποθάνοιμ' ἄρα.}\\
 \glossmix{ξυμβῶ}{Aor. subj. συμβαίνω}
}

\glossline{178}{%
 \glossmix{Ὡς σεμνὸς ὁ κατάρατος}{I.e. Ὡς σεμνός [ἐστιν] ὁ κατάρατος}\\
 \glossmix{Οὐκ οἰμώξεται}{Almost a command, Smyth 1918.}
}

\glossline{179}{%
 \glossmix{βαδιοῦμαι}{Fut. βαδίζω. X. agrees to carry the baggage.}
}

\glossline{180}{%
 \glossmix{παραβαλοῦ}{Aor. mid. impv. παραβάλλω}
}

\glossline{182}{%
 \glossmix{ἣν ἔφραζε}{implied Heracles as subj.}
}

\glossline{183}{%
 \glossmix{κἄστι}{καί ἐστι}
 \glossmix{οὑτοσί}{οὖτος with deictic iota. \textit{And this right here is Charon.}}
}

\glossline{184}{%
 \glossmix{χαῖρ' ὦ Χάρων}{Σ claims the line comes wholesale from Achaios' \textit{Aithon}: Δημήτριός φησιν Ἀχαιοῦ ὅλον εἶναι ἐκ τοῦ Αἴθωνος. λέγουσι δ' αὐτὸ οἱ σάτυροι, "Χαῖρ' ὦ Χάρων χαῖρ' ὦ Χάρων χαῖρ' ὦ Χάρων."}
}

\glossline{183}{%
 \glossmix{κἄστι}{καί ἐστι}
}

\glossline{185}{%
 \glossmix{Τίς}{Sc. a verb of motion. Charon announces possible destinations.}
}

\glossline{186}{%
 \glossmix{'ς}{ἐς}\\
 \glossmix{Ὄκνου πλοκάς}{\textit{Oknos' twistings}. A sinner who twisted a rope that a donkey ate as quickly as he made it. A Sisyphean futile and endless labor. Depicted in a painting by Polygnotus in the Lesche in Athens (Paus. 10.29)}
}

\glossline{187}{%
 \glossmix{Κερβερίους}{\textit{the Cerberians}, play on Cerberus}\\
 \glossmix{κόρακας}{wordplay on the common curse ἐς κόρακας, \textit{go to the crows}, i.e. \textit{go to hell}, from a curse that crows eat your unburied body}\\
 \glossmix{'πὶ}{ἐπὶ}\\
 \glossmix{Ταίναρον}{Real; tip of the middle peninsula at the bottom of the Peloponnese, in Spartan territory; one of the locations where Heracles was said to have descended to Hades}
}

\end{parcolumns}
\end{multicols}

% https://www.researchgate.net/figure/Figura-4-Caronte-com-eidola-4966-Frankfurt-Liebieghaus-560_fig3_313682002
% creative commons license
\begin{figure}
    \centering
    \includegraphics[width=0.1\linewidth]{charon.png}
    %Earliest known depiction of Charon, c. 500.
\end{figure}

\newpage

\begin{spacing}{1.5}

\begin{tabularx}{\textwidth}{@{}lXr@{}}
  \textit{Δι.} & ἐγώ. & \\
  \textit{Χα.} & \hspace{2.5em}ταχέως ἔμβαινε. & \\
  \textit{Δι.} & \hspace{10em} ποῖ σχήσειν δοκεῖς; & \\
  & ἐς κόρακας ὄντως; & \\
  \textit{Χα.} & \hspace{9em}ναὶ μὰ Δία σοῦ γʼ οὕνεκα. & \\
  & ἔσβαινε δή. & \\
  \textit{Δι.} & \hspace{6em}παῖ δεῦρο. & \\
  \textit{Χα.} & \hspace{11em}δοῦλον οὐκ ἄγω, & 190 \\
  & εἰ μὴ νεναυμάχηκε τὴν περὶ τῶν κρεῶν. & \\
  \textit{Ξα.} & μὰ τὸν Δίʼ οὐ γὰρ ἀλλʼ ἔτυχον ὀφθαλμιῶν. & \\
  \textit{Χα.} & οὔκουν περιθρέξει δῆτα τὴν λίμνην κύκλῳ; & \\
  \textit{Ξα.} & ποῦ δῆτʼ ἀναμενῶ; & \\
  \textit{Χα.} & \hspace{8em} παρὰ τὸν Αὑαίνου λίθον & \\
  & ἐπὶ ταῖς ἀναπαύλαις. & \\
  \textit{Δι.} & \hspace{9em} μανθάνεις; & \\
  \textit{Ξα.} & \hspace{14em} πάνυ μανθάνω. & 195 \\
  & οἴμοι κακοδαίμων, τῷ ξυνέτυχον ἐξιών; & \\
  \textit{Χα.} & κάθιζʼ ἐπὶ κώπην. εἴ τις ἔτι πλεῖ, σπευδέτω. & \\
  & οὗτος τί ποιεῖς; & \\
  \textit{Δι.} & \hspace{7em} ὅ τι ποιῶ; τί δʼ ἄλλο γʼ ἢ & \\
  & ἵζω ʼπὶ κώπην, οὗπερ ἐκέλευές με σύ; & \\
  
\end{tabularx}

\end{spacing}

\newpage

\begin{multicols}{2}
\small % roughly 9pt
\vocabentry{ἀνά-παυλα, ἡ}{repose, rest; resting-place, inn}
\vocabentry{ἀνα-μένω}{to wait for, await}
\vocabentry{δεῦρο}{to here, hither}
\vocabentry{εἰσ-βαίνω}{to go on board (a ship)}
\vocabentry{ἐμ-βαίνω}{to step in; to embark (on a ship)}
\vocabentry{ἵζω}{to make to sit, seat, place}
\vocabentry{καθ-έζομαι}{to sit down, take one's seat}
\vocabentry{κελεύω}{to urge}
\vocabentry{κόραξ, ακος, ὁ}{crow}
\vocabentry{κρέας, κρέως, τό}{meat}
\vocabentry{κύκλος, ὁ}{a ring, circle, round}
\vocabentry{κώπη, ἡ}{the handle of an oar}
\vocabentry{λίθος, ὁ}{a stone}
\vocabentry{ναί}{(adv.) yea, verily, yes}
\vocabentry{ναυ-μαχέω}{to fight by sea}
\vocabentry{ὄντως}{(adv.) really, actually > εἰμί}
\vocabentry{οὔκουν}{certainly not}
\vocabentry{ὀφθαλμία, ἡ}{ophthalmia, an eye disease; freq. in pl.}
\vocabentry{οὗπερ}{where (rel. pronoun)}
\vocabentry{περι-τρέχω}{to run round and round, run around}
\vocabentry{ποῖ}{to where? whither?}
\vocabentry{συν-τυγχάνω}{to meet with, fall in with (+ dat.)}
\vocabentry{σπεύδω}{to urge on, hasten, quicken}
\vocabentry{τυγχάνω}{to happen to be; to obtain (+ gen.)}

\end{multicols}

\vspace{-1.5em}
\noindent\rule{\linewidth}{0.4pt}
\vspace{-2em}

\begin{multicols}{2}
\begin{parcolumns}[colwidths={1=1.5em, 2=0.9\linewidth}]{2}
\small

\glossline{188}{%
 \glossmix{σχήσειν}{Fut. inf. ἔχω, LSJ II.8, \textit{hold in a certain direction; (of horses and ships) guide, drive, steer}}\\
 \glossmix{ποῖ σχήσειν δοκεῖς;}{\textit{where do you intend to steer?}}
}

\glossline{189}{%
 \glossmix{ναὶ μὰ...}{affirmation, \textit{yes, by...}}\\
 \glossmix{παῖ}{Voc. παῖς}
}

\glossline{191}{%
 \glossmix{τὴν}{Sc. μάχην from ναυ\textbf{μαχ}έω. Disputed. Presumably Charon refers to Arginusai, a major Athenian naval defeat; slaves who had fought were freed (cf. l. 33). Possibly κρεῶν here (bafflingly) means 'corpses,' since famously many corpses had been left behind to rot in the water (Σ). Modern commentators also point to a proverb ὁ λαγὼς τὸν περὶ τῶν κρεῶν τρέχει 'the hare runs [the race] for his own meat (to save his bacon'), first attested in Photius.} 
}

\glossline{192}{%
 \glossmix{μὰ}{μά is used on its own in negation when a negative follows (LSJ III.1); \textit{No, by Zeus...}}
 \glossmix{οὐ γὰρ ἀλλʼ}{Denniston suggests separating οὺ γὰρ and ἀλλ' and translating \textit{No, not I: I happened to have opthalmia} (GP 31). Apparently Xanthias' excuse for not fighting.} 
}

\glossline{193}{%
 \glossmix{οὔκουν}{"freq. with 2sg. fut., to express an urgent or impatient imper", in "impatient or excited questions" (LSJ II, Smyth 2953d). Charon tells Xanthias to run around the lake and they will meet him on the other side.}\\ 
 \glossmix{περιθρέξει}{2s fut. mid. (deponent) περιτρέχω}
}

\glossline{194}{%
 \glossmix{ἀναμενῶ}{Fut. X. asks where he should wait for D. on the other side.}\\
 \glossmix{τὸν Αὑαίνου λίθον}{\textit{the Withering stone}, presumably from αὐαίνω to dry (Att. αὑαίνω). Unclear. Dover dismisses the scholia's possibilities that the stone was a landmark in Attica or the underworld as "pure guesswork."}
}

\glossline{196}{%
 \glossmix{οἴμοι κακοδαίμων}{See note on l. 33}\\ 
 \glossmix{τῷ}{= τίνι < τίς. \textit{what [thing] did I happen on while leaving [my house].} Commentators assume a reference to a superstition but no good parallel elsewhere.}
}

\glossline{197}{%
 \glossmix{ἐπὶ κώπην}{\textit{at the oar.}, lit. \textit{to} (LSJ C.I.2)}\\
 \glossmix{εἴ τις ἔτι πλεῖ,}{i.e. if anyone is still interested in sailing, hurry up.}\\
  \glossmix{σπευδέτω}{3rd person impv. σπεύδω}
}

\glossline{198}{%
 \glossmix{οὗτος}{See note on l. 171}\\ 
 \glossmix{ὅ τι}{neut. ὅστις, indirect interrogative. \textit{"[You're asking me] what..."}}\\
 \glossmix{τί δʼ ἄλλο γʼ ἢ}{\textit{What other than}. Cf. \textit{Nu}. 1495.}
}
\glossline{199} {%
 \glossmix{’πὶ κώπην}{While Charon used ἐπὶ to mean \textit{at} the oar, Dionysus has misunderstood (perhaps deliberately). Perhaps he is sitting on it.}
}

\end{parcolumns}
\end{multicols}

\newpage

\begin{spacing}{1.5}

\begin{tabularx}{\textwidth}{@{}lXr@{}}

    \textit{Χα.} & οὔκουν καθεδεῖ δῆτʼ ἐνθαδὶ, γάστρων; & \\
    \textit{Δι.} & \hspace{16em} ἰδού. & 200 \\
    \textit{Χα.} & οὔκουν προβαλεῖ τὼ χεῖρε κἀκτενεῖς; &  \\
    \textit{Δι.} & \hspace{16em} ἰδού. &  \\
    \textit{Χα.} & οὐ μὴ φλυαρήσεις ἔχων ἀλλʼ ἀντιβὰς &  \\
    & ἐλᾷς προθύμως; &  \\
    \textit{Δι.} & \hspace{7em}κᾆτα πῶς δυνήσομαι &  \\
    & ἄπειρος ἀθαλάττωτος ἀσαλαμίνιος &  \\
    & ὢν εἶτʼ ἐλαύνειν; & \\
    \textit{Χα.} & \hspace{7em} ῥᾷστʼ· ἀκούσει γὰρ μέλη & 205 \\
    & κάλλιστʼ, ἐπειδὰν ἐμβάλῃς ἅπαξ, &  \\
    \textit{Δι.} & \hspace{14em} τίνων; &  \\
    \textit{Χα.} & βατράχων κύκνων θαυμαστά. &  \\
    \textit{Δι.} & \hspace{13em}κατακέλευε δή. &  \\
    \textit{Χα.} & ὦ ὀπὸπ ὦ ὀπόπ. &  \\
\end{tabularx}

\noindent\textit{\MakeUppercase{\textls[200]{Βάτραχοι}}}

\begin{tabularx}{\textwidth}{@{}lXr@{}}
    & βρεκεκεκὲξ κοὰξ κοάξ, &  \\
    & βρεκεκεκὲξ κοὰξ κοάξ. & 210 \\
    & λιμναῖα κρηνῶν τέκνα, &  \\

\end{tabularx}

\end{spacing}

\newpage

\begin{multicols}{2}
\small % roughly 9pt
\vocabentry{ἄ-πειρος, ον}{inexperienced}
\vocabentry{ἀ-θαλάσσωτος, ον (Att. ἀθαλάττωτος)}{unused to the sea, a land-lubber}
\vocabentry{ἀντι-βαίνω}{to go against, withstand, resist}
\vocabentry{ἀ-σαλαμίνιος, ον}{not having been at Salamis}
\vocabentry{ἅπαξ}{(adv.) once, immediately}
\vocabentry{βάτραχος, ὁ}{a frog}
\vocabentry{γάστρων, οντος, ὁ}{comic proper name formed from γαστήρ stomach}
\vocabentry{ἐκ-τείνω}{to stretch out}
\vocabentry{ἐλαύνω}{to drive; to travel, drive, sail}
\vocabentry{ἐμ-βάλλω}{to throw in; (LSJ II.3) κώπῃς ἐ. (sc. χεῖρας) to lay oneself to the oars; (ἐ. alone), to pull hard}
\vocabentry{ἐνθαδί}{(adv.) here}
\vocabentry{ἐπειδάν}{when, whenever}
\vocabentry{θαυμαστός, ά, όν}{wondrous, wonderful, marvelous}
\vocabentry{καθ-έζομαι}{to sit down, take one's seat}
\vocabentry{κατα-κελεύω}{to command silence; to give the time in oaring}
\vocabentry{κρήνη, ἡ}{a well, spring, fountain}
\vocabentry{κύκνος, ὁ}{a swan}
\vocabentry{λιμναῖος, ά, όν}{of or from the marsh < λίμνη marsh}
\vocabentry{μέλος, εος, τό}{a song, melody}
\vocabentry{πρόθυμος, ον}{ready, willing, eager, zealous}
\vocabentry{προ-βάλλω}{to throw before/in front, throw}
\vocabentry{ῥᾷστα}{(adv.) very easily; ῥᾷστος superlative of ῥᾴδιος easy}
\vocabentry{τέκνον, τό}{a child}
\vocabentry{φλυαρέω}{to talk nonsense, play the fool}
\vocabentry{ὠόπ}{(also ὠὸπ ὄπ) a cry of the κελευστής to give the time to the rowers}
\end{multicols}

\vspace{-1.5em}
\noindent\rule{\linewidth}{0.4pt}
\vspace{-2em}

\begin{multicols}{2}
\begin{parcolumns}[colwidths={1=1.5em, 2=0.9\linewidth}]{2}
\small

\glossline{200}{%
 \glossmix{καθεδεῖ}{2s fut. καθέζομαι. For οὔκουν + fut. cf. l. 193.}\\
 \glossmix{γάστρων}{\textit{Fatty}}\\
 \glossmix{ἰδού}{\textit{Look!}}
}

\glossline{201}{%
 \glossmix{προβαλεῖ}{2s fut. mid. προβάλλω}\\
 \glossmix{τὼ χεῖρε}{dual}\\
 \glossmix{κἀκτενεῖς}{καὶ ἐκτενεῖς, also a 2s fut. ἐκτείνω. In using these two verbs, Charon attempts to tell Dionysus to perform the act of rowing (i.e., cast forward his hands while holding onto the oar).}
}

\glossline{202}{%
 \glossmix{οὐ μὴ}{οὐ μὴ + 2s fut. ind. "in the dramatic poets denotes a strong prohibition" (Smyth 1919, 2756).}\\
 \glossmix{ἔχων}{The participle of ἔχω with a present verb means to keep on doing something (LSJ IV.2).}\\
 \glossmix{ἀντιβὰς}{refers to planting your foot firmly against the bottom of the ship (or a footrest designed for that purpose) to brace yourself as you row.}
}

\glossline{203}{%
 \glossmix{ἐλᾷς}{Att. fut. ἐλαύνω}
}

\glossline{204}{%
 \glossmix{ἀσαλαμίνιος}{Salamis, an island near Athens, was the site of a famous sea-battle between the Greek alliance and the Persian empire in 480 BCE. Alternatively, perhaps people from Salamis were good sailors (Stanford, Dover), but no obvious ancient source for that claim.}
}

\glossline{205}{%
 \glossmix{ἐλαύνειν}{\textit{to row/sail}, complimentary infinitive of δυνήσομαι above.}\\
 \glossmix{ἀκούσει}{2s fut. mid.} \\
 \glossmix{μέλη}{acc. pl., contracted from μέλεα. The song helps rowers keep time.}
}

\glossline{206}{%
 \glossmix{ἐπειδὰν... ἅπαξ}{\textit{whenever... once}, i.e. \textit{as soon as}. LSJ s.v. ἅπαξ ΙΙ}
}

\glossline{207}{%
 \glossmix{βατράχων κύκνων}{The two nouns are in apposition. Swans were thought to sing a beautiful song when about to die, and were used frequently to describe beautiful singing (e.g. Eur. \textit{IT} 1404f., \textit{HF} 692ff., cf. the English phrase "swan song")."Swan-frogs" means singer-frogs, but the impossibility of a creature being a swan and frog at once heightens the joke of a sweet-singing chorus of frogs.}\\
 \glossmix{θαυμαστά}{Agreeing with μέλη}
}

\glossline{209}{%
 \glossmix{βρεκεκεκὲξ κοὰξ κοάξ}{The Frog chorus begins its song with a famous imitation of the croaking of frogs (which sounds like a rower's beat?). The chorus of frogs only occurs here in this song, replaced throughout the rest of the play by a chorus of dead initiates. Such a wholesale change in chorus is unique in comedy. Some have speculated (with Σ) that the chorus is only heard and did not appear on stage (cf. 205 ἀκούσει).\\ \\ The bulk of Attic drama is dialogue in Attic iambic trimeters, but Attic choruses sing in lyric poetry. Crucial differences include:
 \begin{itemize}[leftmargin=*]
  \item mode of performance: song and dance
  \item meter: lyric meters
  \item vocabulary \& syntax: more poetic
  \item dialect: Doric, or at least Doricized Attic
\end{itemize}
The most obvious feature of Doric is that it often has long α for Attic η. This first lyric section is astrophic, i.e. it does not consist of two matching stanzas.}
}

\end{parcolumns}
\end{multicols}

\newpage

\begin{spacing}{1.5}

\begin{tabularx}{\textwidth}{@{}lXr@{}}
& ξύναυλον ὕμνων βοὰν &  \\
& φθεγξώμεθʼ, εὔγηρυν ἐμὰν ἀοιδάν, &  \\
& κοὰξ κοάξ, &  \\
& ἣν ἀμφὶ Νυσήιον & 215 \\
& Διὸς Διόνυσον ἐν &  \\
& Λίμναισιν ἰαχήσαμεν, &  \\
& ἡνίχʼ ὁ κραιπαλόκωμος &  \\
& τοῖς ἱεροῖσι Χύτροισι &  \\
& χωρεῖ κατʼ ἐμὸν τέμενος λαῶν ὄχλος. & 220 \\
& βρεκεκεκὲξ κοὰξ κοάξ. &  \\
\textit{Δι.} & ἐγὼ δέ γʼ ἀλγεῖν ἄρχομαι &  \\
& τὸν ὄρρον ὦ κοὰξ κοάξ· &  \\
& ὑμῖν δʼ ἴσως οὐδὲν μέλει. &  \\
\textit{Βα.} & βρεκεκεκὲξ κοὰξ κοάξ. & 225 \\
\textit{Δι.} & ἀλλʼ ἐξόλοισθʼ αὐτῷ κοάξ· &  \\
& οὐδὲν γάρ ἐστʼ ἀλλʼ ἢ κοάξ. &  \\
\textit{Βα.} & εἰκότως γʼ, ὦ πολλὰ πράττων. &  \\
& ἐμὲ γὰρ ἔστερξαν εὔλυροί τε Μοῦσαι &  \\
& καὶ κεροβάτας Πὰν ὁ καλαμόφθογγα παίζων· & 230 \\

\end{tabularx}

\end{spacing}

\newpage

\begin{multicols}{2}
\small % roughly 9pt
\vocabentry{ἄρχω}{(to be first) to begin, to rule}
\vocabentry{ἀλγέω}{to feel bodily pain, suffer}
\vocabentry{ἀοιδή, ἡ (Dor. ἀοιδά)}{a song}
\vocabentry{βοή, ἡ (Dor. βοά)}{a loud cry; (in lyric) song}
\vocabentry{εἰκότως}{(adv.) similarly, reasonably, naturally}
\vocabentry{ἐξ-όλλυμι}{to destroy utterly; (mid.) to perish}
\vocabentry{εὔ-γηρυς, εῖα, ύ}{sweet-sounding < γῆρυς voice}
\vocabentry{εὔ-λυρος, ον}{playing well on the lyre}
\vocabentry{ἡνίκα}{(at the time) when}
\vocabentry{ἴσως}{(adv.) equally; perhaps}
\vocabentry{ἰαχέω}{to cry, shout, shriek}
\vocabentry{καλαμό-φθογγος, ον}{voiceful reed < κάλαμος reed + φθόγγος voice, sound}
\vocabentry{κερο-βάτης, ὁ (Dor. -τας)}{horn-footed, hoofed}
\vocabentry{κραιπαλό-κωμος, ον}{rambling in drunken revelry; < κραιπάλη drinking bout + κῶμος drunken band of revellers}
\vocabentry{λαός, ὁ (Att. λεώς)}{the people}
\vocabentry{Λίμναι, αἱ}{a quarter of Athens (once prob. marshy), near the Acropolis, with temple of Dionysus (Thuc. 2.15.4) < λίμνη lake, marsh}
\vocabentry{Νυσήιος, α, ον}{adj for Νῦσα, name of several mountains sacred to Dionysus}
\vocabentry{ὄρρος, ὁ}{butt, ass; (cogn. Eng. ass)}
\vocabentry{ὄχλος, ὁ}{a moving crowd, a throng, mob}
\vocabentry{παίζω}{to play; (LSJ I.4) to play (an instrument)}
\vocabentry{στέργω}{to love}
\vocabentry{σύν-αυλος, ον (Att. ξύν-)}{in concert with the aulos; sounding in concord or unison, harmonious}
\vocabentry{τέμενος, εος, τό}{a piece of land cut off and assigned as an official domain, sacred precinct/district}
\vocabentry{ὕμνος, ὁ}{a hymn, festive song}
\vocabentry{φθέγγομαι}{to utter, to speak loud and clear}
\vocabentry{Χύτροι, οἱ}{pot-feast at Athens, the 3rd day of the Anthesteria at Athens, a major drinking festival < χύτρα clay pot}
\end{multicols}

\vspace{-1.5em}
\noindent\rule{\linewidth}{0.4pt}
\vspace{-2em}

\begin{multicols}{2}
\begin{parcolumns}[colwidths={1=1.5em, 2=0.9\linewidth}]{2}
\small

\glossline{212}{%
 \glossmix{ξύναυλον}{The \textit{aulos} was a pair of double-reed pipes played simultaneously by one player; used in public and private contexts. An aulete (aulos-player) provided the musical accompaniment for Attic drama.\\ Choral self-referentiality, where a chorus describes an action they are taking as characters but also simultaneously as actors.}\\
 \glossmix{βοὰν}{Doric for Att. βοὴν, likewise ἀοιδάν.}
}

\glossline{213}{%
 \glossmix{φθεγξώμεθʼ}{φθεγξώμεθα, aor. subj. φθέγγομαι, hortative}\\
 \glossmix{φθεγξώμεθʼ... ἐμὰν}{Note plural and singular in a single line. Typical of choruses, of disputed interpretation.}
}

\glossline{214}{%
 \glossmix{ἐμὰν}{Doric for ἐμὴν}\\
 \glossmix{ἀοιδάν}{in apposition to βοὰν.}
}

\glossline{215}{%
 \glossmix{ἣν}{antecedent ἀοιδάν.}\\
 \glossmix{ἀμφὶ}{(LSJ C.I.5) \textit{about, for the sake of}}\\
 \glossmix{Νυσήιον}{With Διόνυσον, in typically lyric hyperbaton. The frogs used to sing hymns to Dionysus but now are fighting with him.}
}

\glossline{216}{%
 \glossmix{Διὸς}{\textit{[son] of Zeus}}
}

\glossline{217}{%
 \glossmix{Λίμναισιν}{Cf. λιμναῖα (l. 211) and vocab. An Attic local historian described the first day ("Jar-opening") of the Anthesterion festival there: "At the sanctuary of Dionysos \textit{en limnais} the Athenians used to mix the wine for the god from the jars which they transported along there and then taste it themselves ... Delighted with the mixture, they celebrated Dionysos with songs, danced, and invoked him as the Fair-flowering, the Dithyrambos, the Reveller and the Stormer." (Phanodemus fr. 12 Brill's New Jacoby, trans. Burkert)}\\
 \glossmix{ἰαχήσαμεν}{Aor. The dead frogs remember the song that they sang while alive.}
}

\glossline{218}{%
 \glossmix{ὁ κραιπαλόκωμος}{with ὄχλος, l. 220.}
}

\glossline{219}{%
 \glossmix{Χύτροισι}{dative of time of which, regular with festivals, Smyth 1541.}
}

\glossline{220}{%
 \glossmix{κατʼ ἐμὸν τέμενος}{i.e the Limnai.}
}

\glossline{223}{%
 \glossmix{τὸν ὄρρον}{acc. of respect with ἀλγεῖν.}
}

\glossline{224}{%
 \glossmix{ὑμῖν δʼ ἴσως οὐδὲν μέλει.}{\textit{but maybe you don't care at all}, i.e. about Dionysus' pain in his butt. μέλει is impersonal with dative for the person doing the caring.}
}

\glossline{226}{%
 \glossmix{ἐξόλοισθʼ}{ἐξόλοισθε, 2p. aor. mid. opt. ἐξόλλυμι, opt. of wish. \textit{may you perish}}\\
 \glossmix{αὐτῷ κοάξ}{\textit{koax and all}, Smyth 1525. κοάξ evidently indeclinable (though it does not fit into the categories at Smyth 284, KG 142)}
}

\glossline{227}{%
 \glossmix{ἐστʼ}{ἐστι or better ἐστε}\\
 \glossmix{ἀλλʼ ἢ}{ἄλλο ἢ, \textit{other than}}
}

\glossline{228}{%
 \glossmix{εἰκότως γʼ}{\textit{Reasonably, [we are all koax]}}\\
 \glossmix{ὦ πολλὰ πράττων}{\textit{busybody}. (LSJ III.4) πολλὰ πράττειν = πολυπραγμονεῖν to be a meddlesome, inquisitive busybody. Nom. for voc.}
}

\glossline{229}{%
 \glossmix{ἔστερξαν}{Aor. στέργω. Gnomic aorist (Smyth 1931).}
}

\glossline{230}{%
 \glossmix{καλαμόφθογγα}{likely referring to the pan-pipe, or syrinx, though the aulos is also possible.}
}

\end{parcolumns}
\end{multicols}

\newpage

\begin{spacing}{1.5}

\begin{tabularx}{\textwidth}{@{}lXr@{}}
& προσεπιτέρπεται δʼ ὁ φορμικτὰς Ἀπόλλων, & 231-2 \\
& ἕνεκα δόνακος, ὃν ὑπολύριον &  \\
& ἔνυδρον ἐν λίμναις τρέφω. &  \\
& βρεκεκεκὲξ κοάξ κοάξ. & 235 \\
\textit{Δι.} & ἐγὼ δὲ φλυκταίνας γʼ ἔχω, &  \\
& χὠ πρωκτὸς ἰδίει πάλαι, &  \\
& κᾆτʼ αὐτίκʼ ἐκκύψας ἐρεῖ— &  \\
\textit{Βα.} & βρεκεκεκὲξ κοὰξ κοάξ. &  \\
\textit{Δι.} & ἀλλʼ ὦ φιλῳδὸν γένος & 240 \\
& παύσασθε. &  \\
\textit{Βα.} & \hspace{5em} μᾶλλον μὲν οὖν &  \\
& φθεγξόμεσθʼ, εἰ δή ποτʼ εὐηλίοις & 242a \\
& ἐν ἁμέραισιν & 242b \\
& ἡλάμεσθα διὰ κυπείρου &  \\
& καὶ φλέω, χαίροντες ᾠδῆς &  \\
& πολυκολύμβοισι μέλεσιν, & 245 \\
& ἢ Διὸς φεύγοντες ὄμβρον &  \\
& ἔνυδρον ἐν βυθῷ χορείαν &  \\
& αἰόλαν ἐφθεγξάμεσθα &  \\
& πομφολυγοπαφλάσμασιν. &  \\
\end{tabularx}

\end{spacing}

\newpage

\begin{multicols}{2}
\small % roughly 9pt
\vocabentry{ἅλλομαι}{to spring, leap, bound}
\vocabentry{αἰόλος, α, ον}{quick-moving}
\vocabentry{βυθός, ὁ}{the depth}
\vocabentry{δόναξ, ακος, ὁ}{reed}
\vocabentry{ἔν-υδρος, ον}{with water in it, holding water; living in the water}
\vocabentry{ἐκ-κύπτω}{to peep out of}
\vocabentry{ἕνεκα}{(+ gen.) on account of, for the sake of, because of, for}
\vocabentry{εὐ-ήλιος, ον}{well-sunned, sunny, genial}
\vocabentry{ἡμέρα, ἡ (Dor. ἁμέρα)}{day}
\vocabentry{ἰδίω}{to sweat}
\vocabentry{κύπειρος, ὁ}{galangal, similar to ginger or turmeric}
\vocabentry{μέλος, εος, τό}{limb; song; (pl. μέλη) lyric poetry, choral songs}
\vocabentry{ὄμβρος, ὁ}{rain}
\vocabentry{πάλαι}{(adv.) long ago, in olden time, in days of yore, in time gone by}
\vocabentry{πολυ-κόλυμβος, ον}{oft-diving < κολυμβάω to dive}
\vocabentry{πομφολυγο-πάφλασμα, ματος, τό}{the noise made by bubbles rising < πομφόλυξ bubble + παφλάζω to splash}
\vocabentry{προσ-επι-τέρπομαι}{to rejoice in besides, to delight in also}
\vocabentry{πρωκτός, ὁ}{ass, the anus}
\vocabentry{ὑπο-λύριος, ον}{under the lyre}
\vocabentry{φεύγω}{to flee, take flight, run away}
\vocabentry{φιλ-ῳδός, όν}{song-loving}
\vocabentry{φλέως, ω, ὁ}{wool-tufted reed}
\vocabentry{φλύκταινα, ἡ}{a blister}
\vocabentry{φορμ-ικτής, οῦ (Dor. -τάς), ὁ,}{phorminx-player}
\vocabentry{χαίρω}{to rejoice, be glad; (+ dat.) to rejoice at, take pleasure in}
\vocabentry{χορεία}{dance, esp. choral dance with music}
\vocabentry{ᾠδή, ἡ}{a song, lay, ode}
\end{multicols}

\vspace{-1.5em}
\noindent\rule{\linewidth}{0.4pt}
\vspace{-2em}

\begin{multicols}{2}
\begin{parcolumns}[colwidths={1=1.5em, 2=0.9\linewidth}]{2}
\small

\glossline{231-2}{%
 \glossmix{φορμικτὰς}{The \textit{phorminx} is either an alternate name for the lyre in general, or a more specific type of lyre without a large sounding bowl. The instruments and gods mentioned in this song are all traditional pairings.}
}

\glossline{232}{%
 \glossmix{ὑπολύριον / ἔνυδρον}{Predicate adjs with the reed. \textit{the reed, which I nurture... living in the water...}}
}

\glossline{233}{%
 \glossmix{ἕνεκα δόνακος... ὑπολύριον}{The reed is part of the interior ('under') of the lyre. The \textit{Homeric Hymn to Hermes} tells the story of an infant Hermes creating the first lyre and trading it away to his older brother Apollo. Hermes bores holes into a tortoiseshell, stretches δόνακας καλάμοιο through them and ties them off, then spreads an oxhide over the opening. This explanation of the high esteem of the frogs applies not only to Apollo, but also to the Muses (who likewise play lyre), and Pan (whose pipes are made of reeds) mentioned above.}
}

\glossline{237}{%
 \glossmix{χὠ}{καὶ ὁ}\\
 \glossmix{ἰδίει πάλαι}{\textit{has been sweating since long ago, for a long time}, present of continuing effect, Smyth 1885}
}

\glossline{238}{%
 \glossmix{κᾆτʼ}{καὶ εἶτα}\\
 \glossmix{ἐκκύψας ἐρεῖ}{Aor. pple. ἐκκύπτω. The subject is πρωκτὸς, or the liquid seeping out of it.}\\
 \glossmix{ἐρεῖ}{Fut. λέγω}
}

\glossline{239}{%
 \glossmix{βρεκεκεκὲξ κοὰξ κοάξ.}{Evidently the sound a butt makes.}
}

\glossline{241}{%
 \glossmix{μὲν οὖν}{\textit{No, on the contrary} (GP 475).}
}

\glossline{242a}{%
 \glossmix{φθεγξόμεσθʼ}{φθεγξόμεσθα, 1p fut. φθέγγομαι. The ending -μεσθα for -μεθα is common in epic and drama (Smyth 465d).}
}

\glossline{242b}{%
 \glossmix{ἁμέραισιν}{Dor. for Att. ἡμέραις. The ending -αισι(ν) for Att. -αις is found in epic and Attic poetry (Smyth 215).}
}

\glossline{243}{%
 \glossmix{ἡλάμεσθα}{1p aor. mid. ἅλλομαι.}
}

\glossline{247}{%
 \glossmix{ἔνυδρον}{ambiguous with ὄμβρον or χορείαν}
}

\glossline{248}{%
 \glossmix{αἰόλαν}{With χορείαν}\\
 \glossmix{ἐφθεγξάμεσθα}{1p aor. φθέγγομαι}
}

\end{parcolumns}
\end{multicols}

\newpage
\begin{spacing}{1.5}
\begin{tabularx}{\textwidth}{@{}lXr@{}}
\textit{Δι.} & βρεκεκεκὲξ κοὰξ κοάξ. &  \\
& τουτὶ παρʼ ὑμῶν λαμβάνω. &  \\
\textit{Βα.} & δεινά τἄρα πεισόμεσθα. &  \\
\textit{Δι.} & δεινότερα δʼ ἔγωγʼ, ἐλαύνων &  \\
& εἰ διαρραγήσομαι. & 255 \\
\textit{Βα.} & βρεκεκεκὲξ κοὰξ κοάξ. &  \\
\textit{Δι.} & οἰμώζετʼ· οὐ γάρ μοι μέλει. &  \\
\textit{Βα.} & ἀλλὰ μὴν κεκραξόμεσθά γʼ &  \\
& ὁπόσον ἡ φάρυξ ἂν ἡμῶν &  \\
& χανδάνῃ διʼ ἡμέρας. & 260 \\
\textit{Δι.} & βρεκεκεκὲξ κοὰξ κοάξ. &  \\
& τούτῳ γὰρ οὐ νικήσετε. &  \\
\textit{Βα.} & οὐδὲ μὴν ἡμᾶς σὺ πάντως. &  \\
\textit{Δι.} & οὐδὲ μὴν ὑμεῖς γʼ ἐμὲ &  \\
& οὐδέποτε· κεκράξομαι γὰρ &  \\
& κἂν δέῃ διʼ ἡμέρας & 265 \\
& βρεκεκεκὲξ κοὰξ κοάξ, & 265a \\
& ἕως ἂν ὑμῶν ἐπικρατήσω τῷ κοάξ, &  \\
& βρεκεκεκὲξ κοὰξ κοάξ. &  \\
\end{tabularx}
\end{spacing}

\newpage
\begin{multicols}{2}
\small % roughly 9pt
\vocabentry{δεινός}{fearful, terrible}
\vocabentry{δια-ρρήγνυμι}{to break through, cleave asunder}
\vocabentry{ἐπι-κρατέω}{to rule over; (LSJ II.2) (freq. w. gen.) to prevail over, get the mastery of}
\vocabentry{ἕως}{until; as long as}
\vocabentry{ἡμέρα, ἡ}{day}
\vocabentry{κράζω}{to croak (freq. in pf. with pres. sense)}
\vocabentry{ὁπόσος, η, ον}{as much/many as}
\vocabentry{οὐδέ-ποτε}{(adv.) never}
\vocabentry{πάντως}{(adv.) altogether}
\vocabentry{φάρυγξ (only here φάρυξ), φάρυγος, ἡ}{throat}
\vocabentry{χανδάνω}{to take in, hold, comprise, contain}
\end{multicols}

% vertical line across page separating vocab and notes
\vspace{-1.5em}
\noindent\rule{\linewidth}{0.4pt}
\vspace{-2em}

\begin{multicols}{2}
\begin{parcolumns}[colwidths={1=1.5em, 2=0.9\linewidth}]{2}
\small

\glossline{252}{%
 \glossmix{τουτὶ παρʼ ὑμῶν λαμβάνω}{\textit{I'm taking this here from you}, i.e. D. is stealing the refrain.}
}

\glossline{253}{%
 \glossmix{τἄρα}{τοι ἄρα}\\
 \glossmix{πεισόμεσθα}{Form is ambiguous but likely fut. πείθω but rather fut. πάσχω, which frequently is with an adj. like δεινά (LSJ III).}
}

\glossline{254}{%
 \glossmix{ἐλαύνων}{See l. 205}
}

\glossline{255}{%
 \glossmix{διαρραγήσομαι}{Fut. pass. διαρρήγνυμι}
}

\glossline{257}{%
 \glossmix{οἰμώζετʼ}{οἰμώζετε. "in familiar Att., οἴμωζε, as a curse, plague take you!" (LSJ I.2)}
}

\glossline{258}{%
 \glossmix{ἀλλὰ μὴν}{\textit{All right then} (GP 342)}\\
 \glossmix{κεκραξόμεσθά}{Fut. pf. κράζω}
}

\glossline{259}{%
 \glossmix{φάρυξ}{The MSS are split between φάρυξ and φάρυγξ; based on IE parallels φάρυξ is likely the older form, but φάρυγξ appears from the 5th c. onward. See Merisio on φάρυγξ (Digital Encyclopedia of Atticism).}
}

\glossline{260}{%
 \glossmix{ὁπόσον... ἂν... χανδάνῃ}{\textit{as much as... holds}, more vivid conditional relative, Smyth 2565}\\
 \glossmix{διʼ ἡμέρας}{\textit{the whole day long, all day} (LSJ s.v. ἡμέρας III)}
}

\glossline{262}{%
 \glossmix{γὰρ}{\textit{[I'll keep saying koax koax,] for...}}
}

\glossline{263}{%
 \glossmix{οὐδὲ μὴν}{\textit{Nor indeed} (GP 338-9).}
}

\glossline{264}{%
 \glossmix{κεκράξομαι}{cf. 258}
}

\glossline{265}{%
 \glossmix{κἂν}{καὶ ἂν}\\
 \glossmix{δέῃ}{here 3s act, not 2s mid (unlike 161)}
}

\glossline{266}{%
 \glossmix{ἕως ἂν... ἐπικρατήσω}{\textit{until...}; ἕως means \textit{until} when action is in the future and subordinate clause has ἕως ἄν + aor. subj, Smyth 2426}\\
 \glossmix{ἐπικρατήσω}{Aor. subj.}
}


\end{parcolumns}
\end{multicols}

% begin template for new page
\newpage
\begin{spacing}{1.5}
\begin{tabularx}{\textwidth}{@{}lXr@{}}
\textit{Δι.} & ἔμελλον ἄρα παύσειν ποθʼ ὑμᾶς τοῦ κοάξ. &  \\
\textit{Χα.} & ὢ παῦε παῦε, παραβαλοῦ τῷ κωπίῳ, &  \\
& ἔκβαινʼ, ἀπόδος τὸν ναῦλον. & \\
\textit{Δι.} & \hspace{12em}ἔχε δὴ τὠβολώ. & 270 \\
& ὁ Ξανθίας. ποῦ Ξανθίας; ἤ, Ξανθία. &  \\
\textit{Ξα.} & ἰαῦ. &  \\
\textit{Δι.} & \hspace{2em}βάδιζε δεῦρο. &  \\
\textit{Ξα.} & \hspace{8em}χαῖρʼ ὦ δέσποτα. &  \\
\textit{Δι.} & τί ἔστι τἀνταυθοῖ; &  \\
\textit{Ξα.} & \hspace{8em}σκότος καὶ βόρβορος. &  \\
\textit{Δι.} & κατεῖδες οὖν που τοὺς πατραλοίας αὐτόθι &  \\
& καὶ τοὺς ἐπιόρκους, οὓς ἔλεγεν ἡμῖν; &  \\
\textit{Ξα.} & \hspace{15.5em}σὺ δʼ οὔ; & 275 \\
\textit{Δι.} & νὴ τὸν Ποσειδῶ ʼγωγε, καὶ νυνί γʼ ὁρῶ. &  \\
& ἄγε δὴ τί δρῶμεν; &  \\
\textit{Ξα.} & \hspace{7.5em}προϊέναι βέλτιστα νῷν, &  \\
& ὡς οὗτος ὁ τόπος ἐστὶν οὗ τὰ θηρία &  \\
& τὰ δείνʼ ἔφασκʼ ἐκεῖνος. &  \\
\textit{Δι.} & \hspace{10em}ὡς οἰμώξεται. &  \\
& ἠλαζονεύεθʼ ἵνα φοβηθείην ἐγώ, & 280 \\


\end{tabularx}
\end{spacing}

\newpage
\begin{multicols}{2}
\small % roughly 9pt
\vocabentry{ἀλαζονεύομαι}{to brag, exaggerate}
\vocabentry{ἀπο-δίδωμι}{to pay, return}
\vocabentry{αὐτόθι}{(adv.) = αὐτοῦ, there, here}
\vocabentry{βέλτιστος, α, ον}{best}
\vocabentry{βόρβορος, ὁ}{mud, mire}
\vocabentry{ἐνταυθοῖ}{(adv.) hither; here}
\vocabentry{ἐπί-ορκος, ον}{sworn falsely, perjured}
\vocabentry{ἤ}{an exclamation expressing disapproval; or to call attention; hey!}
\vocabentry{θηρίον, τό}{a wild animal, beast}
\vocabentry{ἰαῦ}{ho! holla! yo!}
\vocabentry{κωπίον, τό}{Dim. of κώπη, oar}
\vocabentry{ναῦλος, ὁ}{passage-money, the fare}
\vocabentry{οὗ}{where (rel. pronoun)}
\vocabentry{παρα-βάλλω}{to throw beside; (LSJ III.3) to bring alongside, (mid.) to bring [your boat] alongside [a dock]}
\vocabentry{πατραλοίας, gen. α and ου, ὁ}{one who slays his father, a parricide}
\vocabentry{πρό-ειμι}{to go forward, advance}
\vocabentry{σκότος, εος, τό}{darkness, gloom}
\vocabentry{τόπος, ὁ}{a place}
\vocabentry{φοβέω}{to put to flight, to terrify; mid. to fear}
\end{multicols}

% vertical line across page separating vocab and notes
\vspace{-1.5em}
\noindent\rule{\linewidth}{0.4pt}
\vspace{-2em}

\begin{multicols}{2}
\begin{parcolumns}[colwidths={1=1.5em, 2=0.9\linewidth}]{2}
\small

% put notes here
\glossline{267}{%
 \glossmix{ἔμελλον ἄρα}{\textit{Turns out I was going to...}. As the choral song ends and dialogue in a conversational meter resumes, D. claims victory.}\\
 \glossmix{ὑμᾶς τοῦ κοάξ}{παύω can take an acc. of person and gen. of thing (stop acc. from doing gen.)}
}

\glossline{269}{%
 \glossmix{ἔκβαινʼ}{ἔκβαινε, pres. impv.}
 \glossmix{ἀπόδος}{Aor. impv. ἀποδίδωμι}
}

\glossline{270}{%
 \glossmix{τὠβολώ}{τὼ ὀβολώ, dual}
}

\glossline{271}{%
 \glossmix{ὁ Ξανθίας}{For nom. see note on l. 40}\\
 \glossmix{Ξανθία}{vocative of Ξανθίας.}
}

\glossline{273}{%
 \glossmix{τἀνταυθοῖ}{τὰ ἐνταυθοῖ. Two options: 1) \textit{the here [things], this stuff} or 2) \textit{the hither [things], the things on the way here}}
}

\glossline{274}{%
 \glossmix{κατεῖδες}{Aor. καθ-οράω}
}

\glossline{275}{%
 \glossmix{ἔλεγεν}{subject is Heracles.}
}

\glossline{276}{%
 \glossmix{ʼγωγε}{ἔγωγε}
 \glossmix{καὶ νυνί γʼ ὁρῶ}{i.e. the audience, cf. \textit{Nu.} 1096-8.}
}

\glossline{277}{%
 \glossmix{δρῶμεν}{Deliberative. subj.}\\
 \glossmix{προϊέναι βέλτιστα νῷν}{\textit{[it is] best for us to proceed}.}\\
 \glossmix{προϊέναι}{Pres. inf. πρόειμι}\\
 \glossmix{νῷν}{gen/dat dual of ἔγω, ἡμεῖς, here dative.}
}

\glossline{279}{%
 \glossmix{τὰ δείνʼ ἔφασκʼ ἐκεῖνος}{sc. εἶναι. \textit{where that guy said the terrible beasts were.} Cf. 143-4.}\\
 \glossmix{ὡς οἰμώξεται}{\textit{how he'll wail!}, cf. note on 257. Exclamatory ὡς or possibly \textit{[I assure you] that...} (Dover).}
}

\glossline{279}{%
 \glossmix{ἠλαζονεύεθʼ}{ἠλαζονεύετο, impf.}\\
 \glossmix{φοβηθείην}{aor. opt. φοβέομαι. Purpose clause, 2ndary sequence.}
}

\end{parcolumns}
\end{multicols}

\newpage
\begin{spacing}{1.5}
\begin{tabularx}{\textwidth}{@{}lXr@{}}
& εἰδώς με μάχιμον ὄντα φιλοτιμούμενος. &  \\
& οὐδὲν γὰρ οὕτω γαῦρόν ἐσθʼ ὡς Ἡρακλῆς. &  \\
& ἐγὼ δέ γʼ εὐξαίμην ἂν ἐντυχεῖν τινι &  \\
& λαβεῖν τʼ ἀγώνισμʼ ἄξιόν τι τῆς ὁδοῦ. &  \\
\textit{Ξα.} & νὴ τὸν Δία καὶ μὴν αἰσθάνομαι ψόφου τινός. & 285 \\
\textit{Δι.} & ποῦ ποῦ ʼστιν; &  \\
\textit{Ξα.} & \hspace{6.5em}ἐξόπισθεν. &  \\
\textit{Δι.} & \hspace{11em}ἐξόπισθʼ ἴθι. &  \\
\textit{Ξα.} & ἀλλʼ ἐστὶν ἐν τῷ πρόσθε. &  \\
\textit{Δι.} & \hspace{10.5em}πρόσθε νυν ἴθι. &  \\
\textit{Ξα.} & καὶ μὴν ὁρῶ νὴ τὸν Δία θηρίον μέγα. &  \\
\textit{Δι.} & ποῖόν τι; &  \\
\textit{Ξα.} & \hspace{4em}δεινόν· παντοδαπὸν γοῦν γίγνεται &  \\
& τοτὲ μέν γε βοῦς, νυνὶ δʼ ὀρεύς, τοτὲ δʼ αὖ γυνὴ & 290 \\
& ὡραιοτάτη τις. &  \\
\textit{Δι.} & \hspace{7em}ποῦ ʼστι; φέρʼ ἐπʼ αὐτὴν ἴω. &  \\
\textit{Ξα.} & ἀλλʼ οὐκέτʼ αὖ γυνή ʼστιν, ἀλλʼ ἤδη κύων. &  \\
\textit{Δι.} & Ἔμπουσα τοίνυν ἐστί. &  \\
\textit{Ξα.} & \hspace{9em}πυρὶ γοῦν λάμπεται &  \\
& ἅπαν τὸ πρόσωπον. &  \\
\textit{Δι.} & \hspace{8.5em}καὶ σκέλος χαλκοῦν ἔχει; &  \\
\textit{Ξα.} & νὴ τὸν Ποσειδῶ, καὶ βολίτινον θάτερον, & 295 \\

\end{tabularx}
\end{spacing}

\newpage
\begin{multicols}{2}
\small % roughly 9pt
\vocabentry{ἀγώνισμα, ματος, τό}{conflict, contest, battle; feat, achievement}
\vocabentry{αἰσθάνομαι}{to perceive; take notice of, have perception of (+ gen.)}
\vocabentry{βολίτινος, α, ον}{of cow-dung (< βόλιτον cow-dung)}
\vocabentry{γαῦρος, ον}{exulting in; (abs.) haughty, disdainful}
\vocabentry{Ἔμπουσα, ἡ}{Empusa, a fearful monster.}
\vocabentry{ἐν-τυγχάνω}{to light upon, fall in with, meet with (+ dat.)}
\vocabentry{ἐξ-όπισθε(ν)}{(adv.) behind, in rear}
\vocabentry{θηρίον, τό}{a wild animal, beast}
\vocabentry{λάμπω}{(act/mid) to give light, shine}
\vocabentry{μάχιμος, α, ον}{fit for battle, warlike}
\vocabentry{ὀρεύς, ὁ}{a mule}
\vocabentry{παντοδαπός, ή, όν}{of every kind, of all sorts, manifold}
\vocabentry{πρόσθεν}{(adv.) before, in front of}
\vocabentry{πρόσωπον, τό}{the face, visage, countenance}
\vocabentry{πῦρ, πυρός, τὸ}{fire}
\vocabentry{σκέλος, εος, τό}{the leg}
\vocabentry{τοτέ}{(adv.) at times, now and then}
\vocabentry{φιλο-τιμέομαι}{to love or seek after honour; to be ambitious, emulous}
\vocabentry{χάλχεος, α, ον}{brazen, made of bronze}
\vocabentry{ψόφος, ὁ}{a sound, noise}
\vocabentry{ὡραῖος, α, ον}{seasonable, youthful, beautiful}
\end{multicols}

% vertical line across page separating vocab and notes
\vspace{-1.5em}
\noindent\rule{\linewidth}{0.4pt}
\vspace{-2em}

\begin{multicols}{2}
\begin{parcolumns}[colwidths={1=1.5em, 2=0.9\linewidth}]{2}
\small

% put notes here
\glossline{281}{%
 \glossmix{εἰδώς}{Pple. οἶδα. Note perfect in form but present in meaning, like οἶδα. Verbs of knowing and showing regularly use a participle for indirect discourse, Smyth 2106-9.}\\
 \glossmix{φιλοτιμούμενος}{For D's belief that he is in martial competition with Heracles cf. ll. 40-1.}
}

\glossline{282}{%
 \glossmix{οὐδὲν γὰρ οὕτω γαῦρόν ἐσθʼ ὡς Ἡρακλῆς.}{\textit{For nothing is as boastful as Heracles}. Σ claim from Eur's \textit{Philoctetes}: Παρὰ τὰ ἐκ Φιλοκτήτου Εὐριπίδου: "οὐδὲν γὰρ οὕτω γαῦρον ὡς ἀνὴρ ἔφυ." 283-4 are also tragic parody in style.}
}

\glossline{283}{%
 \glossmix{εὐξαίμην ἂν}{potl. opt. εὔχομαι}
}

\glossline{284}{%
 \glossmix{ἀγώνισμʼ}{ἀγώνισμα}\\
 \glossmix{τῆς ὁδοῦ}{Prob. with ἀγώνισμʼ ἄξιόν τι, \textit{some achievement worthy of the journey}, or possibly \textit{on the road}, cf. Smyth 1448.}
}

\glossline{285}{%
 \glossmix{καὶ μὴν}{"(7) ... calling attention to something just seen or heard. 'See!': 'Hark!'" (GP 356).}
}

\glossline{286}{%
 \glossmix{ἴθι}{Impv. εἶμι.}
}

\glossline{289}{%
 \glossmix{γοῦν}{\textit{at any rate} (GP 450)}\\
 \glossmix{παντοδαπὸν... γίγνεται}{\textit{it becomes every kind [of thing]}}
}

\glossline{290}
 \glossmix{φέρʼ ἐπʼ αὐτὴν ἴω.}{Impv. φέρε can come before a 1st person subjunctive, functioning as a command (\textit{c'mon!}). C.f. LSJ s.v. IX.2}\\
 \glossmix{ἴω}{Subj. εἶμι. Hortatory. \textit{Come, let me go after her}. D. is interested in the attractive lady.}
 }

\glossline{293}{%
 \glossmix{Ἔμπουσα}{“Empousa is a creature of the popular imagination, a fearful monster. Despite the ability to change shape, Empousa seems to have been regarded as primarily female, usually a hag-like creature. Alciphron (3.26.3) refers to her as an ἐπιτύμβιος γραῦς, and at \textit{Eccl}. 1056f the young man says that the second old man must be some sort of Empousa: … Ἔμπουσά τις, | ἐξ αἵματος φλύκταιναν ἠμφιεσμένη” (Brown 1991: 42).}
}

\glossline{295}{%
 \glossmix{θάτερον}{τὸ ἕτερον [σκέλος], \textit{the other}}
}
\end{parcolumns}
\end{multicols}

\newpage
\begin{spacing}{1.5}
\begin{tabularx}{\textwidth}{@{}lXr@{}}
& σάφʼ ἴσθι. & \\
\textit{Δι.} & \hspace{4.5em}ποῖ δῆτʼ ἂν τραποίμην; &  \\
\textit{Ξα.} & \hspace{15em}ποῖ δʼ ἐγώ; &  \\
\textit{Δι.} & ἱερεῦ διαφύλαξόν μʼ, ἵνʼ ὦ σοι ξυμπότης. &  \\
\textit{Ξα.} & ἀπολούμεθʼ ὦναξ Ἡράκλεις. &  \\
\textit{Δι.} & \hspace{12em}οὐ μὴ καλεῖς μʼ &  \\
& ὦνθρωφʼ, ἱκετεύω, μηδὲ κατερεῖς τοὔνομα. &  \\
\textit{Ξα.} & Διόνυσε τοίνυν. & \\
\textit{Δι.} & \hspace{7em}τοῦτό γʼ ἧττον θατέρου. & 300 \\
\textit{Ξα.} & ἴθʼ ᾗπερ ἔρχει. δεῦρο δεῦρʼ ὦ δέσποτα. &  \\
\textit{Δι.} & τί δʼ ἔστι; &  \\
\textit{Ξα.} & \hspace{4.5em}θάρρει· πάντʼ ἀγαθὰ πεπράγαμεν, &  \\
& ἔξεστί θʼ ὥσπερ Ἡγέλοχος ἡμῖν λέγειν, &  \\
& "ἐκ κυμάτων γὰρ αὖθις αὖ γαλῆν ὁρῶ." &  \\
& ἥμπουσα φρούδη. & \\
\textit{Δι.} & \hspace{8em}κατόμοσον. & \\
\textit{Ξα.} & \hspace{13.5em}νὴ τὸν Δία. & 305 \\
\textit{Δι.} & καὖθις κατόμοσον. &  \\
\textit{Ξα.} & \hspace{8em}νὴ Δίʼ. &  \\
\textit{Δι.} & \hspace{11em}ὄμοσον. &  \\
\textit{Ξα.} & \hspace{15em}νὴ Δία. &  \\


\end{tabularx}
\end{spacing}

\newpage
\begin{multicols}{2}
\small % roughly 9pt
\vocabentry{ἄναξ, ὁ}{a lord, master}
\vocabentry{γαλέη (contr. γαλῆ), ἡ}{any kind of weasel, weasel, ferret}
\vocabentry{δια-φυλάσσω}{to watch closely, guard carefully}
\vocabentry{ἔξ-εστι}{it is allowed, it is in one's power, is possible}
\vocabentry{Ἔμπουσα, ἡ}{Empusa, a hobgoblin}
\vocabentry{ᾗπερ}{on which way/path, on the road which}
\vocabentry{Ἡγέλοχος, ὁ}{Hegelochus}
\vocabentry{θαρσέω}{to be of good courage, take courage}
\vocabentry{ἱερεύς, ὁ}{a priest, sacrificer}
\vocabentry{κατ-ερέω}{(fut) to speak against; to declare, tell plainly}
\vocabentry{κατ-όμνυμι}{to confirm by oath}
\vocabentry{κῦμα, ματος, τό}{wave, flood}
\vocabentry{ὄνομα, ματος, τό}{name}
\vocabentry{συμπότης, ὁ (Att. ξυμ-)}{symposiast (participant in symposium), a fellow-drinker, boon-companion}
\vocabentry{τρέπω}{to turn; (mid.) to betake oneself, go}
\vocabentry{φροῦδος, η, ον}{gone away, clean gone}
\end{multicols}

% vertical line across page separating vocab and notes
\vspace{-1.5em}
\noindent\rule{\linewidth}{0.4pt}
\vspace{-2em}

\begin{multicols}{2}
\begin{parcolumns}[colwidths={1=1.5em, 2=0.9\linewidth}]{2}
\small

% put notes here
\glossline{296}{%
 \glossmix{σάφʼ}{σάφα, adverbial.}\\
 \glossmix{ἴσθι}{Impv. οἶδα.}\\
 \glossmix{ἂν τραποίμην}{Aor. med. potential optative from τρέπω.}
}

\glossline{297}{%
 \glossmix{ἱερεῦ}{vocative of ἱερεῦς, referring to the priest of Dionysus Elethereus sitting in the front row (Csapo and Slater 1995: 289). Some humor in the god begging his own priest for help.}\\
 \glossmix{διαφύλαξόν}{Aor. impv. διαφυλάσσω.}\\
 \glossmix{ὦ}{1s subjunctive εἰμί, in purpose clause after ἵν(α). Perhaps Dionysus and his priest will attend a symposium after the drama (e.g. Agathon's victory is the occasion of Plato's \textit{Symp}.).}
}

\glossline{298}{%
 \glossmix{ἀπολούμεθʼ}{ἀπολούμεθα, fut. mid. ἀπόλλυμι.}\\
 \glossmix{ὦναξ}{ὦ ἄναξ. Xanthias turns to Dionysus (still partially disguised as Heracles, cf. ll. 45-6).}\\
 \glossmix{οὐ μὴ καλεῖς}{See note on l. 202}
}

\glossline{299}{%
 \glossmix{ὦνθρωφʼ}{ὦ ἄνθρωπε}\\
 \glossmix{ἱκετεύω}{Parenthetical. \textit{I beg you}, or even just \textit{please}.}\\
}

\glossline{300}{%
 \glossmix{θατέρου}{τοῦ ἑτέρου, gen. of comparison. \textit{that's even inferior (worse) than the other [name]}. D. doesn't want his disguise punctured and his true identity revealed.}
}

\glossline{301}{%
 \glossmix{ἴθʼ ᾗπερ ἔρχει}{ἴθι, impv. εἶμι. \textit{go on the way you're going}. Used to dismiss an evil spirit or divinity (apotropaic), cf. \textit{Lys.} 834 Ἴθ' ὀρθὴν ἥνπερ ἔρχει τὴν ὁδόν.}
}

\glossline{302}{%
 \glossmix{πεπράγαμεν}{Pf. πράσσω. This verb frequently means 'to fare' with neuter pronoun or adj (LSJ II)}
}

\glossline{303}{%
 \glossmix{ἔξεστί θʼ ὥσπερ Ἡγέλοχος ἡμῖν λέγειν}{\textit{Just like Hegelochus we can say}. Three years earlier (408), while performing Eur. \textit{Orestes}, the actor Hegelochus had mispronounced γαλήν' (calm things) as γαλῆν (ferret). Also parodied by Sannyrion and Strattis}
}

\glossline{305}{%
 \glossmix{ἥμπουσα}{ἡ Ἔμπουσα}\\
 \glossmix{κατόμοσον}{Aor. impv. κατόμνυμι}
}

\end{parcolumns}
\end{multicols}

\newpage

\begin{spacing}{1.5}
\begin{tabularx}{\textwidth}{@{}lXr@{}}
\textit{Δι.} & οἴμοι τάλας, ὡς ὠχρίασʼ αὐτὴν ἰδών. & \\
\textit{Ξα.} & ὁδὶ δὲ δείσας ὑπερεπυρρίασέ σου. &  \\
\textit{Δι.} & οἴμοι, πόθεν μοι τὰ κακὰ ταυτὶ προσέπεσεν; &  \\
& τίνʼ αἰτιάσομαι θεῶν μ' ἀπολλύναι; & 310 \\
\textit{Ξα.} & αἰθέρα Διὸς δωμάτιον ἢ χρόνου πόδα;  & \small{\shortstack{αὐλεῖ τις\\ἔνδοθεν}} \\
\textit{Δι.} & οὗτος. & \\
\textit{Ξα.} & \hspace{3em}τί ἔστιν; &  \\
\textit{Δι.} & \hspace{7em}οὐ κατήκουσας; &  \\
\textit{Ξα.} & \hspace{14em}τίνος; &  \\
\textit{Δι.} & αὐλῶν πνοῆς. &  \\
\textit{Ξα.} & \hspace{6em}ἔγωγε, καὶ δᾴδων γέ με &  \\
& αὔρα τις εἰσέπνευσε μυστικωτάτη. &  \\
\textit{Δι.} & ἀλλʼ ἠρεμὶ πτήξαντες ἀκροασώμεθα. & 315 \\
\textit{Χο.} & Ἴακχʼ ὦ Ἴακχε. &  \\
& Ἴακχʼ ὦ Ἴακχε. &  \\
\textit{Ξα.} & τοῦτʼ ἔστʼ ἐκεῖνʼ, ὦ δέσποθʼ· οἱ μεμυημένοι &  \\
& ἐνταῦθά που παίζουσιν, οὓς ἔφραζε νῷν. &  \\
& ᾄδουσι γοῦν τὸν Ἴακχον ὅνπερ δι' ἀγορᾶς. & 320 \\
\textit{Δι.} & κἀμοὶ δοκοῦσιν. ἡσυχίαν τοίνυν ἄγειν &  \\
& βέλτιστόν ἐσθʼ, ἕως ἂν εἰδῶμεν σαφῶς. &  \\

\end{tabularx}
\end{spacing}

\newpage
\begin{multicols}{2}
\small % roughly 9pt
\vocabentry{ἀείδω (Att. ᾄδω)}{to sing}
\vocabentry{ἀκροάζομαι}{to listen}
\vocabentry{αἰτιάομαι}{to accuse, censure (+ acc. & inf.)}
\vocabentry{αὔρα, ἡ}{air in motion, a breeze}
\vocabentry{αὐλέω}{to play on the aulos}
\vocabentry{αὐλός, ὁ}{aulos (wind instrument like oboe)}
\vocabentry{βέλτιστος, α, ον}{best}
\vocabentry{δαΐς, δαΐδος, (Att. contr. δᾴς, δᾳδός), ἡ}{firebrand, torch}
\vocabentry{δωμάτιον, τό}{a room, bedroom}
\vocabentry{ἔνδοθεν}{(adv.) from within}
\vocabentry{ἕως}{until; as long as}
\vocabentry{εἰσ-πνέω}{to breathe upon (+ acc.)}
\vocabentry{ἠρεμί}{= ἠρέμα (adv.) gently, softly}
\vocabentry{ἡσυχία, ἡ}{stillness, rest, quiet}
\vocabentry{Ἴακχος, ὁ}{Iacchos}
\vocabentry{κατ-ακούω}{to hear and obey; to hear plainly (+ acc. or gen.)}
\vocabentry{μυέω}{to initiate into the mysteries}
\vocabentry{μυστικός, ή, όν}{connected with the mysteries, mystical}
\vocabentry{πνοή, ἡ}{a blowing, blast, breeze; (LSJ IV) breath (of a wind instrument)}
\vocabentry{πόθεν}{(adv.) whence? (i.e., from where?)}
\vocabentry{προσ-πίπτω}{to fall upon (+ dat.)}
\vocabentry{πτήσσω}{to scare, alarm; (intr.) crouch or cower for fear}
\vocabentry{τάλας, τάλαινα, τάλαν}{suffering, wretched}
\vocabentry{ὑπερ-πυρριάω}{to grow orange-red, tawny for (+ gen.) < πυρρός orange-red, tawny}
\vocabentry{ὠχριάω}{to be pallid}
\end{multicols}

% vertical line across page separating vocab and notes
\vspace{-1.5em}
\noindent\rule{\linewidth}{0.4pt}
\vspace{-2em}

\begin{multicols}{2}
\begin{parcolumns}[colwidths={1=1.5em, 2=0.9\linewidth}]{2}
\small

\glossline{307}{%
 \glossmix{ὠχρίασʼ}{ὠχρίασα, aor. ὠχριάω.}
}

\glossline{308}{%
 \glossmix{ὁδὶ}{ὁδὲ + deictic iota. Antecedent unclear. Probably here κροκωτός saffron robe (cf. l. 49), or πρωκτός ass. Xanthias responds to D's claim that he's gone white; actually he's shit his pants.}\\
 \glossmix{σου}{\textit{for you}, dependent on the prefix ὑπερ}
}

\glossline{310}{%
 \glossmix{τίνʼ αἰτιάσομαι θεῶν μ' ἀπολλύναι}{\textit{Which of the gods will I accuse of destroying me?} This line and the previous tragic parody.}
}

\glossline{311}{%
 \glossmix{αἰθέρα Διὸς δωμάτιον ἢ χρόνου πόδα;}{Callback to l. 100.}
}

\glossline{312}{%
 \glossmix{οὗτος}{\textit{That.} D. hears an \textit{aulos}.  Most MSS have the stage direction "someone plays the \textit{aulos} from inside," i.e. from within the stage building.}
}

\glossline{314}{%
 \glossmix{εἰσέπνευσε}{Aor. εἰσπνέω.}
}

\glossline{316}{%
 \glossmix{Ἴακχε}{"One of the deities of the Mysteries of  Eleusis... I. is the personification of the ecstatic cultic cry (íakchos, onomatopoetic) by the participants in the Mysteries during their procession from Athens to the Eleusinian sanctuary where they underwent initiation into the mysteries (Hdt. 8,65)... His image, which was kept in a temple of Demeter, Kore and I. by the Pompeion at the Sacred Gate (Paus. 1,2,4...), was carried ahead of this procession by the \textit{iakchagōgós} (‘leader of I.’)... His attribute is the torch in the light of which participants arrived at Eleusis (Aristoph. Ran. 340-353, cf. Paus. 1,2,4) and ecstatic dance is his characteristic (Aristoph. Ran. 316-353; Str. 10,3,10)... Since Soph. Ant. 1152 and Eur. Ion 1074-1077 I. has therefore been identified in the literature with  Dionysus..." (New Pauly s.v. Iacchus). As one can see from the references, this choral song in the \textit{Frogs} is one of the few extended descriptions of the god.}
}

\glossline{318}{%
 \glossmix{τοῦτʼ ἔστʼ ἐκεῖνʼ}{\textit{This is it}, i.e. what Heracles said.}
}

\glossline{319}{%
 \glossmix{νῷν}{Cf. 277. Subj. Heracles like 279.}
}

\glossline{320}{%
 \glossmix{δι' ἀγορᾶς}{Presumably the initiates in the Eleusinian Mysteries passed through the Athenian Agora on their way to Eleusis. So the Σ: Ἴακχον λέγειν, ὃν ᾄδουσιν ἐξ ἄστεως διὰ τῆς ἀγορᾶς ἐξιόντες εἰς Ἐλευσῖνα. Alternatively read Διάγορας, a lyric poet and atheist (\textit{Birds} 1073)}
}

\glossline{321}{%
 \glossmix{ἡσυχίαν... ἄγειν}{\textit{keep quiet}, LSJ s.v. ἡσυχία 4.a.}\\
 \glossmix{κἀμοὶ}{καὶ ἐμοὶ}
}

\glossline{322}{%
 \glossmix{ἐσθʼ}{ἐστι}\\
 \glossmix{ἕως ἂν εἰδῶμεν}{\textit{until we know}, subj. οἶδα, cf. 266n.}
}

\end{parcolumns}
\end{multicols}

\newpage
\begin{spacing}{1.5}
\begin{tabularx}{\textwidth}{@{}lXr@{}}
\textit{Χο.} & Ἴακχʼ ὦ πολυτίμητʼ ἐν ἕδραις ἐνθάδε ναίων, &  \\
& Ἴακχʼ ὦ Ἴακχε, & 325 \\
& ἐλθὲ τόνδʼ ἀνὰ λειμῶνα χορεύσων &  \\
& ὁσίους ἐς θιασώτας, &  \\
& πολύκαρπον μὲν τινάσσων &  \\
& περὶ κρατὶ σῷ βρύοντα &  \\
& στέφανον μύρτων, θρασεῖ δʼ ἐγκατακρούων & 330 \\
& ποδὶ τὰν ἀκόλαστον &  \\
& φιλοπαίγμονα τιμάν, &  \\
& χαρίτων πλεῖστον ἔχουσαν μέρος, ἁγνάν, ἱερὰν & 335 \\
& ὁσίοις μύσταις χορείαν. &  \\
\textit{Ξα.} & ὦ πότνια πολυτίμητε Δήμητρος κόρη, &  \\
& ὡς ἡδύ μοι προσέπνευσε χοιρείων κρεῶν. &  \\
\textit{Δι.} & οὔκουν ἀτρέμʼ ἕξεις, ἤν τι καὶ χορδῆς λάβῃς; &  \\
\textit{Χο.} & ἐγείρων φλογέας λαμπάδας ἐν χερσὶ προσήκεις, & 340 \\
& Ἴακχʼ ὦ Ἴακχε, &  \\
& νυκτέρου τελετῆς φωσφόρος ἀστήρ. &  \\
& φλογὶ φέγγεται δὲ λειμών· &  \\
& γόνυ πάλλεται γερόντων· & 345 \\

\end{tabularx}
\end{spacing}

\newpage
\begin{multicols}{2}
\small % roughly 9pt
\vocabentry{ἀ-κόλαστος, ον}{undisciplined, unbridled; licentious}
\vocabentry{ἀστήρ, ος, ὁ}{star}
\vocabentry{ἀτρέμα(ς)}{(adv.) without trembling, without motion}
\vocabentry{ἁγνός, ή, όν}{pure, chaste, holy}
\vocabentry{βρύω}{to be full to bursting; (+ gen.) to be full of}
\vocabentry{γέρων, ὁ}{an old man (in apposition as adj., old)}
\vocabentry{γόνυ, τό}{the knee}
\vocabentry{ἐγ-κατα-κρούω}{hapax < κρούω strike, smite. "χορείαν τοῖς μύσταις tread (a measure) among (them)" (LSJ)}
\vocabentry{ἕδρα, ἡ}{a sitting-place; seat, abode, (freq. in pl., esp. of the gods) sanctuary, temple}
\vocabentry{ἡδύς, εῖα, ύ}{sweet}
\vocabentry{θιασώτης, ὁ}{member of a θίασος, reveler}
\vocabentry{θρασύς, εῖα, ύ}{bold, spirited, courageous, confident}
\vocabentry{κόρη, ἡ}{a maiden, maid; pupil of the eye}
\vocabentry{(κράς), κρατός, ἡ}{the head; (poet. form of κάρα)}
\vocabentry{μύρτον, τό}{a myrtle-berry}
\vocabentry{μύστης, ὁ}{one initiated}
\vocabentry{ναίω}{to dwell, abide}
\vocabentry{νύκτερος, ον}{= νυκτερινός by night, nightly < νύξ}
\vocabentry{ὅσιος, α, ον}{hallowed, righteous, pious}
\vocabentry{πάλλω}{to poise, shake, sway, leap}
\vocabentry{πλεῖστος, α, ον}{most, largest}
\vocabentry{πότνια, ἡ}{mistress, queen}
\vocabentry{πολύ-καρπος, ον}{rich in fruit}
\vocabentry{πολυ-τίμητος, ον}{highly honored}
\vocabentry{προσ-ήκω}{to have come, be at hand, be present}
\vocabentry{προσ-πνέω}{to breathe upon, inspire; (intr.) to blow to or over (+ gen.)}
\vocabentry{στέφανος, ὁ}{wreath, crown}
\vocabentry{τελετή, ἡ}{rite, esp. initiation in the mysteries}
\vocabentry{τιμή, ἡ}{worship, esteem, honor}
\vocabentry{τινάσσω}{to shake}
\vocabentry{φέγγω}{= φαίνω to make bright; (intr./pass.) to shine}
\vocabentry{φιλο-παίγμων, ον}{fond of play, playful, sportive}
\vocabentry{φλόγεος, α, ον}{burning, flaming < φλόξ}
\vocabentry{φλόξ, φλογός, ἡ}{a flame}
\vocabentry{φωσ-φόρος, ον}{bringing or giving light}
\vocabentry{χάρις, ἡ}{gratitude, favor, grace, charm}
\vocabentry{χοίρειος, α, ον}{of a swine, pig}
\vocabentry{χορδή, ἡ}{guts; things made out of guts: string of gut, sausage}
\end{multicols}

% vertical line across page separating vocab and notes
\vspace{-1.5em}
\noindent\rule{\linewidth}{0.4pt}
\vspace{-2em}

\begin{multicols}{2}
\begin{parcolumns}[colwidths={1=1.5em, 2=0.9\linewidth}]{2}
\small

% put notes here
\glossline{324}{%
 \glossmix{ἐν ἕδραις ἐνθάδε ναίων}{\textit{dwelling here in [your] temple}, presumably referring to an Iaccheion in Athens (Plut. Arist. 27.3), or perhaps in Hades.}
}

\glossline{326}{%
 \glossmix{τόνδʼ ἀνὰ λειμῶνα}{\textit{up through this meadow}. ἀνά is the preposition. This word order (adj prep noun) common in poetry (Smyth 1664), as in the following line ὁσίους ἐς θιασώτας.}
 \glossmix{χορεύσων}{Fut. pple. expressing purpose after verb of motion.}
}

\glossline{327}{%
 \glossmix{ὁσίους ἐς θιασώτας}{\textit{to [your] pious revelers}}
}

\glossline{330}{%
 \glossmix{μύρτων}{with βρύοντα}\\
 \glossmix{ἐγκατακρούων... ὁσίοις μύσταις χορείαν}{\textit{striking a dance among (εγ-) the holy initiates}. τιμάν in apposition to χορείαν. Lyric long separation between the participle and its object.}
}

\glossline{334}{%
 \glossmix{μύρτων}{with βρύοντα. Hyperbaton.}
}

\glossline{337}{%
 \glossmix{Δήμητρος κόρη}{= Persephone, also called Kore.}
}

\glossline{338}{%
 \glossmix{προσέπνευσε... κρεῶν}{Intransitive. \textit{it blew, there was a breeze... of meats}. Initiates into the Eleusinian Mysteries sacrificed piglets, e.g. \textit{Peace} 374-5: εἰς χοιρίδιόν μοί νυν δάνεισον τρεῖς δραχμάς· / δεῖ γὰρ μυηθῆναί με πρὶν τεθνηκέναι. \textit{Then lend me three drachmas for a piglet; I’ve got to get initiated before I die. }}
}

\glossline{339}{%
 \glossmix{οὔκουν ἀτρέμʼ ἕξεις}{\textit{Won't you hold still!} For οὔκουν cf. 193. See LSJ s.v. ἔχω B ("hold oneself, i.e. keep... ἔχε ἠρέμα keep still, Pl. Cra. 399e, etc.").}\\
 \glossmix{ἤν τι καὶ χορδῆς λάβῃς}{\textit{in the hope that you also get some bit of sausage} [in addition to the smell]. Double entendre, piglet = vagina, sausage = penis.}
}

\glossline{340}{%
 \glossmix{ἐγείρων φλογέας λαμπάδας}{\textit{rousing burning torches}, cf. LSJ s.v. ἐγείρω I.2}
}

\glossline{343}{%
 \glossmix{ἀστήρ}{Nom. in apposition to vocative, Smyth 1287.}
}

\end{parcolumns}
\end{multicols}

\newpage
\begin{spacing}{1.5}
\begin{tabularx}{\textwidth}{@{}lXr@{}}
& ἀποσείονται δὲ λύπας &  \\
& χρονίους τʼ ἐτῶν παλαιῶν ἐνιαυτοὺς & 347-8 \\
& ἱερᾶς ὑπὸ τιμᾶς. & \\
& σὺ δὲ λαμπάδι φέγγων & 350 \\
& προβάδην ἔξαγʼ ἐπʼ ἀνθηρὸν ἕλειον δάπεδον & 351-2 \\
& χοροποιὸν, μάκαρ, ἥβαν. &  \\
\textit{Χο.} & εὐφημεῖν χρὴ κἀξίστασθαι τοῖς ἡμετέροισι  & \\
& \hspace{3em}χοροῖσιν, &\\
& ὅστις ἄπειρος τοιῶνδε λόγων ἢ γνώμῃ μὴ &\\
& \hspace{3em}καθαρεύει, & 355 \\
& ἢ γενναίων ὄργια Μουσῶν μήτʼ εἶδεν μήτʼ &  \\
& \hspace{3em}ἐχόρευσεν, & \\
& μηδὲ Κρατίνου τοῦ ταυροφάγου γλώττης Βακχεῖʼ &  \\
&\hspace{3em}ἐτελέσθη, & \\
& ἢ βωμολόχοις ἔπεσιν χαίρει μὴ ʼν καιρῷ τοῦτο &  \\
&\hspace{3em}ποιοῦσιν, & \\
& ἢ στάσιν ἐχθρὰν μὴ καταλύει μηδʼ εὔκολός ἐστι &  \\
&\hspace{3em} πολίταις, &\\ 
& ἀλλʼ ἀνεγείρει καὶ ῥιπίζει κερδῶν ἰδίων & \\
& \hspace{3em}ἐπιθυμῶν, & 360 \\

\end{tabularx}
\end{spacing}

\newpage
\begin{multicols}{2}
\small % roughly 9pt
\vocabentry{ἄ-πειρος, ον}{ignorant of, unexperienced in (+ gen.)}
\vocabentry{ἀν-εγείρω}{to wake up, rouse, excite}
\vocabentry{ἀνθηρός, ή, όν}{flowering, blooming}
\vocabentry{ἀπο-σείω}{to shake off}
\vocabentry{Βακχεῖον, τό}{Bacchic revelry, Bacchic rites}
\vocabentry{βωμολόχος, ον}{coarse, crude, ribald}
\vocabentry{γενναῖος, α, ον}{high-born, noble; high-minded}
\vocabentry{δάπεδον, τό}{level surface; ground, soil}
\vocabentry{ἕλειος, α, ον}{of the marsh < ἕλος marsh}
\vocabentry{ἐνιαυτός, ὁ}{year}
\vocabentry{ἐξ-άγω}{lead on, lead away, excite}
\vocabentry{ἐξ-ίστημι}{to stand out of; (mid.) stand aside from, stand out of the way of (+ gen./dat.)}
\vocabentry{ἔτος, εος, τό}{year}
\vocabentry{εὔ-κολος, ον}{easily satisfied, contented, good-natured}
\vocabentry{εὐ-φημέω}{to use words of good omen; keep a religious silence}
\vocabentry{ἐχθρός, ή, όν}{hated, hateful; enemy}
\vocabentry{ἥβη, ἡ}{manhood, youthful prime, youth}
\vocabentry{ἴδιος, α, ον}{one's own, pertaining to oneself, private}
\vocabentry{καθαρεύω}{to be pure, clean}
\vocabentry{καιρός, ὁ}{time; the right moment, opportunity}
\vocabentry{κατα-λύω}{to put down; to bring to an end, resolve, end (war, disputes, etc.)}
\vocabentry{κέρδος, εος, τό}{gain, profit, advantage}
\vocabentry{Κρατῖνος, ὁ}{Cratinus, "the most important comic dramatist of the generation before Aristophanes... his \textit{Pytine} defeated \textit{Clouds}" (Dover)}
\vocabentry{λύπη, ἡ}{pain (of body or mind), grief}
\vocabentry{μάκαρ, αρος, ὁ}{blessed, happy}
\vocabentry{ὄργια, τό}{rites, mysteries}
\vocabentry{παλαιός, ή, όν}{old in years}
\vocabentry{πολίτης, ου, ὁ}{citizen}
\vocabentry{προβάδην}{(adv.) as one walks; onward < προβαίνω}
\vocabentry{ῥιπίζω}{to fan (a flame) < ῥιπίς fan for raising the fire}
\vocabentry{στάσις, εως, ἡ}{placing, setting; faction, sedition, discord, civil war}
\vocabentry{ταυρο-φάγος, ον}{bull-eating}
\vocabentry{τελέω}{to fulfill, accomplish; to initiate (in the mysteries); (pass.) to have oneself initiated (+ acc. into)}
\vocabentry{τιμή, ἡ}{worship, esteem, honor}
\vocabentry{φέγγω}{= φαίνω to make bright; (intr./pass.) to shine}
\vocabentry{χορο-ποιός, όν}{instituting or arranging a chorus; leading the dance}
\vocabentry{χρόνιος, α, ον}{after a long time, late; for a long time, long, long-continued}
\end{multicols}

% vertical line across page separating vocab and notes
\vspace{-1.5em}
\noindent\rule{\linewidth}{0.4pt}
\vspace{-2em}

\begin{multicols}{2}
\begin{parcolumns}[colwidths={1=1.5em, 2=0.9\linewidth}]{2}
\small

% put notes here
\glossline{346}{%
 \glossmix{ἀποσείονται}{Mid. \textit{They shake off (from themselves)}}
}

\glossline{347-8}{%
 \glossmix{ἐτῶν παλαιῶν}{Redundant and virtually synonymous with χρονίους... ἐνιαυτοὺς. Possibly ἐνιαυτός here means 'cycle' (LSJ I) and hence \textit{[they shake off] the long cycles of ancient years}}
}

\glossline{349}{%
 \glossmix{ὑπὸ}{\textit{through, by}, LSJ A.II}
}

\glossline{353}{%
 \glossmix{χοροποιὸν... ἥβαν}{Obj. of ἔξαγ(ε). \textit{Lead forward [our] youth...}}\\
 \glossmix{μάκαρ}{Voc.}
}

\glossline{354-371}{%
 \glossmix{}{These lines are written in anapests and resemble a parabasis, a direct address by the chorus to the audience.}
}

\glossline{354}{%
 \glossmix{κἀξίστασθαι}{= καὶ ἐξίστασθαι. The subject of both infinitives is the person referred to in the ὅστις clause following.} \\
 \glossmix{ἢ}{\textit{either}, followed by a series of 'or's (ἤ) and 'nor's (μήτ’, μηδέ). (ll. 356-359).} \\
 \glossmix{μὴ... μήτʼ... μηδὲ... μὴ... μηδʼ... }{μή + indic. used in a relative clause with conditional force \textit{whoever...} = \textit{if he ever...}}
}

\glossline{356}{%
 \glossmix{εἶδεν}{Aor. ὁράω}
}

\glossline{357}{%
 \glossmix{μηδὲ Κρατίνου τοῦ ταυροφάγου γλώττης Βακχεῖʼ
ἐτελέσθη}{\textit{nor was initiated into the Bacchic rites of the tongue of bull-eating Kratinos}. I.e. did not participate in (observe? perform?) the comedy of Kratinos. According to Σ ταυροφάγου was an epithet of Dion. in Sophocles: εἴρηται δὲ παρὰ τὸ Σοφοκλέους ἐκ Τυροῦς «Διονύσου τοῦ ταυροφάγου.» }
}

\glossline{358}{%
 \glossmix{μὴ ʼν καιρῷ τοῦτο ποιοῦσιν}{Dat. pl. pple., not finite verb. \textit{[words], not doing it (i.e. being ribald) at the right time}, i.e. \textit{[words] if they are inappropriately timed}.}
}

\glossline{360}{%
 \glossmix{ἀνεγείρει καὶ ῥιπίζει}{Sc. στάσιν as obj.}
 \glossmix{ἐπιθυμῶν}{verb takes genitive, here with κερδῶν ἰδίων}
}

\end{parcolumns}
\end{multicols}

\newpage
\begin{spacing}{1.5}
\begin{tabularx}{\textwidth}{@{}lXr@{}}
& ἢ τῆς πόλεως χειμαζομένης ἄρχων καταδωροδοκεῖ- & \\
& \hspace{3em}ται, &  \\
& ἢ προδίδωσιν φρούριον ἢ ναῦς, ἢ τἀπόρρητʼ ἀπο- & \\
& \hspace{3em}πέμπει &  \\
& ἐξ Αἰγίνης Θωρυκίων ὢν εἰκοστολόγος κακο- & \\
&\hspace{3em}δαίμων, &  \\
& ἀσκώματα καὶ λίνα καὶ πίτταν διαπέμπων εἰς & \\ 
&\hspace{3em}Ἐπίδαυρον, &  \\
& ἢ χρήματα ταῖς τῶν ἀντιπάλων ναυσὶν παρέχειν & \\ 
&\hspace{3em}τινὰ πείθει, & 365 \\
& ἢ κατατιλᾷ τῶν Ἑκαταίων κυκλίοισι χοροῖσιν ὑπ- & \\
&\hspace{3em}ᾴδων, &  \\
& ἢ τοὺς μισθοὺς τῶν ποιητῶν ῥήτωρ ὢν εἶτʼ ἀπο- & \\
&\hspace{3em}τρώγει, &  \\
& κωμῳδηθεὶς ἐν ταῖς πατρίοις τελεταῖς ταῖς τοῦ & \\ 
&\hspace{3em}Διονύσου· &  \\
& τούτοις ἀπαυδῶ καὖθις ἀπαυδῶ καὖθις τὸ τρίτον μάλʼ & \\ 
&\hspace{3em}ἀπαυδῶ &  \\
& ἐξίστασθαι μύσταισι χοροῖς· ὑμεῖς δʼ ἀνεγείρετε & \\ 
&\hspace{3em}μολπὴν & 370 \\
& καὶ παννυχίδας τὰς ἡμετέρας αἳ τῇδε πρέπουσιν & \\ 
&\hspace{3em}ἑορτῇ. &  \\

\end{tabularx}
\end{spacing}

\newpage
\begin{multicols}{2}
\small % roughly 9pt
\vocabentry{Αἴγινα, ἡ}{Aegina, island in the Saronic Gulf between Athens and Epidaurus}
\vocabentry{ἀν-εγείρω}{to wake up, rouse, excite}
\vocabentry{ἀντί-παλος, ον}{rival, antagonist}
\vocabentry{ἀπ-αυδάω}{to forbid}
\vocabentry{ἀπο-πέμπω}{to send off}
\vocabentry{ἀπό-ρρητος, ον}{forbidden; (pl.) forbidden (exports), contraband}
\vocabentry{ἀπο-τρώγω}{to bite, nibble at, eat up}
\vocabentry{ἄρχων, οντος, ὁ}{> ἄρχω, an archon, a chief magistrate of Athens}
\vocabentry{ἄσκωμα, ματος, τό}{the leather padding of the hole which served for the rowlock}
\vocabentry{αὐδάω}{call out, forbid}
\vocabentry{δια-πέμπω}{to send off in different directions, send to and fro, send about}
\vocabentry{ἐξ-ίστημι}{to stand out of; (mid.) stand aside from, stand out of the way of (+ gen./dat.)}
\vocabentry{Ἐπίδαυρον, ὁ}{Epidauros}
\vocabentry{Ἑκαταῖος, α, ον}{of Hecate, Hecatean}
\vocabentry{εἰκοστο-λόγος, ὁ}{tax-collector (one who collects the twentieth)}
\vocabentry{Θωρυκίων, οντος, ὁ}{}
\vocabentry{κατα-δωροδοκέω}{to accept bribes}
\vocabentry{κατα-τιλάω}{to shit on (+ gen.)}
\vocabentry{κύκλιος, α, ον}{round, circular}
\vocabentry{κωμῳδέω}{to represent in a comedy, to satirize}
\vocabentry{λίνον, τό}{anything made of flax; cord, line, net, sail-cloth}
\vocabentry{μισθός, ὁ}{wages, pay, hire}
\vocabentry{μολπή, ἡ}{the song and dance, a chant}
\vocabentry{μύστης, ὁ}{one initiated; (adj.) mystic, initiatory}
\vocabentry{πάτριος, α, ον}{of one’s father, ancestral, native}
\vocabentry{παν-νυχίς, ίδος, ἡ}{night-festival, vigil}
\vocabentry{πίσσα, ἡ (Att. πίττα)}{pitch, resin}
\vocabentry{πρέπω}{to befit (+ dat.) (πρέπει it is fitting)}
\vocabentry{προ-δίδωμι}{to give up, betray, forsake, abandon}
\vocabentry{ῥήτωρ, ορος, ὁ, ἡ}{οrator, public speaker, politician}
\vocabentry{τελετή, ἡ}{rite, esp. initiation in the mysteries}
\vocabentry{τρίτος, α, ον}{the third}
\vocabentry{ὑπ-ᾴδω}{to sing along to, accompany with the voice (+ dat.)}
\vocabentry{φρούριον, τό}{a watch-post, garrisoned fort, citadel < φρουρός guard}
\vocabentry{χειμάζω}{to pass the winter; to toss like a storm, (pass.) to be tempest-tossed, distressed (esp. of the state considered as a ship)}
\vocabentry{χρῆμα, ματος, τό}{thing, (pl.) goods, property, money}
\end{multicols}

% vertical line across page separating vocab and notes
\vspace{-1.5em}
\noindent\rule{\linewidth}{0.4pt}
\vspace{-2em}

\begin{multicols}{2}
\begin{parcolumns}[colwidths={1=1.5em, 2=0.9\linewidth}]{2}
\small

% put notes here
\glossline{361}{%
 \glossmix{τῆς πόλεως χειμαζομένης}{gen. abs., or with ἄρχων, or even possibly with καταδωροδοκεῖται (Dover)}
}

\glossline{362}{%
 \glossmix{ναῦς}{acc. pl.}
}

\glossline{363}{%
 \glossmix{Θωρυκίων}{Unknown person, maybe one of these tax collectors. Together with 383, these lines imply that Thorykion was a corrupt tax collector who participated in exporting contraband from Aegina (Athenian possession since 431) to Epidaurus (Spartan ally and opponent in the war, across the Saronic Gulf from Athens).}
}

\glossline{366}{%
 \glossmix{τῶν Ἑκαταίων}{probably statues of shrines of Hekate outside the house, cf. \textit{Wasps} 804. The joke is probably about Kinesias (cf. line 153), known for his circular dances (\textit{Birds} 1403) and for shitting in public (\textit{Eccl.} 329).}\\
 \glossmix{κατατιλᾷ τῶν Ἑκαταίων κυκλίοισι χοροῖσιν ὑπᾴδων}{per the vocab κατατιλᾷ takes a genitive and ὑπᾴδων a dative, i.e. \textit{shits on [shrines of] Hekate accompanying circular dances}}
}

\glossline{367}{%
 \glossmix{τοὺς μισθοὺς}{Presumably some proposal by a politician to reduce the compensation for poets at the dramatic festivals. There is no evidence for such a proposal outside Σ: τοῦτο εἰς Ἀρχῖνον. μήποτε δὲ καὶ εἰς Ἀγύρριον. μέμνηται δὲ τούτων καὶ Πλάτων ἐν Σκευαῖς καὶ Σαννυρίων ἐν Δανάῃ. οὗτοι γὰρ προϊστάμενοι τῆς δημοσίας τραπέζης τὸν μισθὸν τῶν κωμῳδῶν ἐμείωσαν κωμῳδηθέντες. ("this is about Archinus. And maybe also about Agyrrhios. They are mentioned both by Plato [the comedian] in Costumes and Sannyrion in Danae. For they had been mocked in comedy and reduced the pay for comic poets, although they claimed to be acting for the public bank.)}
}

\glossline{369}{%
 \glossmix{τούτοις ἀπαυδῶ...}{The meaning \textit{I forbid... to stand apart from} is exactly the opposite of what is needed, a discrepancy which has worried some commentators (e.g. Wilson 2007 emends this line to τούτοις αὐδῶ καὖθις ἐπαυδῶ καὖθις τὸ τρίτον μάλʼ ἐπαυδῶ).}
}

\end{parcolumns}
\end{multicols}

\newpage
\begin{spacing}{1.5}
\begin{tabularx}{\textwidth}{@{}lXr@{}}
\textit{Χο.} & χώρει νυν πᾶς ἀνδρείως &  \\
& ἐς τοὺς εὐανθεῖς κόλπους &  \\
& λειμώνων ἐγκρούων &  \\
& κἀπισκώπτων & 375 \\
& καὶ παίζων καὶ χλευάζων, &  \\
& ἠρίστηται δʼ ἐξαρκούντως. &  \\
\textit{Χο.} & ἀλλʼ ἔμβα χὤπως ἀρεῖς &  \\
& τὴν Σώτειραν γενναίως &  \\
& τῇ φωνῇ μολπάζων, & 380 \\
& ἣ τὴν χώραν &  \\
& σῴσειν φήσʼ ἐς τὰς ὥρας, &  \\
& κἂν Θωρυκίων μὴ βούληται. &  \\
\textit{Χο.} & ἄγε νυν ἑτέραν ὕμνων ἰδέαν τὴν καρποφόρον & \\
& \hspace{3em}βασίλειαν &  \\
& Δήμητρα θεὰν ἐπικοσμοῦντες ζαθέαις μολπαῖς & \\
& \hspace{3em }κελαδεῖτε. & 385 \\
\textit{Χο.} & Δήμητερ ἁγνῶν ὀργίων &  \\
& ἄνασσα συμπαραστάτει, &  \\
& καὶ σῷζε τὸν σαυτῆς χορόν, &  \\
& καί μʼ ἀσφαλῶς πανήμερον &  \\
& παῖσαί τε καὶ χορεῦσαι· & 390 \\


\end{tabularx}
\end{spacing}

\newpage
\begin{multicols}{2}
\small % roughly 9pt
\vocabentry{ἀείρω}{to lift, raise up; to praise, extol}
\vocabentry{ἄνασσα, ἡ}{a queen, lady, mistress}
\vocabentry{ἀριστάω}{to take breakfast}
\vocabentry{ἀσφαλῶς}{(adv.) firmly, steadily; safely, without faltering}
\vocabentry{ἁγνός, ή, όν}{full of religious awe}
\vocabentry{βασίλεια, ἡ}{a queen, princess}
\vocabentry{ἐγ-κρούω}{to knock or hammer in; to stomp on the ground, dance}
\vocabentry{ἐμ-βαίνω}{to step in}
\vocabentry{ἐξ-αρκούντως}{(adv.) enough, sufficiently}
\vocabentry{ἐπι-κοσμέω}{to add ornaments to, adorn; honor, celebrate}
\vocabentry{ἐπι-σκώπτω}{to laugh at, make fun of; (intr) to jest, make fun}
\vocabentry{εὐ-ανθής, ές}{blooming, budding}
\vocabentry{ζά-θεος, α, ον}{very divine, sacred}
\vocabentry{ἰδέα, ἡ}{form; kind, sort}
\vocabentry{καρπο-φόρος, ον}{fruit-bearing, fruitful}
\vocabentry{κελαδέω}{to sound as rushing water; to shout aloud; to sing of, celebrate loudly}
\vocabentry{κόλπος, ὁ}{bosom, lap; any bosom-like hollow; vale}
\vocabentry{μολπάζω}{to sing of}
\vocabentry{μολπή, ἡ}{the song and dance, a chant}
\vocabentry{ὄργια, τό}{rites, mysteries}
\vocabentry{παν-ήμερος, ον}{all day long; neut. as adv.}
\vocabentry{συμ-παρα-στατέω}{to stand by so as to assist}
\vocabentry{σώζω}{to save; to keep safe, preserve}
\vocabentry{Σώτειρα, ἡ}{the Savior}
\vocabentry{ὕμνος, ὁ}{a hymn, festive song}
\vocabentry{φωνή, ἡ}{a sound, tone}
\vocabentry{χλευάζω}{to joke, jest, scoff}
\vocabentry{χώρα, ἡ}{land}
\vocabentry{χωρέω}{to give way, withdraw; go forward, advance}
\vocabentry{ὥρα, ἡ}{time, season, climate}
\end{multicols}

% vertical line across page separating vocab and notes
\vspace{-1.5em}
\noindent\rule{\linewidth}{0.4pt}
\vspace{-2em}

\begin{multicols}{2}
\begin{parcolumns}[colwidths={1=1.5em, 2=0.9\linewidth}]{2}
\small

% put notes here
\glossline{372}{%
 \glossmix{χώρει... πᾶς}{\textit{Go forward... each [of you]}}
}

\glossline{374}{%
 \glossmix{ἐγκρούων}{Like the other following participles, masc. nom. sg. pple. agreeing with πᾶς}
}

\glossline{375}{%
 \glossmix{κἀπισκώπτων}{καὶ ἐπισκώπτων. Probably a reference to the jokes and mockery encountered on the way to Eleusis, cf. the \textit{Homeric Hymn to Demeter}.}
}

\glossline{377}{%
 \glossmix{ἠρίστηται δʼ ἐξαρκούντως}{3s pf passive of ἀριστάω, impersonal. \textit{And it has been breakfasted enough}, i.e. \textit{there has been enough of breakfast}.}
}

\glossline{378}{%
 \glossmix{ἔμβα}{Impv. "imper. βῆθι... βᾱ in compds. ἔμβα, κατάβα, etc." (LSJ s.v. βαίνω)}\\
 \glossmix{χὤπως ἀρεῖς}{καὶ ὅπως. \textit{and praise}. ἀρεῖς 2s fut. ἀείρω, and ὅπως + fut. ind. = imperative, "urgent exhortations and prohibitions" (Smyth 1920).}
}

\glossline{379}{%
 \glossmix{Σώτειραν}{perhaps Athena Soteira or Kore Soteira (i.e. Persephone), both attested in Attic cult (Dover).}
}

\glossline{381}{%
 \glossmix{ἣ}{Antecedent Σώτειραν. \textit{the Savior... who...}}
}

\glossline{382}{%
 \glossmix{φήσ'}{φήσι. \textit{who says that she will preserve...}}
 \glossmix{ἐς τὰς ὥρας}{\textit{for the seasons}, i.e. \textit{for all seasons}. Perhaps cf. Theoc. 15.74 κἠς ὥρας κἤπειτα \textit{next year and for ever}}
}

\glossline{383}{%
 \glossmix{κἂν}{καὶ ἄν. \textit{even if}}
}

\glossline{384}{%
 \glossmix{ἰδέαν}{Tricky, perhaps intentional ambiguity with the double accusatives resolved by a pple & verb. Maybe ἰδέαν with κελαδεῖτε and βασίλειαν with ἐπικοσμοῦντες, i.e. \textit{Sing another kind of hymns... celebrating the queen...}}
}

\glossline{387}{%
 \glossmix{συμπαραστάτει}{Sc. ἡμᾶς.}
}

\glossline{389}{%
 \glossmix{ἀσφαλῶς}{In the context of a dancing chorus, perhaps both literal (\textit{without tripping}) and metaphorical (\textit{in safety})}
}

\glossline{390}{%
 \glossmix{παῖσαί... χορεῦσαι}{aor. act. inf. Supply something like δὸς: \textit{[Grant that] I...} (Smyth 2013c, 2014).}
}

\end{parcolumns}
\end{multicols}

\newpage
\begin{spacing}{1.5}
\begin{tabularx}{\textwidth}{@{}lXr@{}}
\textit{Χο.} & καὶ πολλὰ μὲν γέλοιά μʼ εἰ- &  \\
& πεῖν, πολλὰ δὲ σπουδαῖα, καὶ &  \\
& τῆς σῆς ἑορτῆς ἀξίως &  \\
& παίσαντα καὶ σκώψαντα νι- &  \\
& κήσαντα ταινιοῦσθαι. & 395 \\
\textit{Χο.} & ἄγʼ εἶα &  \\
& νῦν καὶ τὸν ὡραῖον θεὸν παρακαλεῖτε δεῦρο &  \\
& ᾠδαῖσι, τὸν ξυνέμπορον τῆσδε τῆς χορείας. &  \\
\textit{Χο.} & Ἴακχε πολυτίμητε, μέλος ἑορτῆς &  \\
& ἥδιστον εὑρών, δεῦρο συνακολούθει & 400 \\
& πρὸς τὴν θεὸν &  \\
& καὶ δεῖξον ὡς ἄνευ πόνου &  \\
& πολλὴν ὁδὸν περαίνεις. &  \\
& Ἴακχε φιλοχορευτὰ συμπρόπεμπέ με. &  \\
\textit{Χο.} & σὺ γὰρ κατεσχίσω μὲν ἐπὶ γέλωτι & 405 \\
& κἀπʼ εὐτελείᾳ τόδε τὸ σανδαλίσκον &  \\
& καὶ τὸ ῥάκος, &  \\
& κἀξηῦρες ὥστʼ ἀζημίους &  \\
& παίζειν τε καὶ χορεύειν. &  \\
& Ἴακχε φιλοχορευτὰ συμπρόπεμπέ με. & 410 \\

\end{tabularx}
\end{spacing}

\newpage
\begin{multicols}{2}
\small % roughly 9pt
\vocabentry{ἄνευ}{without (+ gen.)}
\vocabentry{ἀ-ζήμιος, ον}{free from further payment: without loss, scot-free}
\vocabentry{γέλοιος, α, ον}{amusing, funny < γέλως}
\vocabentry{γέλως, ὁ}{laughter}
\vocabentry{ἐξ-ευρίσκω}{to find out, discover}
\vocabentry{εἶα}{on! up! away! come then! well now!}
\vocabentry{εὐτέλεια, ἡ}{cheapness}
\vocabentry{ἡδύς, ή, όν}{sweet}
\vocabentry{κατα-σχίζω}{to cleave asunder, split up}
\vocabentry{παρα-καλέω}{to call to; call in, send for, summon}
\vocabentry{πόνος, ὁ}{work}
\vocabentry{ῥάκος, εος, τό}{a ragged garment, a rag}
\vocabentry{σανδαλίσκος, ὁ}{dim. σάνδαλον, τό sandal}
\vocabentry{σκώπτω}{to hoot, mock, jeer, scoff at}
\vocabentry{σπουδαῖος, η, ον}{earnest, serious}
\vocabentry{συμ-προ-πέμπω}{to join in escorting}
\vocabentry{συν-ακολουθέω}{to follow closely, to accompany}
\vocabentry{συν-έμπορος, ὁ, ἡ (Att. ξυν-)}{a fellow-traveler, companion < ἔμπορος traveler, merchant}
\vocabentry{ταινιόω}{to bind with a head-band, esp. as a victor < ταινία headband}
\vocabentry{φιλο-χορευτής, ὁ}{friend of the choral dance}
\vocabentry{ᾠδή, ἡ}{a song, lay, ode}
\vocabentry{ὡραῖος, η, ον}{produced at the right season; (LSJ III.2) in the prime of life, youthful}
\end{multicols}

% vertical line across page separating vocab and notes
\vspace{-1.5em}
\noindent\rule{\linewidth}{0.4pt}
\vspace{-2em}

\begin{multicols}{2}
\begin{parcolumns}[colwidths={1=1.5em, 2=0.9\linewidth}]{2}
\small

% put notes here
\glossline{391}{%
 \glossmix{εἰπεῖν}{Likewise sc. δός.}
}

\glossline{393}{%
 \glossmix{τῆς σῆς ἑορτῆς ἀξίως}{\textit{worthily of your festival}}
}

\glossline{394}{%
 \glossmix{νικήσαντα}{More choral self-referentiality.}
 \glossmix{παίσαντα καὶ σκώψαντα νικήσαντα ταινιοῦσθαι.}{καὶ joins the two first two participles, and perhaps νικήσαντα is closer with τανιοῦσθαι. \textit{since we have played and mocked, [grant that I] win and be crowned victor}. }
}

\glossline{396}{%
 \glossmix{τὸν ὡραῖον θεὸν}{I.e. Iacchus, cf. l. 399.}
}

\glossline{397}{%
 \glossmix{ἄγʼ εἶα}{ἄγε. \textit{Come, well now!}. Extrametrical interjection, almost meaningless.}
}

\glossline{400}{%
 \glossmix{εὑρών}{Aor. εὑρίσκω.\textit{having invented most pleasant song of festival}. Or maybe (my own reading???) \textit{having found [our] song sweetest in the festival}, i.e. when we have won.}
}

\glossline{401}{%
 \glossmix{πρὸς τὴν θεὸν}{\textit{to the goddess}, presumably Demeter}
}

\glossline{404}{%
 \glossmix{φιλοχορευτὰ}{Voc.}
}

\glossline{405}{%
 \glossmix{κατεσχίσω}{2s mid. aor. κατασχίζω. Generally difficult. \textit{For you split, for the sake of laughter and cheapness, my little sandal and rag}. Evidently initiates to the Eleusinian Mysteries wore and dedicated old clothing, cf. \textit{Plut} 842-6. Alternatively possibly the chorus is complaining about the poor quality of their costumes and the poor funding for their performance (Stanford).}
}

\glossline{408}{%
 \glossmix{κἀξηῦρες}{καὶ ἐξηῦρες. 2s aor. ἐξευρίσκω.}\\
 \glossmix{ὥστʼ}{ὥστ(ε) + inf. expresses natural result (Smyth 2258). \textit{so that [we] without payment play...}}
}

\end{parcolumns}
\end{multicols}

\newpage
\begin{spacing}{1.5}
\begin{tabularx}{\textwidth}{@{}lXr@{}}
\textit{Χο.} & καὶ γὰρ παραβλέψας τι μειρακίσκης &  \\
& νῦν δὴ κατεῖδον καὶ μάλʼ εὐπροσώπου &  \\
& συμπαιστρίας &  \\
& χιτωνίου παραρραγέν- &  \\
& τος τιτθίον προκύψαν. & 415 \\
& Ἴακχε φιλοχορευτὰ συμπρόπεμπέ με. &  \\
\textit{Δι.} & ἐγὼ δʼ ἀεί πως φιλακόλου- &  \\
& θός εἰμι καὶ μετʼ αὐτῆς &  \\
& παίζων χορεύειν βούλομαι. &  \\
\textit{Ξα.} & κἄγωγε πρός. &  \\
\textit{Χο.} & βούλεσθε δῆτα κοινῇ & 420 \\
& σκώψωμεν Ἀρχέδημον; &  \\
& ὃς ἑπτέτης ὢν οὐκ ἔφυσε φράτερας. &  \\
\textit{Χο.} & νυνὶ δὲ δημαγωγεῖ &  \\
& ἐν τοῖς ἄνω νεκροῖσι, &  \\
& κἀστὶν τὰ πρῶτα τῆς ἐκεῖ μοχθηρίας. & 425 \\
\textit{Χο.} & τὸν Κλεισθένους δʼ ἀκούω &  \\
& ἐν ταῖς ταφαῖσι πρωκτὸν &  \\
& τίλλειν ἑαυτοῦ καὶ σπαράττειν τὰς γνάθους· &  \\

\end{tabularx}
\end{spacing}

\newpage
\begin{multicols}{2}
\small % roughly 9pt
\vocabentry{ἄνω}{up, upwards}
\vocabentry{Ἀρχέδημος, ὁ}{Archedemus, Athenian demagogue. ὁ τοῦ δήμου τότε προεστηκώς in 406 (Xen. HG 1.7.2)}
\vocabentry{γνάθος, ἡ}{the jaw; cheek}
\vocabentry{δημ-αγωγέω}{to lead the people}
\vocabentry{ἐκεῖ}{(adv.) there, in that place}
\vocabentry{ἑπτ-έτης, ές}{seven years old}
\vocabentry{εὐ-πρόσωπος, ον}{fair of face}
\vocabentry{κατ-εῖδον}{to look down; to view, see}
\vocabentry{Κλεισθένης, ους, ὁ}{Kleisthenes, frequently attacked in comedy for his alleged effeminacy (cf. l. 48)}
\vocabentry{κοινός, ή, όν}{common, shared in common}
\vocabentry{μειρακίσκη, ἡ}{a little girl}
\vocabentry{μοχθηρία, ἡ}{bad condition, badness; wickedness, depravity}
\vocabentry{παρα-βλέπω}{to look aside, take a side look}
\vocabentry{παρα-ρρήγνυμι}{to break at the side}
\vocabentry{προ-κύπτω}{to stoop and bend forward, to peep out}
\vocabentry{πρωκτός, ὁ}{the anus}
\vocabentry{πως}{somehow, in some way}
\vocabentry{σκώπτω}{to hoot, mock, jeer, scoff at}
\vocabentry{σπαράσσω}{to tear, rend in pieces, mangle}
\vocabentry{συμ-παιστής, οῦ, ὁ (fem. συμ-παίστρια)}{playmate}
\vocabentry{συμ-προ-πέμπω}{to join in escorting}
\vocabentry{ταφή, ἡ}{burial; burial-place, grave}
\vocabentry{τίλλω}{to pluck, pull out (hair) (+ acc. of body part from which hair is plucked)}
\vocabentry{τιτθίον, τό}{a woman’s breast (dim. of τιτθός)}
\vocabentry{φιλ-ακόλουθος, ον}{readily following}
\vocabentry{φιλο-χορευτής, ὁ}{friend of the choral dance}
\vocabentry{φράτηρ, ερος, ὁ}{member of a φράτρα, clansman}
\vocabentry{χιτώνιον, τό}{a woman's frock (dim. χιτών)}
\end{multicols}

% vertical line across page separating vocab and notes
\vspace{-1.5em}
\noindent\rule{\linewidth}{0.4pt}
\vspace{-2em}

\begin{multicols}{2}
\begin{parcolumns}[colwidths={1=1.5em, 2=0.9\linewidth}]{2}
\small

% put notes here
\glossline{411}{%
 \glossmix{καὶ γὰρ}{\textit{Yes, and} (GP 109).}\\
 \glossmix{παραβλέψας}{Aor. pple. παραβλέπω. \textit{having taken a side glance}}
}

\glossline{414}{%
 \glossmix{χιτωνίου παραρραγέντος}{Aor. pass. pple. παραρρήγνυμι. Gen. abs. \textit{since her dress burst open (lit. was broken) at the side}}
}

\glossline{415}{%
 \glossmix{προκύψαν}{Aor. neut. pple. προκύπτω agreeing with τιτθίον.}\\
 \glossmix{}{the whole sentence: \textit{yes, and, having taken a side glance just then I saw something of a young girl, a very pretty playmate, since her dress burst open at the side, -- [I saw] her little breast peeping out}}
}

\glossline{419}{%
 \glossmix{κἄγωγε}{καὶ ἔγωγε}\\
 \glossmix{πρός}{(adv.) also, besides (LSJ D)}
}

\glossline{420}{%
 \glossmix{βούλεσθε}{Cf. 127n.}
}

\glossline{422}{%
 \glossmix{ἑπτέτης ὢν}{Concessive (Smyth 2066). \textit{although seven years old}}\\
 \glossmix{φράτερας}{\textit{has not grown his citizen-teeth}. Two key points:
  \begin{itemize}[leftmargin=*]
  \item "The Athenians had no central citizen register and citizens were listed in their phratries and in their demes. A male citizen was entered in his father’s phratry when he was three to four years old..." (Hansen 1991: 96).
  \item Σ: ἀντὶ τοῦ εἰπεῖν ὀδόντας φραστῆρας, φράτορας
εἶπε. Evidently adult teeth were called ὀδόντας φραστῆρας, and Ar. here makes a pun.
\end{itemize}
 }
}

\glossline{424}{%
 \glossmix{ἐν τοῖς ἄνω νεκροῖσι}{\textit{among the corpses above}, a comic reversal of "the dead below}
}

\glossline{425}{%
 \glossmix{τὰ πρῶτα}{"as Subst. in neut. pl. πρῶτα, τά... \textit{first, highest} (in degree) (LSJ s.v. πρῶτος B.II.3)}
}

\glossline{426}{%
 \glossmix{τὸν Κλεισθένους}{Κλεισθένους is gen. \textit{the [son] of Kleisthenes}. This son is otherwise unknown; on the father cf. l. 48. Pulling hair and mangling the cheeks are gestures of mourning the dead, though pulling hair from the butt is also a joke on anal sex.}
}

\end{parcolumns}
\end{multicols}

\newpage
\begin{spacing}{1.5}
\begin{tabularx}{\textwidth}{@{}lXr@{}}
\textit{Χο.} & κἀκόπτετʼ ἐγκεκυφώς, &  \\
& κἄκλαε κἀκεκράγει & 430 \\
& Σεβῖνον ὅστις ἐστὶν Ἁναφλύστιος. &  \\
\textit{Χο.} & καὶ Καλλίαν γέ φασι &  \\
& τοῦτον τὸν Ἱπποβίνου &  \\
& κύσθου λεοντῆν ναυμαχεῖν ἐνημμένον. &  \\
\textit{Δι.} & ἔχοιτʼ ἂν οὖν φράσαι νῷν & 435 \\
& Πλούτωνʼ ὅπου ʼνθάδʼ οἰκεῖ; &  \\
& ξένω γάρ ἐσμεν ἀρτίως ἀφιγμένω. &  \\
\textit{Χο.} & μηδὲν μακρὰν ἀπέλθῃς, &  \\
& μηδʼ αὖθις ἐπανέρῃ με, &  \\
& ἀλλʼ ἴσθʼ ἐπʼ αὐτὴν θύραν ἀφιγμένος. & 440 \\
\textit{Δι.} & αἴροιʼ ἂν αὖθις ὦ παῖ. &  \\
\textit{Ξα.} & τουτὶ τί ἦν τὸ πρᾶγμα; &  \\
& ἀλλʼ ἢ Διὸς Κόρινθος ἐν τοῖς στρώμασιν. &  \\
\textit{Χο.} & χωρεῖτε &  \\
& νῦν ἱερὸν ἀνὰ κύκλον θεᾶς, ἀνθοφόρον ἀνʼ ἄλσος & 445 \\
& παίζοντες οἷς μετουσία θεοφιλοῦς ἑορτῆς· &  \\
& ἐγὼ δὲ σὺν ταῖσιν κόραις εἶμι καὶ γυναιξίν, &  \\
& οὗ παννυχίζουσιν θεᾷ, φέγγος ἱερὸν οἴσων. &  \\

\end{tabularx}
\end{spacing}

\newpage
\begin{multicols}{2}
\small % roughly 9pt
\vocabentry{ἄλσος, εος, τό}{a glade}
\vocabentry{Ἀναφλύστιος -ου, ὁ}{< Ἀνάφλυστος, Attic deme on the coast near modern Anavyssos}
\vocabentry{ἀρτίως}{= ἄρτι, just now}
\vocabentry{ἀνθοφόρος, ον}{bearing flowers, flowery}
\vocabentry{ἀπ-έρχομαι}{to go away, depart from}
\vocabentry{ἐγ-κύπτω}{to stoop down}
\vocabentry{ἐν-άπτω}{to bind on, to kindle; to be fitted with, clad in (+ acc.)}
\vocabentry{ἐπ-αν-έρομαι}{to question again and again}
\vocabentry{θεο-φιλής, ές}{dear to the gods, highly favoured}
\vocabentry{Ἱππόβινος, ὁ}{}
\vocabentry{Καλλίας, ὁ}{Callias, "son of Hipponikos, a very wealthy and distinguished Athenian of the late fifth century and a patron of intellectuals" (Dover)}
\vocabentry{κλαίω (Att. κλάω)}{to cry}
\vocabentry{κόρη, ἡ}{a maiden, maid; pupil of the eye}
\vocabentry{Κόρινθος, ἡ}{Corinth}
\vocabentry{κράζω}{to croak (freq. in pf. with pres. sense)}
\vocabentry{κύκλος, ὁ}{circle}
\vocabentry{κύσθος, ὁ}{pudenda muliebria, cunt}
\vocabentry{λεοντέη, ἡ}{a lion's skin}
\vocabentry{μακράν}{(adv.) a long way, far, far away; a long time, at length}
\vocabentry{μετ-ουσία, ἡ}{participation, partnership, communion}
\vocabentry{ὅπου}{(adv.) where}
\vocabentry{παν-νυχίζω}{to celebrate a night festival, keep vigil}
\vocabentry{Σεβῖνος, ὁ}{}
\vocabentry{φέγγος, εος, τό}{light, splendour, lustre}
\end{multicols}

% vertical line across page separating vocab and notes
\vspace{-1.5em}
\noindent\rule{\linewidth}{0.4pt}
\vspace{-2em}

\begin{multicols}{2}
\begin{parcolumns}[colwidths={1=1.5em, 2=0.9\linewidth}]{2}
\small

% put notes here
\glossline{429}{%
 \glossmix{κἀκόπτετʼ}{καὶ ἐκόπτετο. \textit{and he was beating himself [in grief]} (LSJ s.v. κόπτω II)}\\
 \glossmix{ἐγκεκυφώς}{pf. pple. ἐγκύπτω. \textit{having bent down}. Another joke playing on grief and anal sex.}
}

\glossline{430}{%
 \glossmix{κἄκλαε}{καὶ ἔκλαε}\\
 \glossmix{κἀκεκράγει}{καὶ ἐκεκράγει. plpf. κράζω. This verb frequently appears as a perfect that is equivalent to a present, and hence the pluperfect is equivalent to an imperfect.}
}

\glossline{431}{%
 \glossmix{Σεβῖνον...}{\textit{Sebinus from Anaphlystus, whoever that is}. Probably invented foreign-sounding name, playing on βῖνειν fuck. On ὅστις cf. 39n.}
}

\glossline{432}{%
 \glossmix{καὶ Καλλίαν...}{\textit{And they say that famous (τοῦτον) Kallias, the son of Horse-fucker, fights naval battles clad in a pussy-skin}. "Herakles conquered the lion of Nemea, and Kallias wears a suitable trophy of his own 'conquests'" (Dover)}
}

\glossline{433}{%
 \glossmix{τὸν Ἱπποβίνου}{Like Σεβῖνον a play on βῖνειν fuck. K's father's actual name was Hipponikos.}
}

\glossline{434}{%
 \glossmix{ἐνημμένον}{Pf. pass. pple. ἐνάπτω.}
}

\glossline{435}{%
 \glossmix{Ἔχοιτ' ἂν}{Ἔχοιτε ἂν. Potential opt.}\\
 \glossmix{νῷν}{dual, cf. 277n}
}

\glossline{436}{%
 \glossmix{Πλούτωνʼ ὅπου ʼνθάδʼ οἰκεῖ;}{\textit{Pluto, where he lives here?}}
}

\glossline{437}{%
 \glossmix{ἀφιγμένω}{Pf. pple. dual ἀφικνέομαι}
}

\glossline{438}{%
 \glossmix{ἀπέλθῃς}{Prohibitative subj. \textit{Don't go far away}}
}

\glossline{440}{%
 \glossmix{ἰσθι}{Impv. οἶδα. \textit{Know that you have arrived...}}
}


\end{parcolumns}
\end{multicols}

\newpage
\begin{spacing}{1.5}
\begin{tabularx}{\textwidth}{@{}lXr@{}}
\textit{Χο.} & χωρῶμεν ἐς πολυρρόδους &  \\
& λειμῶνας ἀνθεμώδεις, & 450 \\
& τὸν ἡμέτερον τρόπον &  \\
& τὸν καλλιχορώτατον &  \\
& παίζοντες, ὃν ὄλβιαι &  \\
& Μοῖραι ξυνάγουσιν. &  \\
\textit{Χο.} & μόνοις γὰρ ἡμῖν ἥλιος & 455 \\
& καὶ φέγγος ἱλαρόν ἐστιν, &  \\
& ὅσοι μεμυήμεθʼ εὐ- &  \\
& σεβῆ τε διήγομεν &  \\
& τρόπον περὶ τοὺς ξένους &  \\
& καὶ τοὺς ἰδιώτας. &  \\
\textit{Δι.} & ἄγε δὴ τίνα τρόπον τὴν θύραν κόψω; τίνα; & 460 \\
& πῶς ἐνθάδʼ ἄρα κόπτουσιν οὑπιχώριοι; &  \\
\textit{Ξα.} & οὐ μὴ διατρίψεις, ἀλλὰ γεύσει τῆς θύρας, &  \\
& καθʼ Ἡρακλέα τὸ σχῆμα καὶ τὸ λῆμʼ ἔχων. &  \\
\textit{Δι.} & παῖ παῖ. &  \\
\end{tabularx}

\noindent\textit{\MakeUppercase{\textls[200]{αιακος}}}

\begin{tabularx}{\textwidth}{@{}lXr@{}} & {\hspace{4.5em}}τίς οὗτος; &  \\
\textit{Δι.} &{\hspace{9em}}Ἡρακλῆς ὁ καρτερός. &  \\

\end{tabularx}
\end{spacing}

\newpage
\begin{multicols}{2}
\small % roughly 9pt
% put vocab here
\end{multicols}

% vertical line across page separating vocab and notes
\vspace{-1.5em}
\noindent\rule{\linewidth}{0.4pt}
\vspace{-2em}

\begin{multicols}{2}
\begin{parcolumns}[colwidths={1=1.5em, 2=0.9\linewidth}]{2}
\small

% put notes here
\glossline{insert line number here}{%
}

\end{parcolumns}
\end{multicols}

% begin template for new page -- do not copy this line
\newpage
\begin{spacing}{1.5}
\begin{tabularx}{\textwidth}{@{}lXr@{}}
%put text here

\end{tabularx}
\end{spacing}

\newpage
\begin{multicols}{2}
\small % roughly 9pt
% put vocab here
\end{multicols}

% vertical line across page separating vocab and notes
\vspace{-1.5em}
\noindent\rule{\linewidth}{0.4pt}
\vspace{-2em}

\begin{multicols}{2}
\begin{parcolumns}[colwidths={1=1.5em, 2=0.9\linewidth}]{2}
\small

% put notes here
\glossline{insert line number here}{%
}

\end{parcolumns}
\end{multicols}
%end template for new page -- do not copy this line

\newpage

% end of text
\end{greek}
\end{document}
